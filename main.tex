\documentclass[11pt, openany]{book}

\usepackage{geometry}
\geometry{a4paper, centering, scale=0.8}

\usepackage{amsmath}
\usepackage{amssymb}
\usepackage[version=3]{mhchem}
\usepackage{graphicx}
\usepackage[UTF8]{ctex}
\usepackage{fontspec}
\usepackage{setspace}
\usepackage{bm}
\usepackage{cite}
\usepackage{braket}
% 定义一些命令,可以方便地输入一些记号
\newtheorem{law}{定律}
\newtheorem{ded}{推论}
\newtheorem{thm}{定理}
\newtheorem{asg}{作业}
\def\dd{\mathrm{d}}
\def\ee{\mathrm{e}}
\def\ii{\mathrm{i}}
\def\dps{\displaystyle}
\renewcommand{\d}{\mathrm{d}}
\newcommand{\e}{\mathrm{e}}
\renewcommand{\i}{\mathrm{i}}
\newcommand{\mm}[1]{\mathrm{#1}}
\newcommand{\mr}[1]{\mathrm{#1}}
\newcommand{\mb}[1]{\mathbf{#1}}
\newcommand{\mc}[1]{\mathcal{#1}}
\newcommand{\tr}[1]{\textrm{#1}}
\newcommand{\Tr}[1]{\textrm{Tr}\left[#1\right]}
\newcommand{\dv}[2]{\frac{\dd{#1}}{\dd{#2}}}
\newcommand{\pdv}[2]{\frac{\partial{#1}}{\partial{#2}}}
\newcommand{\splitline}{\noindent\hfil\ \hfil$*$\hfil$*$\hfil$*$\hfil\ \hfil}
\def\degree{$^{\circ}$}

% 字体设置
%\setmainfont{Times New Roman}
% 如果不是Mac系统请注释掉下面的两行
%\setsansfont{Helvetica}
%\setCJKsansfont{STHeitiSC-Medium}

\usepackage{makecell}
\newcommand{\addcell}[2][4]{\makecell{\zihao{#1}\textsf{#2}}}
\usepackage{titlesec}
\usepackage{booktabs}

% 取消首行缩进
%\setlength\parindent{0 em}
% 采用超链接
\usepackage{hyperref}
%\hypersetup{hidelinks}
% 采用颜色
\usepackage{color,xcolor}
% 脚注编号改为圈+数字
\renewcommand\thefootnote{\raisebox{.5pt}{\textcircled{\raisebox{-.9pt}{\arabic{footnote}}}}}

% 设置图注、表注
\usepackage{caption}
\usepackage{bicaption}
\captionsetup{labelsep=quad, font={small, bf}, skip=2pt}
\renewcommand\figurename{图}
\renewcommand\tablename{表}
\DeclareCaptionOption{english}[]{
    \renewcommand\figurename{图}
    \renewcommand\tablename{表}
}
\captionsetup[bi-second]{english}


\usepackage[sectionbib]{chapterbib}

\begin{document}
    \title{\Huge \heiti 化学中的数学}
    \author{\kaishu 刘剑 \\ \kaishu 蒋然 \ \kaishu 王崇斌 \ \kaishu 文亦质 \\ \kaishu 薛炜之}
    \maketitle
    \tableofcontents

    % 设置段间距
    \setlength\parskip{0.62 em}
    \mainmatter

    \chapter{数学准备}

    在开始本书的阅读之前,我们希望读者已经学过了高等数学(或称“微积分”)和线性代数课程,
    能够熟练进行简单的积分计算。

    \section{Gauss积分及其拓展}
        回顾一维Gauss积分
        \footnote{事实上, 这种类型的积分是Euler最先通过将两个高斯函数相乘, 
        再进行极坐标变换算出来的, 按理来说应该称为“Euler积分”。}
        的计算:
        \begin{equation}
            \begin{aligned}
                I &= \int_0^{+\infty} \mathrm{e}^{-ax^2} x^{n} \mathrm{d}x
            \end{aligned}
        \end{equation}
        令$t = ax^2$, 则$\mathrm{d}t = 2ax\mathrm{d}x$
        所以
        \begin{equation}
            \begin{aligned}
                I &= \int_0^{+\infty} \mathrm{e}^{-t} \bigg(\frac ta\bigg)^{\frac n2} \frac {\mathrm{d}t}{\sqrt{at}}\\
                &= \frac 1{2a^{\frac {n+1}2}} \int_0^{+\infty} \mathrm{e}^{-t} t^{\frac {n-1}2} \mathrm{d}t\\
                &= \frac {\Gamma(\frac {n+1}2)}{2a^{\frac {n+1}2}}
            \end{aligned}
        \end{equation}

        利用一维Gauss积分的计算结果可以计算多维的Gauss积分. 
        先举一个最简单的例子,其中应用的思想可以用于计算很多不同种类的Gauss
        积分:
        \begin{equation}
            I = \int \ee^{-\bm{x}^\mathrm{t}\mb{A}\bm{x}}\dd \bm{x}
        \end{equation}
        其中$\bm{A}$是正定实对称矩阵
        \footnote{
            这里必须要求$\mb{A}$是正定的,否则这个Gauss积分不收敛。
        }
        ,根据线性代数中的定理,$\mb{A}$可以在正交
        变换下对角化:
        \begin{equation}
            \mb{D} = \mb{S}^{\mr{t}}\mb{AS}
        \end{equation}
        其中$\mb{S}$为正交矩阵,那么原积分可以表示为:
        \begin{equation}
            \begin{split}
                I &= \int\ee^{-(\mb{S}\bm{x})^{\mr{t}}\mb{S}^{\mr{t}}\mb{AS}\mb{S}^{\mr{t}}\bm{x}}\dd \bm{x}\\
                &= \int\ee^{-(\mb{S}\bm{x})^{\mr{t}}\mb{D}\mb{S}^{\mr{t}}\bm{x}} \dd \bm{x}
            \end{split}
        \end{equation}
        做变量替换$\bm{y} = \mb{S}^{\mr{t}}\bm{x}$, 利用重积分的换元:
        \begin{equation}
            \begin{split}
                I &= \int\ee^{-\bm{y}^{\mr{t}}\mb{D}\bm{y}}\left|\pdv{\bm{x}}{\bm{y}}\right|\dd \bm{y}\\
                &= \int\ee^{-\bm{y}^{\mr{t}}\mb{D}\bm{y}}\left|\mb{S}\right|\dd \bm{y}\\
                &= \prod_{j=1}^{N}\int_{-\infty}^{+\infty}\ee^{-y_j^2D_{jj}}\dd y_j\\
                &= \prod_{j=1}^{N}\sqrt{\frac{\pi}{D_{jj}}}\\
                &= \frac{\pi^{\frac{N}{2}}}{\sqrt{\det{\mb{A}}}}
            \end{split}
        \end{equation}
        稍微复杂点的一个例子, Gauss函数与二次型相乘: 
        \begin{equation}
            I = \int \bm{x}^\mathrm{t}\mb{B}\bm{x} \mathrm{e}^{-\bm{x}^\mathrm{t}\mb{A}\bm{x}}\mathrm{d}\bm{x}
        \end{equation}
        其中$\mb{A}$为正定实对称矩阵, $\mb{B}$为实对称矩阵
        \footnote{
            任意二次型均可以表示为$\bm{x}^{\mr{t}}\mb{B}\bm{x}$, 其中$\mb{B}$为对称矩阵.
        }
        . 仿照上一个例子将$\bm{A}$对角化, 得到:
        \begin{equation}
            \mb{S}^\mathrm{t} \mb{AS} = \mb{D}
        \end{equation}
        所以:
        \begin{equation}
            I = \int \bm{x}^\mathrm{T}\mb{B}\bm{x} \mathrm{e}^{-\bm{x}^\mathrm{t}\mb{SDS}^\mathrm{t}\bm{x}}\mathrm{d}\bm{x}
        \end{equation}
        对重积分进行换元$\bm{y} = \mb{S}^\mathrm{t}\bm{x}$: 
        \begin{equation}
            \mathrm{d}\bm{y} = \det\mb{S}^\mathrm{t}\mathrm{d}\bm{x} = \mathrm{d}\bm{x}
        \end{equation}
        原积分化为
        \begin{equation}
            \begin{aligned}
                I = \int \bm{y}^\mathrm{t}\mb{S}^\mathrm{t}\mb{BS}\bm{y}
                 \mathrm{e}^{-\bm{y}^\mathrm{t}\mb{D}\bm{y}}\mathrm{d}\bm{y}
            \end{aligned}
            \label{change variable}
        \end{equation}
        令$\mb{E} = \mb{S}^\mathrm{t}\mb{BS}$, 于是
        \begin{equation}
            \begin{aligned}
                I &= \int \bm{y}^\mathrm{t}\mb{E}\bm{y} \mathrm{e}^{-\bm{y}^\mathrm{t}\bm{D}\bm{y}}\mathrm{d}\bm{y}\\
                &= \int \sum_{i=1}^{N} \sum_{j=1}^{N} E_{ij} y_i y_j \mathrm{e}^{-\sum_{k=1}^{N} D_{kk} y_k^2} \prod_{k=1}^{N} \mathrm{d}y_k
            \end{aligned}
        \end{equation}
        可以将求和号放到积分外面, 可以分别计算每一个积分, 显然$i \neq j$的项都
        是奇函数对全空间的积分, 得到的结果为0, 只有$i=j$的项会有贡献. 于是积分化简为
        \footnote{请仔细验证一下}:
        \begin{equation}
            \begin{aligned}
                I &= \int \sum_{i=1}^N E_{ii} y_i^2 \mathrm{e}^{-\sum_{k=1}^{N} D_{kk} y_k^2} \prod_{k=1}^{N} \mathrm{d}y_k\\
                &= \prod_{k=1}^{N} \sqrt{\frac {\pi}{D_{kk}}} \sum_{i=1}^{N} \frac {E_{ii}}{2D_{ii}}\\
                &= \frac {\pi^{\frac N2}}{2\sqrt{\det{\mb{D}}}} \mathrm{Tr} (\mb{ED}^{-1})
            \end{aligned}
        \end{equation}
        利用迹的性质进一步化简:
        \begin{equation}
            \begin{aligned}
                \det{\mb{D}} = \det{\mb{S}^{\mr{t}}\mb{A}\mb{S}} = \det{\mb{SS}^{\mr{t}}\mb{A}} = \det{\mb{A}}
            \end{aligned}
        \end{equation}
        类似的: 
        \begin{equation}
            \begin{aligned}
                \mathrm{Tr}(\mb{ED}^{-1}) = \mathrm{Tr}(\mb{S}^\mathrm{t}\mb{BS}\mb{S}^\mathrm{t} \mb{A}^{-1}\mb{T})
                = \mathrm{Tr}(\mb{S}^\mathrm{t}\mb{BA}^{-1}\mb{S})
                = \mathrm{Tr}(\mb{BA}^{-1})
            \end{aligned}
        \end{equation}
        所以:
        \begin{equation}
            \begin{aligned}
                I = \frac {\pi^{\frac N2}}{2\sqrt{\det{\mb{A}}}} \mathrm{Tr} (\mb{BA}^{-1})
            \end{aligned}
        \end{equation}

        接下来尝试计算
        \footnote{注意这里是对于一个矩阵的积分,很容易理解,这相当于对每一个矩阵元积分}
        :
        \begin{equation}
            \begin{aligned}
                \mb{I} = \int \bm{xx}^\mathrm{T} \mathrm{e}^{-\bm{x}^\mathrm{t}\mb{A}\bm{x}}\mathrm{d}\bm{x}
            \end{aligned}
        \end{equation}
        用相同的换元方法得到:
        \begin{equation}
            \begin{aligned}
                \mb{I} = \int \mb{S}\mb{y}\bm{y}^\mathrm{t}\mb{S}^\mathrm{t} \mathrm{e}^{-\bm{y}^\mathrm{t}\mb{D}\bm{y}}\mathrm{d}\bm{y}
            \end{aligned}
        \end{equation}
        计算每个元素: 
        \begin{equation}
            \begin{aligned}
                I_{ij} &= \int (\mb{S}\bm{y})_{i}(\bm{y}^\mathrm{t}\mb{S}^\mathrm{t})_{j} \mathrm{e}^{-\bm{y}^\mathrm{t}\mb{D}\bm{y}}\mathrm{d}\bm{y}\\
                &= \int\sum_{k=1}^{N} S_{ik}y_k \sum_{l=1}^{N} S_{jl} y_l \mathrm{e}^{-\bm{y}^\mathrm{t}\mb{D}\bm{y}}\mathrm{d}\bm{y}\\
                &= \int \sum_{k,l=1}^{N} S_{ik} S_{jl} y_k y_l \mathrm{e}^{-\bm{y}^\mathrm{t}\mb{D}\bm{y}}\mathrm{d}\bm{y}
            \end{aligned}
        \end{equation}
        同样, 只有在$k=l$的项对积分有贡献才有贡献, 那么:
        \begin{equation}
            \begin{aligned}
                I_{ij} &= \int \sum_{k=1}^{N} S_{ik} S_{jk}y_k^2 \mathrm{e}^{-\bm{y}^\mathrm{t}\mb{D}\bm{y}}\mathrm{d}\bm{y}\\
                &= \prod_{l=1}^{N} \sqrt{\frac {\pi}{D_{ll}}} \sum_{k=1}^N S_{ik} S_{jk} \frac 1{2D_{kk}}\\
                &= \frac {\pi^{\frac N2}}{2 \sqrt{\det{\mb{D}}}} (\mb{SD}^{-1}\mb{S}^\mathrm{t})_{ij}\\
                &= \frac {\pi^{\frac N2}}{2 \sqrt{\det{\mb{A}}}} \mb{A}^{-1}_{ij}
            \end{aligned}
        \end{equation}
        因此: 
        \begin{equation}
            \begin{aligned}
                \bm{I} = \frac {\pi^{\frac n2}}{2 \sqrt{\det{\bm{A}}}} \bm{A}^{-1}
            \end{aligned}
        \end{equation}
        利用之前的计算结果,可以得到一个有趣的结论:
        \begin{equation}
            \frac{\int\ee^{-\bm{x}^{\mr{t}}\mb{A}\bm{x}}(\bm{x}\bm{x}^{\mr{t}})\dd \bm{x}}{\int\ee^{-\bm{x}^{\mr{t}}\mb{A}\bm{x}}}\dd \bm{x}
            = \frac{1}{2}\mb{A}^{-1}
        \end{equation}
        最为普遍的一类Gauss积分可以定义为:
        \begin{equation}
            I = \int\ee^{-\bm{x}^{\mr{t}}\mb{A}\bm{x}}g(\bm{x})\dd \bm{x}
        \end{equation}
        其中$g(\bm{x})$可以取
        \footnote{对于最一般的$g(\bm{x})$,此积分也可以计算,感兴趣可以查阅相关资料}
        :
        \begin{equation}
            g(\bm{x}) = 
            \left\{
            \begin{split}
                    &\exp\left[-(\bm{x} - \bm{x_0})^{\mr{t}}\mb{C}(\bm{x} - \bm{x_0})\right]\\
                    &\exp\left[-V_0^{\mr{t}}(\bm{x} - \bm{x_0})\right]\\
                    &(\bm{x}^{\mr{t}}\mb{B}\bm{x})\cdot(\bm{x}^{\mr{t}}\mb{D}\bm{x})
            \end{split}
            \right.
        \end{equation}
        \begin{asg}
            计算正文中提到的Gauss积分。
        \end{asg}

        
        \bibliographystyle{plain}
        \bibliography{ref_mathprep}
    \chapter{Hamilton力学}
    \section{牛顿运动方程}
    \subsection{牛顿运动方程及保守系统}
    \par
    这里抛开经典力学的时空观和经典力学的相对性原理(即经典力学中的物理规律在
    伽利略变换下不变)不谈,关注经典力学的另一个特征——决定性.实验事实(指一定精度下
    的实验,完全有可能被更为精确的实验所推翻)告诉我们对于一个封闭的力学系统,
    其初始位置$\bm{x}(t_0)$和初始速度$\dot{\bm{x}}(t_0)$的情况下可以唯一确定这个
    系统今后的运动状态.
    \par 既然对于一个力学系统其初始位置和初始速度可以决定其运动状态,那么它们也决定了系统
    任意时刻的加速度,即存在一个函数$\bm{F}$使得:
    \begin{equation}
        m\ddot{\bm{x}} = \bm{F}(\bm{x}, \dot{\bm{x}}, t)
    \end{equation}
    由微分方程解的存在唯一性定理,若已知$\bm{x}(t_0),\, \dot{\bm{x}}(t_0)$与$\mb{F}$
    则上述微分方程唯一确定了一个运动.函数$F$的形式要通过实验来确定,如果确定了其形式,
    那么就知道了对应力学系统的运动方程.
    \par 如果一个力学系统的运动方程可以写为:
    \begin{equation}
        m\ddot{\bm{x}} = -\pdv{V(\bm{x})}{\bm{x}}
    \end{equation}
    这样的力学系统称为保守系统.一个完全等价的说法是,在外力场中运动的系统,外力
    对其所做的功与路径无关,只与起点和终点有关;用数学语言描述,对于位形空间中的任何一条
    闭合的光滑曲线$C$,下式成立:
    \begin{equation}
        \int_{C}\bm{F}\cdot\mathrm{d} \bm{r} = 0
    \end{equation}
    可以证明,若上述条件满足,那么存在一个函数$U(\bm{x})$使得$\bm{F} = -\nabla U(\bm{x})$,
    这样的系统为保守系统.一般而言,我们讨论的系统都属于保守系统,比如重力场、中心力场等.
    \par 
    保守系统中的势能函数很大程度上决定了系统的性质,其中最基本的是势能的对称性确保了力学
    系统中的一些守恒量.下面仅用势能的时间平移不变性(势函数不显含时间)来说明系统的能量
    守恒.能量被定义为:
    \begin{equation}
        E = T + V = \frac{1}{2}m\dot{\bm{x}}^2 + V(x)
    \end{equation}
    考虑$E$对时间的全导数,即考虑一个真实路径$\bm{x}(t)$上$E$对时间的导数(带入
    运动方程):
    \begin{equation}
        \begin{split}
            \dv{E}{t} & = m\ddot{\bm{x}}\dot{\bm{x}} + \dv{V(\bm{x}(t))}{t}\\
            & = - \pdv{V(\bm{x})}{\bm{x}} \cdot \dot{\bm{x}} + \pdv{V(\bm{x})}{\bm{x}}
            \cdot \dot{\bm{x}}\\
            & = 0
        \end{split}
    \end{equation}
    这就说明在运动过程中能量$E$是一个守恒量.(类似的还有势能的平移对称性对应的动量守恒和
    旋转对称性对应的角动量守恒,这提示我们势能的对称性和守恒量之间存在对应关系)
    \subsection{使用牛顿方程解决问题}
    牛顿方程是一个二阶微分方程,对于高阶微分方程,一般的研究方法是将其化为一阶微分方程组.
    在这里仅考虑一维系统,引入物理中具有重要意义的量——动量:$\bm{p} := \frac{\dot{\bm{x}}}{m}$,
    将牛顿方程转化为一个微分方程组(此时动量$\bm{p}$与位置$\bm{x}$为独立的变量):
    \begin{equation}
        \dot{x} = \frac pm\\
        \dot{p} = -\frac {\partial V}{\partial x}
    \end{equation}
    \par
    首先研究HCl分子.每个原子的坐标有3个自由度,总共是6个自由度.而这个分子总体有3个平动自由度,
    2个转动自由度,还剩余1个振动自由度.振动自由度的能量由\textbf{势能面}来描述
    \footnote{按照笔记修改者的理解,势能面应该是体系势能与坐标之间的函数关系.对于二体问题而言,
    仅用势能面来描述系统振动自由度的势能是不合适的,它忽略了转动对于振动的影响.如果严格处理这个问题,
    首先在相对位置坐标(直接采用极坐标$(r, \phi)$,其中$\mu$为折合质量)下写出能量守恒的表达式:
    \begin{equation}
        E = \frac{1}{2}\mu\left(\dot{r}^2 + r^2\dot{\phi}^2\right) + V(r)
    \end{equation}
    考虑到中心力场中角动量守恒,即$J = \mu r^2 \dot{\phi}$是一个守恒量,带入能量守恒的表达式:
    \begin{equation}
        E = \frac{1}{2}\mu\left(\dot{r}^2 + \frac{J^2}{\mu^2r^2}\right) + V(r)
    \end{equation}
    将上式对时间求导后就会得到径向运动方程:
    \begin{equation}
        \mu \ddot{r} - \frac{J^2}{\mu r^3} + \pdv{V}{r} = 0
    \end{equation}
    可以看出,只有在不考虑转动时(角动量很小或者为0)才是正文中所讨论的情况
    }
    .势能面是两个原子的距离$r$的函数,满足
    \begin{equation}
        \lim_{r \to \infty} V(r) = 0
    \end{equation}
    当$r$减小时,势能逐渐减小,有一个\textbf{极小值},对应的两原子距离称为平衡位置$r_\mathrm{eq}$, 然后再减小$r$时,势能增大,最后达到
    \begin{equation}
        \lim_{r \to 0^+} V(r) = +\infty
    \end{equation}
    这与两个原子的间距不能无穷近是一致的.
    实际上在平衡位置附近,我们把势能函数用二次函数近似\footnote{
        将势能函数在平衡位置Taylor展开,保留到二阶(除非没有二阶项)
    }(即将真实的物理系统想象成为谐振子).
    通过改变势能零点的定义,我们总可以把势能写为
    \begin{equation}
        V(r) = \frac{1}{2}k(r-r_\mathrm{eq})^2
    \end{equation}
    根据势能的形式可以写出力的形式
    \begin{equation}
        F = -\frac {\partial V}{\partial r} = -k(r-r_\mathrm{eq})
    \end{equation}
    做变换$x = r - r_\mathrm{eq}$, 可以将势能写为
    \begin{equation}
        V(x) = \frac 12 kx^2
    \end{equation}
    带入牛顿运动方程,得到关于位置和动量的微分方程组:
    \begin{equation}
        \begin{split}
            \dot{x} &= \frac pm\\
            \dot{p} &= -kx
        \end{split}
    \end{equation}
    现在求解这个运动方程:
    \begin{equation}
        \ddot{x} = \frac {\dot{p}}m = -\frac {kx}{m}
    \end{equation}
    这是一个二阶常微分方程,通解为:
    \begin{equation}
        \begin{split}
            x &= A \cos{\omega t} + B \sin{\omega t}\\
            p &= -{Am\omega} \sin{\omega t} + {Bm \omega} \cos{\omega t}
        \end{split}
    \end{equation}
    其中$\omega = \sqrt{\frac km}$. 如果给定初始条件:
    \begin{equation}
        x(0) = x_0\\
        p(0) = p_0
    \end{equation}
    将这两个方程代入到通解中,得到:
    \begin{equation}
        \begin{split}
            x &= x_0 \cos{\omega t} + \frac {p_0}{m\omega} \sin{\omega t}\\
            p &= p_0 \cos{\omega t} - {m\omega x_0} \sin{\omega t}
        \end{split}
    \end{equation}
    \subsection{遗留的一个问题}
    匀变速直线运动,应当有
    \begin{equation}\begin{aligned}
        x(t) &= x(0) + vt + \frac 12 at^2 \\
        &= x(0) + \dot{x}t + \frac 12 \ddot{x}t^2
    \end{aligned}\end{equation}
    这相当于位置对时间作了Taylor展开,展开到二阶.但是为什么只考虑前两阶,而不考虑后面的项呢?
    可以这样考虑:在给定了Hamilton函数的情形下,正则方程最多只涉及到对时间的二阶导数,
    最终解出位置对时间的函数,以及动量对时间的函数只有两个待定常数,因此只用位置和动量初始的条件.
    \footnote{
        这里给出的是笔记书写者的看法.为什么牛顿方程是二阶常微分方程,修改者认为这是由经典力学的
        \textbf{决定性}导致的.力学系统的位置和位置对于时间的导数可以唯一决定力学系统今后的状态,
        如果在位形空间中列出运动方程(Newton方程、Euler-Lagrange方程),那必然是二阶微分方程
        (此时微分方程解的存在唯一性定理与经典力学的决定性相容);如果在相空间
    }
    \section{Hamilton 正则方程}
    \subsection{Hamilton 方程的导出和性质}
    前面已经看到,我们通过定义\textbf{动量}为独立变量的方式,将一维系统的一个二阶常微分方程化为了
    两个变量组成的一阶常微分方程组.这样的方法也能推广到$n$维系统,由于$2n$个初始条件(初始坐标
    和初始速度)决定了这个系统的运动,对应的我们也需要$2n$个一阶的方程组来描述这个系统;另一个问题是
    如何选择独立的变量,自然的想法是将$n$个坐标和$n$个“动量”(严格来讲是\textbf{广义动量})选为
    独立变量,这样得到的方程组称为Hamilton正则方程.
    \par 
    关于Hamilton方程组的严格导出需要从Lagrange量和Euler-Lagrange方程出发,这里仅给出相关结论(具体
    的过程可以参考后面的章节).一般而言,系统的Hamilton函数是系统坐标、动量与时间的函数
    $H = H(\{x_i\},\{p_i\},t)$,系统的运动方程由Hamilton正则方程给出:
    \begin{equation}
        \left\{
            \begin{split}
                \dot{x_{i}} &= \pdv{H}{p_i}\\
                \dot{p_{i}} &= -\pdv{H}{x_{i}}
            \end{split}
        \right.
        \quad\quad
        i = 1, 2, 3, \cdots, n
        \label{Hamilton equation}
    \end{equation}
    这里不经证明地给出一维(可以推广到高维)保守体系体系在直角坐标系中的Hamilton函数:
    \begin{equation}
        H(x,p,t) = \frac {p^2}{2m} + V(x)
    \end{equation}
    可以在一维情形下通过牛顿方程验证\textbf{正则方程}的正确性:
    \begin{equation}
        \begin{split}
        \frac {\partial H}{\partial x} &= \frac {\partial V}{\partial x} = -\dot{p}\\
        \frac {\partial H}{\partial p} &= \frac pm = \dot{x}
        \end{split}
    \end{equation}
    现在希望验算对于Hamilton量不含时(具有时间平移对称性)的系统,在其任何一个由正则方程
    决定的路径上Hamilton量守恒,即:
    \begin{equation}
        H(x(t),p(t),t) = H(x(0),p(0),0)~~~~~\forall t
    \end{equation}
    考虑Hamilton量对于时间的导数,同时带入正则方程:
    \begin{equation}
        \frac {\mathrm{d}H}{\mathrm{d}t} = \frac {\partial H}{\partial x} \dot{x} + \frac {\partial H}{\partial p} \dot{p} + \frac {\partial H}{\partial t} = \frac {\partial H}{\partial t} = 0
    \end{equation}
    \par 
    在谐振子模型中,Hamilton函数不显含时间,故
    \begin{equation}
        \frac {\mathrm{d}H}{\mathrm{d}t} = 0
    \end{equation}
    这个体系可以在\textbf{相空间}
    \footnote{相空间就是Hamilton方程中独立变量所张成的空间}
    中描述,即把它的状态画在$(x,p)$二维平面上,观察系统的代表点随时间的运动.
    显然谐振子体系在相空间中的轨迹是一个椭圆:
    \begin{equation}
        \frac{p^2}{2m} + \frac 12 kx^2 = E_0
    \end{equation}
    其中$E_0$由初始状态决定.由于之前已经解出谐振子的运动方程,容易得到运动的周期:
    \begin{equation}
        T = \frac{2\pi}{\omega}
    \end{equation}
    \par 
    但是,对于任意的满足能量守恒的体系,其在相空间中的轨迹不一定是一条封闭的曲线(即并不是所有的运动
    都是周期的,尤其是对于高维的问题),在一些情况下有可能充满相空间的某个区域.常见的例子有中心力场
    \cite{Landau2007mechanics}、二维谐振子等\cite{B2006经典力学的数学方法},这里给出一个简单的例子.考虑一个二维的谐振子,其Hamilton量
    为:
    \begin{equation}
        H = \frac{1}{2m}(p_1^2 + p_2^2) + \frac{m}{2}(\omega_1^2x_1^2 + \omega_2^2x_2^2)
    \end{equation}
    可以解出运动方程为:
    \begin{equation}
        \begin{split}
            x_1 &= A_1\cos(\omega_1 t + \phi_1)\\
            x_2 &= A_2\cos(\omega_2 + \phi_2)
        \end{split}
    \end{equation}
    可以看出,如果$\omega_1 / \omega_2$为一个有理数,那么上面的运动(参数方程所代表的二维曲线)就是
    有周期的;如果是无理数,那么曲线应该在某个区域内是稠密的(没有周期).
    \par
    现在考虑质量是$x,p$的函数,即$m_\mathrm{eff}(x,p)$, 在这种情况下Hamilton函数为
    \begin{equation}
        H(x,p) = \frac {p^2}{2m_\mathrm{eff}(x,p)} + V(x)
    \end{equation}
    在这种情况下的运动方程为:
    \begin{equation}
        \begin{split}
            \dot{x} &= \frac {\partial H}{\partial p} = \frac {p}{2m_{\mathrm{eff}}} - \frac {p^2}{2m_{\mathrm{eff}^2}} \frac {\partial m_\mathrm{eff}}{\partial p} \\
            \dot{p} &= -\frac {\partial H}{\partial x} = \frac {p^2}{2m_\mathrm{eff}^2} \frac {\partial m_\mathrm{eff}}{\partial x} + \frac {\partial V}{\partial x}
        \end{split}
    \end{equation}
    这种情况下能量仍然守恒,因为Hamilton函数不显含时间.
    \subsection{Hamilton 方程的数值解法}
    容易想象(从上面的习题同样可以看出),对于一般的力学系统,给出运动方程的解析形式非常困难,这时候就要求
    我们通过一些其他的手段来获取运动方程的信息,一个常用的方法是数值求解.数值求解的基本思路是用有限差分
    代替微分,然后利用计算机来求解差分方程.对于同一个微分方程,可以设计不同的差分格式,它们在极限情况下
    (步长趋于0)都会回到原来的微分方程,但是在步长有限的情况下,这些差分方程对于问题的描述可能会有明显
    的差异,这里只做简单的介绍.首先考虑一般形式微分方程的初值问题:
    \begin{equation}
        \begin{split}
            \dv{x}{t} &= f(x, t) \\
            x(t_0) &= x_0
        \end{split}
    \end{equation}
    \subsubsection{Euler 法}
    考虑使用有限差分代替微分:
    \begin{equation}
        \dv{x}{t} = f(x, t) \approx \frac{x(t + h) - x(t)}{h}
    \end{equation}
    将上式改写为递推的形式:
    \begin{equation}
        x_{n+1} = x_n + h\cdot f(x_n, t_n)
    \end{equation}
    只要知道初始条件,上式可以不断递推.上面的方法称为向前欧拉法,是一种显式的单步算法.
    \footnote{
        单步:可以通过$x_n$的数值计算$x_{n+1}$的数值;显式:如果$x_n+1$只需要$x_{m<n}$的数值计算
    }
    相应的有向后欧拉法:
    \begin{equation}
        x_{n+1} = x_n + h\cdot f(x_{n+1}, t_{n+1})
    \end{equation}
    这是一个隐式的单步算法,需要在知道$f$的具体形式后从上式中反解$x_{n+1}$.
    \subsubsection{Runge-Kutta 法}
    \paragraph{二阶 Runge-Kutta 法}
    \begin{equation}
        \left\{
            \begin{split}
                k_1 &= h\cdot f(x_n, t_n)\\
                k_2 &= h\cdot f(x_n + \frac{1}{2}k_1, t_n + \frac{1}{2}h)\\
                x_{n+1} &= x_n + k_2 + O(h^3)
            \end{split}
        \right.
    \end{equation}
    \paragraph{四阶 Runge-Kutta 法}
    \begin{equation}
        \left\{
            \begin{split}
                k_1 &= h\cdot f(x_n, t_n)\\
                k_2 &= h\cdot f(x_n + \frac{1}{2}k_1, t_n + \frac{1}{2}h)\\
                k_3 &= h\cdot f(x_n + \frac{1}{2}k_2, t_n + \frac{1}{2}h)\\
                k_4 &= h\cdot f(x_n + k_3, t_n + h)\\
                y_{n + 1} &= y_n + \frac{1}{6}k_1 + \frac{1}{3}k_2 + \frac{1}{3}k_3 + \frac{1}{6}k_4 + O(h^5)
            \end{split}
        \right.
    \end{equation}
    Runge-Kutta法是一种常用的精度较高的单步算法,在同样的$t$步长下拥有比Euler法更高的精度.
    \footnote{
        关于差分格式的误差估计与稳定性分析这里无法展开讨论,应参考相关书籍
    }
    对于常微分方程组,只用将上述差分格式中的$k_i, x_n, f$改为向量即可.
    \subsubsection{velocity-Verlet 方法}
    容易想象,前几种方法求解微分方程时每一步误差都会累积,一定时间后数值解就会与真实解产生明显
    偏离.由于Hamilton方程具有比一般微分方程更加丰富的性质,这就意味着有可能存在适用于Hamilton系统
    的差分方案,它可以保持Hamilton系统中的一些守恒量
    \footnote{
        可以通过数值计算验证无论是Euler法还是Runge-Kutte法都不能保证演化过程中系统的能量稳定
    },从而在相当长时间内给出较为精确
    的数值解
    \footnote{
        当然,保证了能量守恒并不一定能保证数值解在任意时刻都可以与真实解任意接近.
    },下面给出的velocity—Verlet方法就是这样一个差分格式.这里只给出一维系统的例子,容易推广到任意维
    系统.假设系统的哈密顿量为:
    \begin{equation}
        H = \frac{p^2}{2m} + V(x)
    \end{equation}
    \begin{equation}
        \left\{
            \begin{split}
                p_{j+0.5} &= p_j - \frac{\Delta t}{2}\left.\pdv{V}{x}\right|_{x = x_j}\\
                x_{j+1} &= \frac{p_{j+0.5}}{m}\Delta t + x_{j}\\
                p_{j+1} &= p_{j+0.5} - \frac{\Delta t}{2}\left.\pdv{V}{x}\right|_{x=x_{j+1}}
            \end{split}
        \right.
    \end{equation}
    可以写程序验证,至少对于一维四次方势系统,velocity-Verlet方法给出的数值解能量是稳定的.

    \section{Hamilton力学的应用:多自由度小振动}
    \subsection{将平衡位置附近的势能函数展开为二次型}
    多自由度小振动是Hamilton力学的一个重要应用。
    
    考虑\ce{H2O}分子的振动.它总共有3个原子, 所以有9个运动自由度.质心平动3个自由度, 
    刚性转动也有3个自由度, 因此振动是3个自由度.
    \footnote{
    3个振动自由度分别为剪切振动、对称伸缩振动和不对称伸缩振动.
    水分子的O-H振动波数约为3700 cm$^{-1}$, 剪切振动波数约为1600 cm$^{-1}$, 
    伸缩振动1个周期应当约为20.8 fs, 剪切振动周期约为9 fs. 而1 a.u. = 0.024 fs.即可据此估计模拟过程中的时间步长.\\
    多原子分子的振动问题比表面上看起来更加复杂.对于刚体, 
    有三个平动自由度与三个转动自由度, 可以通过描述质心坐标与刚体的旋转
    (特殊正交矩阵描述, 3个自由度)来确定整个刚体的运动.对于分子来说, 振动和转动
    通常是耦合的, 并不能严格定义转动自由度, 但是在\textbf{小振动}的情形下, 可以分离
    平动、转动、振动自由度, 略微下详细的讨论可以参见\cite{Landau2007mechanics}, 
    这里不再展开.
    }
    对于水分子, 定义每个原子的坐标为:
    \begin{equation}
        \begin{split}
        \bm{x} = 
        \begin{pmatrix}
            \bm{x}_\mathrm{O}\\
            \bm{x}_{\mathrm{H1}}\\
            \bm{x}_{\mathrm{H2}}
        \end{pmatrix}
    \end{split}
    \end{equation}
    其中$\bm{x}_\mr{O}$表示氧原子O在三维空间中的Descartes坐标, 其他以此类推.
    给定原子核运动的势能$V(\bm{x})$, 定义质量矩阵:
    \begin{equation}
        \begin{split}
        \bm{M} = \mathrm{diag} \{m_1,...,m_9 \} = 
        \begin{pmatrix}
            m_1 & \cdots & 0\\
            \vdots & \ddots & \vdots\\
            0 & \cdots & m_9
        \end{pmatrix}
        \label{mass matrix}
    \end{split}
    \end{equation}
    其中, $m_1,m_2,m_3$等于氧原子的质量, $m_4,...,m_9$等于氢原子的质量.那么可以将
    系统的动量表示为:
    \begin{equation}
        \bm{p} = \bm{M}\cdot\dot{\bm{x}}
    \end{equation}
    仿照一维系统的Hamilton量, 可以写出这个多维系统的Hamilton量:
    \begin{equation}
        H(\bm{x}, \bm{p}) = \frac{1}{2}\bm{p}^{\mr{t}}\bm{M}^{-1}\bm{p} + V(\bm{x})
    \end{equation}
    理论上, 只要给出势能函数的形式, 就可以完全讨论系统的运动.但是对于系统在平衡位置
    \footnote{一般都是指稳定平衡位置, 即当系统偏离平衡点的距离非常微小时, 
    系统有回到平衡位置的运动趋势, 数学上对应势能函数的\textbf{极小值点}
    }附近的运动, 通常采用\textbf{小振动近似}来得到系统在平衡位置附近运动的解析表达式.
    假设平衡位置为$\bm{x_{\mr{eq}}}$, 那么在平衡点邻域内的函数值可以按照Taylor展开写为:
    \begin{equation}
        \begin{split}
        V(\bm{x_{\mr{eq}}} + \bm{q}) &= V(\bm{x_{\mr{eq}}}) + \left.\pdv{V}{\bm{x}}\right|_{\bm{x} = \bm{x}_{\mr{eq}}}^{\mr{t}}\bm{q}
         + \frac{1}{2}\bm{q}^{\mr{t}}\left.\pdv{^2V}{\bm{x}^2}\right|_{\bm{x} = \bm{x_{\mr{eq}}}}\bm{q} + o(\left|\bm{q}\right|^2)\\
        \end{split}
    \end{equation}
    在平衡位置$\dps\left.\pdv{V}{\bm{x}}\right|_{\bm{x} = \bm{x}_{\mr{eq}}}$
    为0(从数学上讲, 这是函数极小值点的性质;从物理上讲, 平衡位置处系统不受力);
    同时常数项不会影响运动方程, 可以不予考虑;如果在$\bm{q}$比较小时, 忽略2阶以上的项
    \footnote{这就是小振动近似, 但是前提我们假设了Taylor展开的二阶项存在.但是, 
    对于稳定平衡(极小值点)附近的Taylor展开, 完全可能出现二阶项为0的情形, 
    比如势能是四次函数, 这时使用小振动图像得到的结论就会完全出错.
    }
    , 那么就可以将势能函数重写为:
    \begin{equation}
        V(\bm{q}) = \frac{1}{2}\bm{q}^{\mr{t}}\left.\pdv{^2V}{\bm{x}^2}\right|_{\bm{x} = \bm{x}_{\mr{eq}}}\bm{q}
        \label{potential at equilibrium point}
    \end{equation}
    定义矩阵(即\textbf{Hessian矩阵},通常而言是一个半正定
    \footnote{
        一般而言这个矩阵一定有0作为特征值.因为分子在某些
        自由度运动时(刚性转动、质心平动)分子的势能不变, 这说明在势能极小值处
        沿着某些方向运动时势能函数恒定.可以说明代表分子整体平移的矢量是此矩阵特征值为0
        的特征向量, 但是对于分子的转动, 不一定会对应一个特征值为0的特征向量?
    }
    的实对称矩阵):
    \begin{equation}
        \bm{K} = \left.\pdv{^2V}{\bm{x}^2}\right|_{\bm{x} = \bm{x}_{\mr{eq}}}
        \label{potential matrix}
    \end{equation}
    那么系统在平衡位置的Hamilton量可以写为:
    \begin{equation}
        H(\bm{q}, \bm{p}) = \frac{1}{2}\bm{p}^{\mr{t}}\bm{M}^{-1}\bm{p} + \frac{1}{2}\bm{q}^{\mr{t}}\bm{K}\bm{q}
        \label{the Hamiltonian of samll vibration}
    \end{equation}
    其中$\bm{q}$为:
    \begin{equation}
        \bm{q} = \bm{x} - \bm{x_{\mr{eq}}}
    \end{equation}
    表示偏移平衡点的位移.

    \subsection{简正坐标}
    对于更一般的情况, 质量矩阵$\bm{M}$不一定是对角的, 但一定是实对称且正定的矩阵
    \footnote{对于非直角坐标, 比如在某些约束下定义的广义坐标, $\bm{M}$并不对角, 
    但是由于动能只可能是正值, 所以$\bm{M}$一定正定}
    , 这样就可以唯一地定义它正定的平方根$\bm{M}^{\frac{1}{2}}$
    \footnote{这也是一个实对称矩阵, 结论可以通过对角化这个矩阵来理解, 
    详细的证明要参考线性代数教材}, 
    将Hamilton量\ref{the Hamiltonian of samll vibration}写为
    (至于为什么要这么改写, 在Lagrange力学部分会详细解释):
    \begin{equation}
        H(\bm{q},\bm{p}) = \frac{1}{2}\left(\bm{M}^{-\frac{1}{2}}\bm{p}\right)^\mr{t}\bm{M}^{-\frac{1}{2}}\bm{p} + 
        \frac{1}{2}\left(\bm{M}^{\frac{1}{2}}\bm{q}\right)^{t}\bm{M}^{-\frac{1}{2}}\bm{K}\bm{M}^{-\frac{1}{2}}\left(\bm{M}^{\frac{1}{2}}\bm{q}\right)
    \end{equation}
    定义\textbf{质量加权的Hessian矩阵}$\bm{\mathcal{H}}$为 
    \begin{equation}
        \begin{split}
            \bm{\mathcal{H}} = \bm{M}^{-\frac{1}{2}}\bm{K}\bm{M}^{-\frac{1}{2}}
        \end{split}
        \label{Hessian matrix}
    \end{equation}
    它的量纲为s$^{-2}$. 这是一个实对称矩阵, 可以由正交矩阵作对角化:
    \begin{equation}
        \bm{\mathcal{H}} = \bm{S}^\mm{T}\bm{\Omega}\bm{S}
    \end{equation}
    其中$\bm{\Omega}$为一个对角矩阵, $\bm{S}$为\textbf{正交矩阵}
    \footnote{
        正交矩阵满足: 
    \begin{equation}
        \bm{S}^\mathrm{T}\bm{S} = \bm{SS}^\mathrm{T} = \bm{I}
    \end{equation}
    }
    , 其列向量为$\bm{\mathcal{H}}$的特征向量, 
    令:
    \begin{equation}
        \bm{\Omega} = \mathrm{diag} \{\omega_1^2, \dots, \omega_N^2 \}
    \end{equation}
    这样就得到了$N$个($N$为系统的总自由度数目)具有频率量纲的常数, 后面会讨论
    其物理含义.利用上面定义的矩阵, 可以继续改写系统的Hamilton量:
    \begin{equation}
        H(\bm{q},\bm{p}) = \frac{1}{2}\left(\bm{S}^\mr{t}\bm{M}^{-\frac{1}{2}}\bm{p}\right)^\mr{t}\left(\bm{S}^\mr{t}\bm{M}^{-\frac{1}{2}}\bm{p}\right) + 
        \frac{1}{2}\left(\bm{S}^\mr{t}\bm{M}^{\frac{1}{2}}\bm{q}\right)^{t}\bm{\Omega}\left(\bm{S}^\mr{t}\bm{M}^{\frac{1}{2}}\bm{q}\right)
    \end{equation}
    利用上面Hamilton量的形式, 定义如下相空间中的\textbf{坐标变换}(简正坐标变换):
    \begin{equation}
        \left\{
        \begin{split}
            \bm{Q} &= \bm{S}^\mr{t}\bm{M}^{\frac{1}{2}}\bm{q}\\
            \bm{P} &= \bm{S}^\mr{t}\bm{M}^{-\frac{1}{2}}\bm{p}
        \end{split}
        \right.
        \label{normal mode transformation}
    \end{equation}
    这样就可以将Hamilton量表示为:
    \begin{equation}
        \begin{split}
        H(\bm{Q},\bm{P}) &= \frac{1}{2}\bm{P}^{\mr{t}}\bm{P} + \frac{1}{2}\bm{Q}^{\mr{t}}\bm{\Omega}\bm{Q}\\
        & = \frac{1}{2}\sum_{i=1}^{N}P_{i}^2 + \frac{1}{2}\sum_{i=1}^{N}\omega_{i}^2Q_{i}^{2}
        \end{split}
    \end{equation}
    表面上看起来这是$N$个不耦合(之间没有相互作用)的简谐振子的Hamilton量, 将其
    带入Hamilton正则方程就可以得到$N$个独立谐振子的运动方程(在第一章
    中我们已经讨论过).\textbf{但是}, 
    这里忽略了一个重要的问题:我们无法保证坐标变换之后的新变量$(\bm{Q},\bm{P})$
    与对应的Hamilton量$H(\bm{Q},\bm{P})$满足Hamilton方程.

    \splitline

    在此处\textbf{不打算}完全解决这个问题, 而是给出此问题的一个严谨的表述, 
    目的是清楚地认识到简正坐标变换并不是随意进行的, 而是一种特殊的变换.
    考虑系统A的运动可以由Hamilton量$H(\bm{q},\bm{p},t),\,\,(\bm{q},\bm{p})\in\mathbb{R}^{2N}$
    与对应的Hamilton方程来描述, 定义相空间中的一个可逆变换:$\rho:(\bm{q},\bm{p})\mapsto(\bm{Q}, \bm{P}),\,\,\mathbb{R}^{2N}\to\mathbb{R}^{2N}$
    (其中$t$为时间, 也可以将其理解为任意参数)
    \begin{equation}
        \left\{
        \begin{split}
            Q_i &= Q_i(\bm{q},\bm{p}, t)\\
            P_i &= P_i(\bm{q}, \bm{p}, t)
        \end{split}
        \right.
    \end{equation}
    这样的变换可以很丰富, 比前面讨论的简正坐标变换\ref{normal mode transformation}, 
    就是相空间中不包含时间的一个线性变换;还有在第二章讨论过的Hamilton方程的解所定义的
    不同时刻轨线上的点之间的映射
    $\phi^{\tau}:(\bm{q}_{t},\bm{p}_{t})\mapsto(\bm{q}_{t+\tau},\bm{p}_{t+\tau})$.
    如果存在一个函数$K(\bm{Q}, \bm{P}, t)$能够让变换之后的坐标满足
    (即$K(\bm{Q}, \bm{P}, t)$\textbf{对应的}Hamilton方程等价于
    $H(\bm{q},\bm{p},t)$
    \textbf{对应的}Hamilton方程所描述的运动):
    \begin{equation}
        \left\{
        \begin{split}
            \dot{Q_i} &= \pdv{K}{P_i}\\
            \dot{P_{i}} &= -\pdv{K}{Q_i}
        \end{split}
        \right.
    \end{equation}
    那么称这样的变换为\textbf{正则变换}, 前面提到的两种变换都是正则变换.
    \footnote{
        一个变换是正则变换的充要条件是满足:
        \begin{equation}
            \left[\pdv{(\bm{Q},\bm{P})}{(\bm{q}, \bm{p})}\right]^{\mr{t}}\bm{J}\pdv{(\bm{Q},\bm{P})}{(\bm{q}, \bm{p})} = \bm{J}
        \end{equation}
        其中$\pdv{(\bm{Q},\bm{P})}{(\bm{q}, \bm{p})}$是正则变换的Jacobi矩阵, 它可以是时间的函数, $\bm{J}$为
        标准辛矩阵.可以通过这个验证文中所述的变换为正则变换.
        正则变换极大地拓展了Hamilton力学的内涵, 导出了很多在数学和物理上有深刻意义的结论, 
        比如之前详细讨论的Liouville定理, 就可以看作是正则变换不改变相空间体积的一个特例, 
        详细的介绍可以参考\cite{Goldstein2000Classical}.
    }
    对于相空间中的任意变换, “新Hamilton量”$K$的存在性是无法保证的, 为了说明这一点, 
    直接计算新坐标对于时间的导数(方便起见, 此处使用求和约定):
    \begin{equation}
        \left\{
            \begin{split}
                \dot{Q_i} &= \pdv{Q_i}{q_{j}}\dot{q_{j}} + \pdv{Q_i}{p_j}\dot{p_j}
                + \pdv{Q_i}{t} = \{Q_i, H\}_{\bm{q},\bm{p}} + \pdv{Q_i}{t}\\
                \dot{P_i} &= \pdv{P_i}{q_{j}}\dot{q_{j}} + \pdv{P_i}{p_j}\dot{p_j}
                + \pdv{P_i}{t} = \{P_i, H\}_{\bm{q},\bm{p}} + \pdv{P_i}{t}
            \end{split}
        \right.
        \label{the derivative of new coordinate}
    \end{equation}
    等式\ref{the derivative of new coordinate}右边为$(\bm{q},\bm{p},t)$的函数, 可以想象
    , 对于任意的变换, 并不一定可以找到$K(\bm{Q},\bm{P},t)$使得等式右边与
    $\pdv{K}{P_{i}},-\pdv{K}{Q_i}$对应相等.
    
    \splitline

    现在回到水分子的振动问题, 水分子的势能函数在平衡位置附近可以展开为二次型:
    \begin{equation}
        \begin{split}
        V(\bm{x}) &= V(\bm{x}_\mathrm{eq}) + \frac 12 (\bm{x-x}_\mathrm{eq})^\mathrm{t} \bm{V}^{(2)} (\bm{x-x}_\mathrm{eq})\\
        &= V(\bm{x}_\mathrm{eq}) + \frac 12 (\bm{x-x}_\mathrm{eq})^\mathrm{t} \mb{M}^{\frac 12}\bm{\mathcal{H}} \mb{M}^{\frac 12} (\bm{x-x}_\mathrm{eq})\\
        &= V(\bm{x}_\mathrm{eq}) + \frac 12 (\bm{x-x}_\mathrm{eq})^\mathrm{t} \mb{M}^{\frac 12} \mb{S}\bm{\Omega} \mb{S}^\mathrm{t} \mb{M}^{\frac 12} (\bm{x-x}_\mathrm{eq})
        \end{split}
    \end{equation}
    其中:
    \begin{equation}
        \left[\bm{V}^{(2)}\right]_{i,j} := \left.\pdv{^2 V}{x_i\partial x_j}\right|_{x_{eq}}
    \end{equation}
    定义\textbf{简正坐标}$\bm{Q}$:
    \begin{equation}
        \bm{Q} = \mathbf{S}^\mathrm{t} \mb{M}^{\frac 12} (\bm{x-x}_\mathrm{eq})
    \end{equation}
    于是平衡点附近的势能面可以近似表达为:
    \begin{equation}
        V(\bm{Q}) = V(\bm{0}) + \frac 12 \bm{Q}^\mathrm{t} \bm{\Omega Q} = V(\bm{0}) + \sum_{j=1}^N \frac 12 \omega_j^2 Q_j^2
    \end{equation}

    \subsection{简正坐标和Cartesian坐标的关系}
    上一节讨论了简正坐标变换.这一节讨论三个问题:1.如何用简正坐标表示系统的物理量, 2.
    验证简正坐标依然满足Hamilton方程, 3.给定直角坐标的初始条件, 如何通过简正坐标
    求解系统的运动方程.

    \splitline

    在Cartesian坐标系下系统的动量可以表示为:
    \begin{equation}
        \begin{split}
        \bm{p} &= \mb{M}\dot{\bm{q}} = \mb{M}^{\frac 12} \mb{S}\dot{\bm{Q}}\\
        \bm{p} &= \mb{M}^{\frac{1}{2}}\mb{S}\bm{P}
        \end{split}
    \end{equation}
    简正坐标中的动量定义为
    \footnote{
        注意这里的动量为“\textbf{正则动量}”, 并不是系统真实的动量.
        由于正则变换的存在, 原有的广义坐标与广义动量在变换后会被混合, 以至于完全
        失去了“坐标”与“动量”的原有含义, 甚至不具有坐标和动量的量纲, 
        因此在Hamilton力学中不再严格区分坐标与动量, 而是将$(\bm{Q}, \bm{P})$
        合称为正则共轭变量.\\
        可以选取变换:
        \begin{equation}
            \left\{
                \begin{split}
                    Q_{i} &= p_i\\
                    P_{i} &= -q_i
                \end{split}
            \right.
        \end{equation}
        将原本的动量与坐标互换.
    }
    :
    \begin{equation}
        \bm{P} = \mb{S}^\mr{t} \mb{M}^{-\frac 12} \bm{p}
    \end{equation}
    可以得到动能在简正坐标下的表达式:
    \begin{equation}
        \begin{split}
        E_{k} &= \frac{1}{2}\bm{p}^{\mr{t}}\mb{M}^{-1}\bm{p} = \frac{1}{2}\dot{\bm{Q}}^{\mr{t}}\dot{\bm{Q}}\\
        E_{k} &= \frac{1}{2}\bm{p}^{\mr{t}}\mb{M}^{-1}\bm{p} = \frac{1}{2}\bm{P}^{\mr{t}}\bm{P}
        \end{split}
    \end{equation}
    总结简正坐标和Cartesian坐标之间的变换:
    \begin{equation}
        \begin{aligned}
        \bm{Q} &= \mb{S}^\mathrm{t} \mb{M}^{\frac 12} (\bm{x-x}_\mathrm{eq})\\
        \bm{P} &= \mb{S}^\mathrm{t} \mb{M}^{-\frac 12} \bm{p}\\
        \bm{x} &= \bm{x}_\mathrm{eq} + \mb{M}^{-\frac 12}\mb{S}\bm{Q}\\
        \bm{p} &= \mb{M}^{\frac 12}\mb{S}\bm{P}
        \end{aligned}
        \label{normal mode and cartesian coordinate}
    \end{equation}

    \splitline

    有了简正坐标与Cartesian坐标之间的变换, 可以得到简正坐标下的Hamilton量(前一节已经给出):
    \begin{equation}
        H = \frac 12 \bm{P}^\mathrm{t}\bm{P} + \frac 12 \bm{Q}^\mathrm{t} \bm{\Omega Q}
    \end{equation}
    使用Hamilton正则方程, 可以得到简正坐标下的运动方程:
    \begin{equation}
        \begin{aligned}
            \bm{\dot{Q}} &= \frac {\partial H}{\partial \bm{P}} = \bm{P}\\
            \bm{\dot{P}} &= -\frac {\partial H}{\partial \bm{Q}} = -\bm{\Omega Q}
        \end{aligned}
        \label{equation of motion in normal mode coordinate}
    \end{equation}
    上式也可以写为分量的形式:
    \begin{equation}
        \begin{aligned}
            \dot{Q}_j &= P_j\\
            \dot{P}_j &= - \omega_j^2 Q_j
        \end{aligned}
    \end{equation}
    前一节中已经指出, 相空间中的任意变换并不能够保证Hamilton方程的不变, 这里验证一下
    简正坐标变换下Hamilton方程的不变性.首先根据Cartesian坐标下的Hamilton量给出运动方程:
    \begin{equation}
        \begin{split}
            \dot{\bm{q}} &= \pdv{H}{\bm{p}} = \mb{M^{-1}}\bm{P}\\
            \dot{\bm{p}} &= -\pdv{H}{\bm{q}} = -V^{(2)}\bm{q}
        \end{split}
        \label{equation of motion in cartesian coordiante}
    \end{equation}
    上式同样可以写为分量形式:
    \begin{equation}
        \begin{aligned}
            \dot{x}_i &= \frac {p_j}{m_i}\\
            \dot{p}_i &= \sum_j \frac {\partial^2 V}{\partial x_i \partial x_j} (x_j - x_\mathrm{eq}^{(j)})
        \end{aligned}
    \end{equation}
    这要比在简正坐标下的形式要复杂很多.
    将简正坐标变换\ref{normal mode and cartesian coordinate}
    带入Cartesian坐标下的运动方程\ref{equation of motion in cartesian coordiante}, 可以得到
    简正坐标下的运动方程\ref{equation of motion in normal mode coordinate}, 这样就验证了
    简正坐标变换保证了Hamilton方程的不变.

    \splitline

    如果给定0时刻时系统在Cartesian坐标下的初始条件$(\bm{x}(0),\bm{p}(0))$, 
    可以根据简正坐标变换给出在简正坐标下的初始条件
    (同时考虑了质量矩阵为对角矩阵这种常见的情形):
    \begin{equation}
        \begin{aligned}
            Q_j(0) &=\sum_{i,k} S_{ij} M^{\frac{1}{2}}_{ik}(x_k(0) - x_{\mr{eq},k}) =\sum_i S_{ij} M_{ii}^{\frac 12} (x_i(0) - x_{\mathrm{eq},i})\\
            P_j(0) &=\sum_{i,k} S_{ij} M^{-\frac{1}{2}}_{ik}p_k(0)=\sum_i S_{ij} M_{ii}^{-\frac 12} p_{i}(0)
        \end{aligned}
    \end{equation}
    根据正则方程可以解出简正坐标下的运动方程:
    \begin{equation}
        \begin{aligned}
            Q_j(t) &= Q_j(0)\cos{\omega_j t} + \frac{P_j(0)}{\omega_j} \sin{\omega_j t}\\
            P_j(t) &= P_j(0)\cos{\omega_j t} - \omega_j Q_j(0) \sin{\omega_j t}
        \end{aligned}
    \end{equation}
    再根据简正坐标变换得到Cartesian坐标下的运动方程, 这样就利用简正坐标
    完全求解了小振动的问题, 这里不再给出具体形式.

    \begin{asg}
        一维四次势中粒子的运动.尝试在给定初始条件的情况下给出解析解;同时尝试用不同的数值
        方法求解运动方程,比较几种方法的差异(能量是否稳定?)
    \end{asg}
    \begin{asg}
        竖立粉笔的问题.竖立在桌面上的粉笔是否会永远静止?如果不是,请求出粉笔偏离平衡位置的
        角度的平均值与平方平均值.
    \end{asg}
    \begin{asg}
        证明Hessian矩阵的本征值都是实数.
    \end{asg}
    \begin{asg}
        给定水分子简正坐标下的初始系综密度函数, 求键长、键角
        的期望和涨落随时间的变化.
    \end{asg}
    
    \bibliographystyle{plain}
    \bibliography{ref_hamilton}

    \chapter{量子力学的算符形式}

    \section{量子力学的基本假设}
        从本节开始讨论量子力学
        \footnote{若想系统地了解本章所介绍内容, 请参考樱井纯所著《现代量子力学》第二章\cite{MQM}}.
        量子力学中第一个重要的概念是\textbf{态}, 一般使用Hilbert空间
        \footnote{数学上指完备的内积空间}
        来描述一个量子系统, 而态对应着Hilbert空间中一个归一化的向量.
        我们可以对\textbf{态}进行\textbf{测量}, 得到这个态的物理量.
        用$|\psi\rangle$来表示态.

        如何来描述这个态呢?我们可以选择一个\textbf{表象}. 
        在经典力学中, 我们用相空间中的一个点来唯一地描述经典系统的状态. 
        但在量子力学框架下, 我们通常选择位置表象或者动量表象进行描述. 
        如果选取位置空间, 对这个态的描述为: 
        \begin{equation}
            \langle \bm{x} | \psi \rangle = \psi(\bm{x})
        \end{equation}
        将这个函数称为\textbf{波函数}. 如果选取动量表象, 类似地可以描述为:
        \begin{equation}
            \langle \bm{p} | \psi \rangle = \psi(\bm{p})
        \end{equation}
        量子力学中, 位置表象和动量表象都是连续的.
        \footnote{对于量子力学的数学基础, 由于笔者的数学水平有限, 有很多很多问题没有想明白, 
        其中有一个有关坐标、动量本征态. 它们是不是Hilbert空间的基, 
        如果是, 为什么它们的归一化显得比较奇怪(使用$\delta$函数). 
        此外, 如果一个空间可以同时用不可数的"基"(位置本征态)与可数的基来描述, 那么这个空间的维数怎么确定;
        如果不是, 那为什么可以完备的表示所有态?}
        我们也可以在离散的表象中描述态.
        态是一个向量, 它可以用一组基(表象)展开. 回顾在线性代数中:  
        \begin{equation}
            \bm{c} = \sum_n c_n \bm{e}_n
        \end{equation}
        如果这组基是内积空间中的规范正交基, 则: 
        \footnote{最后一个等号采用了量子力学中常用的Dirac记号, 用bra-ket表示内积}
        \begin{equation}
            \bm{c} = \sum_n \bm{e}_n (\bm{e}_n^\mathrm{T}\bm{c}) 
            = \sum_n |n\rangle \langle n|c\rangle
        \end{equation}
        由此可见: 
        \begin{equation}
            \bm{I} = \sum_n |n\rangle \langle n|
        \end{equation}
        在量子力学中, 我们可以类似地描述态: 
        \begin{equation}
            |\psi \rangle = \sum_n |n\rangle \langle n|\psi \rangle = \sum_n c_n |n\rangle
        \end{equation}

        物理量测量都是实数(实验事实). 物理量在量子力学中都对应一个自伴算符(量子力学的基本假设), 
        假设: 
        \begin{equation}
            \hat{A} |n\rangle = a_n |n\rangle
        \end{equation}
        其中$a_n$为实数, 那么$|n\rangle$就是$\hat{A}$的一个本征态.我们可以把态在$\hat{A}$的本征态上
        来展开
        \footnote{根据假设$\hat{A}$表示一个自伴算符, 这里将有限维内积空间中的谱定理
        "推广"到Hilbert空间中, 那么$\hat{A}$的所有本征态构成空间的一个完备基}, 得到: 
        \begin{equation}\begin{aligned}
            \hat{A}|\psi \rangle &= \hat{A}\hat{I}|\psi \rangle
            = \hat{A} \sum_n |n\rangle \langle n|\psi\rangle
            = \sum_n c_n a_n |n\rangle
        \end{aligned}\end{equation}
        可以计算出$\hat{A}$的平均值为
        \begin{equation}\begin{aligned}
            \langle \hat{A} \rangle &= \langle \psi |\hat{A}| \psi \rangle\\
            &= \sum_n \langle n| c_n^* \sum_m c_m a_m |m\rangle\\
            &= \sum_n \sum_m c_n^* c_m a_m \langle n|m\rangle\\
            &= \sum_n |c_n|^2 a_n
        \end{aligned}\end{equation}
        这里用到了: 
        \[\langle n | m \rangle = \delta_{nm}\]
        注意到$|c_n|^2 \in [0,1]$, 且$\sum_n |c_n|^2 = 1$
        (我们总可以让这个态乘一个常数使该式成立, 此时态也满足归一化条件
        $\langle \psi |\psi \rangle = 1$), 所以可以认为这个态处于该本征态的概率. 
        按照\textbf{Copenhagen学派}的观点, 我们测量某一个物理量时这个态会坍塌到
        这个物理量的一个本征态, 而$|c_n^2|$反应了坍塌到第$n$个本征态的概率.
        对于波函数$\langle \bm{x}|\psi \rangle$, 它的模方是在位置空间的概率密度. 
        定义一个位置算符$\hat{\bm{x}}$. 应有: 
        \begin{equation}
            \hat{\bm{x}}|\bm{x}_0\rangle = \bm{x}_0 |\bm{x}_0\rangle
        \end{equation}
        其中态$|\bm{x}_0\rangle$代表精确地处在$\bm{x}_0$位置的态. 
        引入动量算符$\hat{\bm{p}}$, 同样有: 
        \begin{equation}
            \hat{\bm{p}}|\bm{p}_0\rangle = \bm{p}_0 |\bm{p}_0\rangle
        \end{equation}
        类比在可数个物理量本征态下的展开, 同样有: 
        \begin{equation}
            \begin{split}
            \hat{I} &= \int |\bm{x}\rangle \langle \bm{x}| \mathrm{d}\bm{x}\\
            \hat{I} &= \int |\bm{p}\rangle \langle \bm{p}| \mathrm{d}\bm{p}
            \end{split}
        \end{equation}
        于是对位置的测量应有: 
        \begin{equation}
            \hat{\bm{x}}|\psi \rangle = \int \hat{\bm{x}}|\bm{x}\rangle \langle \bm{x}|\psi \rangle \mathrm{d}\bm{x}
            = \int \bm{x}|\bm{x}\rangle \langle \bm{x}|\psi \rangle \mathrm{d}\bm{x}
        \end{equation}
        同样, 任意一个态可以展开到位置空间: 
        \begin{equation}
            |\psi \rangle = \int |\bm{x} \rangle \langle \bm{x}|\psi\rangle \mathrm{d}\bm{x}
        \end{equation}
        类似地得到位置的平均值为
        \begin{equation}
            \langle \psi | \hat{\bm{x}} | \psi \rangle 
            = \int \bm{x} |\langle \bm{x} | \psi \rangle|^2 \mathrm{d}\bm{x}
        \end{equation}
        这就给出了波函数的概率诠释.

        接下来讨论动量算符在位置空间的描述
        \footnote{这个是作为量子力学的基本假设引入的, 不同的教科书可能有着不同的引入方式}
        . 假设: 
        \begin{equation}
            \langle \bm{x} |\hat{\bm{p}} | \psi \rangle
            = \langle \bm{x} | \phi \rangle
        \end{equation}
        如果选择: 
        \begin{equation}
            \hat{\bm{p}} = -\mathrm{i}\hslash \frac {\partial }{\partial \bm{x}}
        \end{equation}
        就得到: 
        \begin{equation}
            \langle \bm{x} |\hat{\bm{p}} | \psi \rangle = -\mathrm{i}\hslash \frac {\partial }{\partial \bm{x}}\langle \bm{x} | \psi \rangle
        \end{equation}
        有了位置算符和动量算符, 我们可以讨论两个算符的\textbf{对易}: 
        \begin{equation}
            [\hat{\bm{x}},\hat{\bm{p}}] = \hat{\bm{x}}\hat{\bm{p}} - \hat{\bm{p}}\hat{\bm{x}}
        \end{equation}
        先计算: 
        \begin{equation}\begin{aligned}
            \langle \bm{x}_0 |\hat{\bm{p}}\hat{\bm{x}} | \psi \rangle &= -\mathrm{i}\hslash \frac {\partial}{\partial \bm{x}_0} \langle \bm{x}_0 |\hat{\bm{x}} | \psi \rangle\\
            &= -\mathrm{i}\hslash \frac {\partial}{\partial \bm{x}_0} (\bm{x}_0\psi(\bm{x}_0))\\
            &= -\mathrm{i}\hslash \psi(\bm{x}_0) - \mathrm{i}\hslash \bm{x}_0 \frac {\partial \psi(\bm{x}_0)}{\partial \bm{x}_0}
        \end{aligned}\end{equation}
        再计算: 
        \begin{equation}\begin{aligned}
            \langle \bm{x}_0 |\hat{\bm{x}} \hat{\bm{p}} | \psi \rangle &= \bm{x}_0 \langle \bm{x}_0 \hat{\bm{p}} | \psi \rangle\\
            &= -\mathrm{i}\hslash \bm{x}_0 \frac {\partial \psi(\bm{x}_0)}{\partial \bm{x}_0}
        \end{aligned}\end{equation}
        这两个式子相比较, 得到: 
        \begin{equation}
            [\hat{\bm{x}},\hat{\bm{p}}] = \mathrm{i}\hslash
        \end{equation}
        此即\textbf{基本对易关系}.

    \section{不确定性原理}
        总结一下量子力学的基本假设: 
        \begin{enumerate}
            \item 波函数: 态$|\psi \rangle$可以在位置空间描述$\langle x|\psi\rangle$
            \item 算符: 物理量对应Hermite算符.
            \item 测量: 对某个态测量某个物理量, 会得到其本征值.
            \item 基本对易关系: $[\hat{\bm{x}},\hat{\bm{p}}] = \mathrm{i}\hslash$
        \end{enumerate}
        在量子力学中, 位置空间、动量空间是连续的, 时间也是连续的, 并且认为质量不变. 
        量子力学中, 位置和动量都有对应的算符, 但时间没有.

        可以定义位置的量子涨落: 
        \begin{equation}
            \Delta x = \sqrt{\langle \hat{x}^2 \rangle - \langle \hat{x} \rangle^2}
        \end{equation}
        同理可以定义动量的量子涨落: 
        \begin{equation}
            \Delta p = \sqrt{\langle \hat{p}^2 \rangle - \langle \hat{p} \rangle^2}
        \end{equation}
        现在定义 : 
        \begin{equation}\begin{aligned}
            \Delta \hat{x} &= \hat{x} - \langle \hat{x} \rangle \\
            \Delta \hat{p} &= \hat{p} - \langle \hat{p} \rangle
        \end{aligned}\end{equation}
        希望求出: 
        \begin{equation}\begin{aligned}
            \langle \Delta \hat{x}^2\rangle\langle \Delta \hat{p}^2\rangle
        \end{aligned}\end{equation}
        设: 
        \begin{equation}\begin{aligned}
            |\phi_x \rangle &= \Delta \hat{x}^2|\psi\rangle\\
            |\phi_p \rangle &= \Delta \hat{p}^2|\psi\rangle
        \end{aligned}\end{equation}
        于是: 
        \begin{equation}\begin{aligned}
            \langle \Delta \hat{x}^2\rangle\langle \Delta \hat{p}^2\rangle = \langle \phi_x | \phi_x \rangle \langle \phi_p | \phi_p \rangle
        \end{aligned}\end{equation}
        考察: 
        \footnote{这里使用了Cauchy不等式}
        \begin{equation}\begin{aligned}
            |\langle \phi_x | \phi_p \rangle| = |\bm{a}^\dagger \bm{b}| = |\bm{a}||\bm{b}|\cos{\theta} \leqslant |\bm{a}||\bm{b}|
        \end{aligned}\end{equation}
        应有: 
        \begin{equation}\begin{aligned}
            \langle \Delta \hat{x}^2\rangle\langle \Delta \hat{p}^2\rangle &= \langle \phi_x | \phi_x \rangle \langle \phi_p | \phi_p \rangle\\ &\geqslant |\langle \phi_x|\phi_p \rangle|^2\\
            &= |\langle \psi |\Delta \hat{x} \Delta \hat{p}|\psi \rangle|^2
        \end{aligned}\end{equation}
        所以只需要求出: 
        \begin{equation}\begin{aligned}
            \Delta \hat{x} \Delta \hat{p} = \frac 12([\Delta \hat{x}, \Delta \hat{p}] + \{\Delta \hat{x}, \Delta \hat{p}\})
        \end{aligned}\end{equation}
        其中反对易关系: 
        \begin{equation}
            \{\hat{A}, \hat{B}\} = \hat{A}\hat{B} + \hat{B}\hat{A}
        \end{equation}
        因此: 
        \begin{equation}
            |\langle \psi |\Delta \hat{x} \Delta \hat{p}|\psi \rangle|^2 = \frac 14 |\langle \psi |[\Delta \hat{x},\Delta \hat{p}]|\psi \rangle + \langle \psi |\{\Delta \hat{x},\Delta \hat{p}\}|\psi \rangle|^2
        \end{equation}
        注意: 
        \begin{equation}
            [\Delta \hat{x},\Delta \hat{p}] = [\hat{x}-\langle \hat{x} \rangle, \hat{p}-\langle \hat{p} \rangle ] = [\hat{x},\hat{p}] = \mathrm{i}\hslash
        \end{equation}
        上面就是一个复数模的平方, 得到: 
        \begin{equation}
            |\langle \psi |\Delta \hat{x} \Delta \hat{p}|\psi \rangle|^2 = \frac {\hslash^2}4 + \frac 14 |\langle \psi |\{\Delta \hat{x},\Delta \hat{p}\}|\psi \rangle|^2 \geqslant \frac {\hslash^2}4
        \end{equation}
        这就是不确定性原理, 完全是由基本对易关系决定的, 可以认为这两者等价.

        现在来在位置空间描述动量本征态, 即求出$\langle x|p \rangle$. 
        这表示动量精确地处在$p$时, 在位置空间的描述. 显然地, 它满足: 
        \begin{equation}
            \langle x|\hat{p}|p\rangle = p\langle x|p \rangle
        \end{equation}
        由此可知: 
        \begin{equation}\begin{aligned}
            -\mathrm{i}\hslash \frac {\partial}{\partial x} \langle x|p \rangle = p \langle x|p \rangle
        \end{aligned}\end{equation}
        解这个常微分方程, 得到: 
        \begin{equation}
            \langle x|p \rangle = C \mathrm{e}^{\frac {\mathrm{i}px}\hslash}
        \end{equation}
        $C$由归一化条件决定. 首先考虑:
        \begin{equation}
            \langle p' | p_0 \rangle = \delta (p'-p_0)
        \end{equation}
        这是因为: 
        \begin{equation}
            |p_0 \rangle = \int |p\rangle \langle p|p_0\rangle \mathrm{d}p
        \end{equation}
        显然$\delta$函数满足这个要求.又
        \begin{equation}
            \langle p' | p_0 \rangle = \delta (p'-p_0) = \frac 1{2\pi\hslash} \int \mathrm{e}^{\frac {\mathrm{i}(p-p_0)x}{\hslash}} \mathrm{d}x
        \end{equation}
        并且: 
        \begin{equation}\begin{aligned}
            \langle p'|p_0 \rangle &= \int \langle p'|x\rangle \langle x|p_0 \rangle \mathrm{d}x\\
            &= \int (\langle x|p' \rangle)^* \langle x|p_0 \rangle \mathrm{d}x\\
            &= C^*C\int \mathrm{e}^{\frac {\mathrm{i}(p_0-p')x}{\hslash}} \mathrm{d}x
        \end{aligned}\end{equation}
        所以: 
        \begin{equation}
            C = \frac 1{\sqrt{2\pi\hslash}}
        \end{equation}
        这样就得到: 
        \begin{equation}
            \langle x|p \rangle = \frac 1{\sqrt{2\pi\hslash}} \mathrm{e}^{\frac {\mathrm{i}px}{\hslash}}
        \end{equation}
        并由此可以得到: 
        \footnote{如何推广到高维情形?}
        \begin{equation}
            \langle p|x \rangle = \frac 1{\sqrt{2\pi\hslash}} \mathrm{e}^{-\frac {\mathrm{i}px}{\hslash}}
        \end{equation}
        \begin{asg}
            第5次作业第1题: 计算动量空间的位置算符.
        \end{asg}
        \begin{asg}
            第5次作业第2题: $\delta$函数算符问题.
        \end{asg}

    \section{量子力学模型体系:一维无限深势阱}

    \subsection{一维无限深势阱}
        考虑动能算符和动量算符的对易关系, 
        \begin{equation}
            [\frac {\hat{p}^2}{2m}, \hat{p}] = 0
        \end{equation}
        事实上可以证明: 
        \begin{equation}
            [f(\hat{A}),g(\hat{A})] = 0
        \end{equation}
        \begin{asg}
            第6次作业第1题(1): 证明上述结论.
        \end{asg}
        如果: 
        \[ [\hat{A},\hat{B}]=0 \]
        并设$|\phi_n\rangle$是$\hat{A}$的一个本征态
        \[ \hat{A} |\phi_n \rangle = a_n |\phi_n \rangle \]
        那么: 
        \[ \hat{A} \hat{B} |\phi_n \rangle = \hat{B}\hat{A} |\phi_n \rangle = a_n \hat{B} |\phi_n \rangle \]
        这说明, 如果$\hat{A},\hat{B}$对易, 则$\hat{B}|\phi_n\rangle$必然是$\hat{A}$的本征态, 
        且本征值为$a_n$.如果不简并, 那么$\hat{B}|\phi_n\rangle$一定是$\phi_n$的一个倍数, 
        即: 
        \begin{equation}
            \hat{B}|\phi_n \rangle = b_n |\phi_n \rangle
        \end{equation}
        对于简并的情况, $\hat{B}\ket{|\phi_n}$只能是所有本征值为$a_n$的本征态的线性组合. 
        也就是说, 如果: 
        \[ \hat{A}\ket{\phi_{n+m}} = a_n \ket{\phi_{n+m}},\ m=0,...,k\]
        并且$\hat{A},\hat{B}$对易, 那么
        \[ \hat{B}|\phi_n \rangle = \sum_{m=0}^k c_m |\phi_{n+m} \rangle \]
        我们可以再将一个$\hat{B}$算符作用上来, 得到
        \[ \hat{B}\hat{A} \sum_{m=0}^k c_m|\phi_{n+m}\rangle = \hat{A} \hat{B} \sum_{m=0}^k c_m|\phi_{n+m} \rangle = \hat{A} \sum_{m=0}^k c_m'|\phi_{n+m} \rangle \]
        在简并的情况下, 可以通过构造得到$\hat{B}$的本征态.这是因为$\hat{B}$是一个Hermite算符, 可以对角化: 
        \[ \bm{U}^\dagger \bm{BU} = \bm{\Lambda} \]
        可以得到其本征态.
        
        \splitline

        我们已经讨论过动能算符和动量算符是对易的, 如果: 
        \[ E_0 = \frac {p_0^2}{2m} \]
        那么: 
        \[ p_0 = \pm \sqrt{2mE_0} \]
        可以对应动能算符的两个本征态.也就是说: 
        \[ \frac {p_0^2}{2m} |\psi\rangle = \frac {\hat{p}^2}{2m} (c_+|p_0\rangle + c_-|p_0\rangle) \]

        现在求解一维无限深势阱的能量本征态. 其势能算符为:   
        \begin{equation}
            V(x) = \left \{
                \begin{aligned}
                    &0,\ x\in [-\frac L2, \frac L2 ]\\
                    &\infty, \ \mathrm{otherwise}
                \end{aligned}
                \right.
        \end{equation}
        Hamilton算符为: 
        \begin{equation}
            \hat{H} = \frac {\hat{p}^2}{2m}
        \end{equation}
        波函数只能在$[-\frac L2, \frac L2 ]$区间内, 并且边界条件给出: 
        \begin{equation}\begin{aligned}
            \phi(x = -\frac L2) &= 0\\
            \phi(x = \frac L2) &= 0
        \end{aligned}\end{equation}
        应有:
        \begin{equation}\begin{aligned}
            \phi_n(x) &= c_+ \langle x|p_n\rangle + c_- \langle x|p_n\rangle\\
            &= \frac 1{\sqrt{2\pi \hslash}}(c_+\mathrm{e}^{\frac {\mathrm{i}xp_n}{\hslash}}+c_-\mathrm{e}^{-\frac {\mathrm{i}xp_n}{\hslash}})
        \end{aligned}\end{equation}
        再加上边界条件:
        \begin{equation}\begin{aligned}
            c_+\mathrm{e}^{\frac {\mathrm{i}Lp_n}{2\hslash}}+c_-\mathrm{e}^{-\frac {\mathrm{i}Lp_n}{2\hslash}} &= 0\\
            c_+\mathrm{e}^{-\frac {\mathrm{i}Lp_n}{2\hslash}}+c_-\mathrm{e}^{\frac {\mathrm{i}Lp_n}{2\hslash}} &= 0
        \end{aligned}\end{equation}
        又有归一化条件:
        \begin{equation}\begin{aligned}
            \langle \phi_n | \phi_n \rangle = \int_{-\frac L2}^{\frac L2} |\phi_n(x)|^2 \mathrm{d}x = 1
        \end{aligned}\end{equation}
        定义算符$\hat{B}$满足:
        \[ \langle x|\hat{B}| p \rangle = \langle x+\lambda |p\rangle \]
        由于:
        \begin{equation}\begin{aligned}
            \langle x|\hat{B}| p \rangle = \langle x+\lambda |p\rangle = \frac {\mathrm{e}^{\frac {\mathrm{i}(x+\lambda)p}{\hslash}}}{\sqrt{2\pi\hslash}}
        \end{aligned}\end{equation}
        显然地, 应有:
        \begin{equation}
            \hat{B} = \mathrm{e}^{\frac {\mathrm{i}\lambda \hat{p}}{\hslash}}
        \end{equation}
        这个算符称为\textbf{平移算符}.左矢形式表达为:
        \[ \langle x| \mathrm{e}^{\frac {\mathrm{i}\lambda \hat{p}}{\hslash}} = \langle x+\lambda| \]
        右矢形式表达为:
        \[ \mathrm{e}^{-\frac {\mathrm{i}\lambda \hat{p}}{\hslash}} | x\rangle = |x+\lambda \rangle \]
        我们使用平移算符, 将一维势阱的体系作平移, 
        将波函数平移到$[0,\frac L2]$的位置上.此时体系满足:
        \begin{equation}
            V(x) = \left \{
                \begin{aligned}
                    &0,\ x\in [0, L]\\
                    &\infty, \ \mathrm{otherwise}
                \end{aligned}
                \right.
        \end{equation}
        由边界条件:
        \begin{equation}\begin{aligned}
            \phi(0)= 0\\
            \phi(L) = 0
        \end{aligned}\end{equation}
        得到:
        \begin{equation}\begin{aligned}
            c_+ +c_- &= 0\\
            c_+\mathrm{e}^{\frac {\mathrm{i}Lp_n}{\hslash}}+c_-\mathrm{e}^{-\frac {\mathrm{i}Lp_n}{\hslash}} &= 0
        \end{aligned}\end{equation}
        将前一个式子代入后一个, 得到:
        \[ c_+\mathrm{e}^{\frac {\mathrm{i}Lp_n}{\hslash}}-c_+ \mathrm{e}^{-\frac {\mathrm{i}Lp_n}{\hslash}} = 0 \]
        于是:
        \begin{equation}\begin{aligned}
            2\mathrm{i}c_+ \sin{\frac {Lp_n}{\hslash}} = 0
        \end{aligned}\end{equation}
        但是$c_+ \neq 0$(否则得到零解), 所以:
        \begin{equation}\begin{aligned}
            \frac {Lp_n}{\hslash} = n\pi
        \end{aligned}\end{equation}
        即:
        \[ p_n = \frac {n\pi \hslash}{L} \]
        $c_+$的选择取决于归一化条件: 
        \[ \int_0^L |c|^2 \sin^2{\frac {n\pi x}L}\mathrm{d}x = 1 \]
        算出:
        \[ c = \sqrt{\frac 2L} \]
        于是, 一维无限深势阱的解为:
        \[ \langle x|\phi_n \rangle = \sqrt{\frac 2L} \sin{\frac {n\pi x}L} \]
        本征值为:
        \[ \epsilon_n = \frac {\hslash^2}{2m} \bigg(\frac {n\pi}L\bigg)^2 \]
        平移回来, 得到:
        \[ \langle x|\phi_n \rangle = \sqrt{\frac 2L} \sin{\bigg(\frac {n\pi}L\bigg(x+\frac L2\bigg)\bigg)}, \ n=1,2,3,... \]
        本征值和平移前一样.
        \begin{asg}
            第5次作业第3题(1): 一维无限深势阱能量本征态在动量空间的表示.
        \end{asg}
        \begin{asg}
            第6次作业第1题(2): 动量平移算符.
        \end{asg}

    \subsection{一维势阱求解自由粒子问题}
        上一节在求解一维无限深势阱的过程中, 引入了平移算符.我们想要了解
        $\mathrm{e}^{\frac {\mathrm{i}\lambda \hat{p}}{\hslash}} \hat{H} \mathrm{e}^{-\frac {\mathrm{i}\lambda \hat{p}}{\hslash}}$
        的性质. 显然地, 这是一个Hermite算符, 并且新的算符和原来的Hamilton算符$\hat{H}$有相同的本征值. 
        这是因为酉变换并不影响算符的本征值.
        \begin{asg}
            第6次作业第1题(2): 证明这个结论.
        \end{asg}
        现在想要来模拟自由粒子, 只需要让$L \to \infty$. 对于自由粒子, 如果给定温度$T$, 
        动量应当满足Boltzmann分布:
        \[ \rho(p) = \sqrt{\frac {\beta}{2\pi m}}\mathrm{e}^{-\frac {\beta p^2}{2m}} \]
        计算能量的平均值:
        \[ \langle \frac {p^2}{2m} \rangle = \frac 1{2\beta} \]

        如果:
        \[ \hat{H} | \phi_n \rangle = \epsilon_n |\phi_n \]
        那么显然有:
        \[ f(\hat{H})|\phi_n \rangle = f(\epsilon_n)|\phi_n \]
        定义Boltzmann算符$\mathrm{e}^{-\beta\hat{H}}$, 并且定义配分函数为
        \footnote{在无限维空间中, 算符的迹并不总是良定义的}
        :
        \[ Z = \mathrm{Tr}\  \mathrm{e}^{-\beta \hat{H}} = \sum_n \langle n| \mathrm{e}^{-\beta \hat{H}} | n \rangle = \sum_n \mathrm{e}^{-\beta \epsilon_n }\]
        那么这个配分函数是否收敛呢?
        \[ Z = \sum_n \mathrm{e}^{-\beta \frac {\hslash^2}{2m} (\frac {n\pi}L)^2} \]
        定义$x_n = \frac nL$, 于是$\Delta x = \frac 1L$. 由此可以将求和近似为积分: 
        \begin{equation}\begin{aligned}
            Z &= L \sum_n \Delta x \mathrm{e}^{-\beta \frac {\hslash^2\pi^2 x^2}{2m} }\\
            &= L \int_0^{+\infty} \mathrm{e}^{-\beta \frac {\hslash^2\pi^2 x^2}{2m}} \mathrm{d}x\\
            &= L\sqrt{\frac m{2\pi \beta \hslash^2}}
        \end{aligned}\end{equation}
        或者我们定义: 
        \[ \mathrm{Tr} \ \mathrm{e}^{-\frac {\beta \hat{p}^2}{2m}} = \int \mathrm{e}^{-\frac {\beta p^2}{2m}} \langle p|p\rangle \mathrm{d}p \]
        发现这里并不好处理, 只能知道该值为$\infty$, 但不能给出具体的表达形式.
        这就是我们使用一维无限深势阱来近似自由粒子的原因.

        有了一维形式的配分函数, 类比得到三维粒子为:
        \[ Z = V \bigg(\frac m{2\pi \beta \hslash^2}\bigg)^{\frac 32} \]

        有了配分函数可以得到一维情况下能量的平均值:
        \begin{equation}\begin{aligned}
            \langle \hat{H} \rangle &= \frac {\mathrm{Tr} \ (\mathrm{e}^{-\beta \hat{H}} \hat{H})}Z
            = \frac {\sum_n \epsilon_n \mathrm{e}^{-\beta \epsilon_n}}{\sum_n \mathrm{e}^{-\beta \epsilon_n}}
            = -\frac {\partial}{\partial \beta} \ln{Z}
        \end{aligned}\end{equation}
        将配分函数代入得到: 
        \[ \langle \hat{H} \rangle = \frac 1{2\beta} \]
        该结果和经典情况得到的结果是一致的. 得到结论, 
        自由粒子的体系经典和量子力学的结果是一致的.
        \begin{asg}
            第5次作业第3题(3): 能量涨落的计算
        \end{asg}
        \begin{asg}
            第5次作业第3题(4): 比热的计算
        \end{asg}
        \begin{asg}
            第5次作业第3题(5): 用一维势阱求解共轭体系
        \end{asg}

    \subsection{用一维势阱模型的本征函数求解其他势能体系}
        我们解出了一维势阱能量本征态在位置空间的波函数, 可以求求在动量空间的波函数:
        \[ \langle p|\phi_n \rangle = \int \langle p|x \rangle \langle x |\phi_n \rangle \mathrm{d}x \]
        这相当于函数的Fourier变换. 

        两个有限维矩阵乘积的求迹: 
        \begin{equation}
            \mathrm{Tr} \ (\bm{AB}) = \mathrm{Tr} \ (\bm{BA})
        \end{equation}
        证明是显然的: 
        \begin{equation}\begin{aligned}
            \mathrm{Tr} \ (\bm{AB}) &= \sum_i (\bm{AB})_{ii}
            = \sum_i \sum_k a_{ik}b_{ki}
            = \sum_k \sum_i b_{ki}a_{ik}
            = \sum_k \bm{(BA)}_{kk}
            = \mathrm{Tr} \ (\bm{BA})
        \end{aligned}\end{equation}
        但对于无限维的, 必须保证二重级数绝对收敛才能够交换次序才能成立?
        \begin{asg}
            第5次作业第3题(2): 位置算符和动量算符乘积交换后迹是否相等?
        \end{asg}

        \splitline

        一维无限深势阱的能级差会随着$n$的增大而增大.
        \[ \Delta \epsilon = \frac {\hslash^2}{2m} \bigg(\frac {n\pi}L \bigg)^2 (2n+1) \]
        虽然如此, 我们仍可以用一维无限深势阱的能量本征态来对其他的体系进行研究
        (这相当于认为选定了Hilbert空间的基, 这组基不一定必须是系统Hamilton算符的本征态). 
        比如一个势能算符为$\hat{V}'$时, 势能矩阵元为: 
        \[
            \langle \phi_k | \hat{V}' | \phi_n \rangle = \int_{-\frac L2}^{+\frac L2} \phi_k^*(x) V(x) \phi_n(x) \mathrm{d}x 
        \]
        但是动能算符对应的矩阵是对角的: 
        \[
            \langle \phi_k | \frac {\hat{p^2}}{2m} | \phi_n \rangle = \delta_{kn} \frac {\hslash^2}{2m} \bigg( \frac {\pi}L \bigg)^2 n^2
        \]
        由此可以得到Hamilton算符的矩阵元$\langle\phi_k |\hat{H} |\phi_n \rangle$, 
        显然, 这是一个无穷维矩阵, 并不能用于实际计算, 通常我们会根据所感兴趣性质计算的
        收敛情况对基进行截断, 截断后矩阵就会变为有限维, 将它对角化就可以
        得到本征值与本征向量, 对应着所求解的系统的能量本征值与能量本征态
        (当然是近似的, 因为选取的基不完备).
        \begin{asg}
            第6次作业第2题: 用一维无限深势阱展开一维谐振子的能量本征态和四次势的本征态.
        \end{asg}


    \section{量子力学模型体系:谐振子}
    \subsection{一维谐振子的求解(1)}

        一维谐振子的势能函数为:
        \[ V(x) = \frac 12 m\omega^2x^2 \]
        Hamilton算符为:
        \[ \hat{H} = \frac {\hat{p}^2}{2m} + \frac 12 m\omega^2 \hat{x}^2 \]
        类比复数域的:
        \[ a^2 + b^2 = (a+b\mathrm{i})(a-b\mathrm{i}) \]
        对于数字这样分解是可以的, 但是对于算符来说, 只有对易的算符才成立.先考虑数字的情况
        \[ \frac {p^2}{2m} + \frac 12 m\omega^2 x^2 = \frac 12 (\frac p{\sqrt{m}} + \mathrm{i}\sqrt{m}\omega x)(\frac p{\sqrt{m}} - \mathrm{i}\sqrt{m}\omega x) \]
        可以先把能量$\hslash \omega$提出来, 这样就可以操作里面的没有量纲的算符, 会更方便一些.

        这样的话,可以定义算符
        \[ \hat{c} = \frac {\hat{p}}{\sqrt{2m}} + \mathrm{i}\sqrt{\frac m2}\omega \hat{x} \]
        算符的对易满足如下性质: 
        \begin{equation}
            \begin{aligned}
                \left[\hat{A} + \hat{B},\hat{C}\right] &= [\hat{A},\hat{C}]+[\hat{B},\hat{C}]\\
                [\alpha \hat{A}, \beta \hat{B}] &= \alpha \beta [\hat{A},\hat{B}]\\
                [\hat{A}\hat{B},\hat{C}] &= \hat{A}[\hat{B},\hat{C}] + [\hat{A},\hat{C}]\hat{B}
            \end{aligned}
        \end{equation}
        \begin{asg}
            第6次作业第3题(1): 证明上述结论
        \end{asg}
        可以计算:
        \begin{equation}
            \begin{aligned}
                \left[\hat{c},\hat{c}^\dagger\right] &= [d_1\hat{p} +\mathrm{i}d_2\hat{x}, d_1\hat{p} - \mathrm{i}d_2\hat{x}]\\
                &= [d_1\hat{p}, d_1\hat{p} - \mathrm{i}d_2\hat{x}] + [\mathrm{i}d_2\hat{x}, d_1\hat{p} - \mathrm{i}d_2\hat{x}]\\
                &= [d_1\hat{p}, -\mathrm{i}d_2\hat{x}] + [\mathrm{i}d_2\hat{x}, d_1\hat{p}]\\
                &= -\mathrm{i}d_1d_2[\hat{p},\hat{x}] + \mathrm{i}d_1d_2[\hat{x},\hat{p}]\\
                &= -2d_1d_2\hslash\\
                &= -\hslash \omega
            \end{aligned}
        \end{equation}
        为了让算符无量纲化, 定义:
        \begin{equation}\begin{aligned}
            \hat{a} = \frac {\hat{c}}{\sqrt{\hslash \omega}} =\frac 1{\sqrt{2}} \bigg(\frac {\hat{p}}{\sqrt{m\hslash\omega}} + \mathrm{i}\sqrt{\frac {m\omega}{\hslash}} \hat{x}\bigg)
        \end{aligned}\end{equation}
        \begin{asg}
            第6次作业第3题(2): 证明$\hat{a}\hat{a}^\dagger$和$\hat{a}^\dagger\hat{a}$都是Hermite算符.
        \end{asg}
        显然
        \[ [\hat{a},\hat{a}^\dagger] = \hat{a}\hat{a}^\dagger - \hat{a}^\dagger \hat{a} =  -1 \]
        可以用$\hat{a}$写出Hamilton算符: 
        \[ \hat{H} = \frac {\hslash \omega}2 (\hat{a}\hat{a}^\dagger + \hat{a}^\dagger \hat{a}) \]
        可以计算
        \begin{equation}\begin{aligned}
            \left[\hat{a}^\dagger\hat{a}, \hat{a}\hat{a}^\dagger\right] &= \hat{a}^\dagger [\hat{a},\hat{a}\hat{a}^\dagger] +  [\hat{a}^\dagger,\hat{a}\hat{a}^\dagger]\hat{a}\\
            &= \hat{a}^\dagger\hat{a}[\hat{a},\hat{a}^\dagger] + [\hat{a}^\dagger,\hat{a}]\hat{a}^\dagger \hat{a}\\
            &= 0
        \end{aligned}\end{equation}
        \begin{asg}
            第6次作业第3题(3): 证明
            \[ \hat{H} = \frac {\hslash \omega}2 (\hat{a}\hat{a}^\dagger + \hat{a}^\dagger \hat{a}) \]
        \end{asg}
        上面结果也给出了:
        \[ \hat{a}^\dagger\hat{a} = \hat{a}\hat{a}^\dagger + 1 \]
        于是:
        \[ \hat{H} = \hslash \omega\bigg(\hat{a}\hat{a}^\dagger + \frac 12\bigg)\]
        现在定义:
        \[ \hat{b} = \frac 1{\sqrt{2}}\bigg(\sqrt{\frac {m\omega}{\hslash}}\hat{x} + \frac {\mathrm{i}\hat{p}}{\sqrt{m\omega\hslash}}\bigg) \]
        用相同的方法得到:
        \[ \hat{H} = \hslash \omega\bigg(\hat{b}^\dagger\hat{b}+ \frac 12\bigg) \]
        并且:
        \[ [\hat{b}, \hat{b}^\dagger] = 1 \]
        定义:
        \[ \hat{N} = \hat{b}^\dagger\hat{b} \]
        于是:
        \[ \hat{H} = \hslash \omega \bigg(\hat{N}+\frac 12\bigg) \]
        因此:
        \[ [\hat{N},\hat{H}] = 0 \]
        两个对易的算符有相同的本征态. 假设:
        \[ \hat{N}|\phi_n \rangle = \lambda_n |\phi_n \rangle \]
        则:
        \begin{equation}\begin{aligned}
            \hat{H}|\phi_n \rangle = \bigg(\hat{N}+\frac 12\bigg)\hslash\omega|\phi_n \rangle = \bigg(\lambda_n + \frac 12\bigg)\hslash\omega|\phi_n\rangle
        \end{aligned}\end{equation}
        并且:
        \[ \langle \phi_n|\hat{N}|\phi_n \rangle = \lambda_n \langle \phi_n |\phi_n \rangle = \lambda_n \]
        而:
        \[ \langle \phi_n|\hat{N}|\phi_n \rangle =  \langle \phi_n |\hat{b}^\dagger\hat{b}|\phi_n \rangle = \lambda_n \]
        令:
        \[ |\psi_n \rangle = \hat{b}|\phi_n\rangle \]
        那么:
        \[ \langle \phi_n|\hat{N}|\phi_n \rangle =  \langle \phi_n |\hat{b}^\dagger\hat{b}|\phi_n \rangle = \langle \psi_n|\psi_n \rangle = \lambda_n \geqslant 0 \]
        这样证明了$\hat{N}$的本征值必然是非负数.

        那么$|\psi_n\rangle$是否仍然是$\hat{N}$的本征态呢?计算:
        \begin{equation}\begin{aligned}
            \hat{N}|\psi_n\rangle = \hat{b}^\dagger\hat{b}^2|\phi_n\rangle = (\hat{b}\hat{b}^\dagger\hat{b}- \hat{b})|\phi_n\rangle = \hat{b}(\hat{N}-\hat{I})|\phi_n\rangle = (\lambda_n-1)\hat{b}|\phi_n\rangle = (\lambda_n - 1)|\psi_n\rangle
        \end{aligned}\end{equation}
        这说明$\hat{b}$作用于$\hat{N}$的本征态以后得到的态仍然是$\hat{N}$的本征态, 
        且本征值减少1. 于是可以设:
        \[ \hat{b}|\phi_n\rangle = |\psi_n\rangle = \sqrt{\lambda_n}|\phi_m\rangle \]
        将$\hat{N}$作用上来, 得到:
        \[ \hat{N}\sqrt{\lambda_n}|\phi_m \rangle = \sqrt{\lambda_n}(\lambda_n-1)|\phi_m\rangle \]
        这构造了一个循环, 将$\hat{b}$作用在$\hat{N}$的本征态上, 
        得到一个新的$\hat{N}$的本征态, 且$\hat{N}$的本征值减少1, 
        并且它仍然是非负的. 依次类推, 总会有一个态$\hat{N}$的本征值为0(否则得到负的本征值), 
        那么$\hat{N}$的本征值都是自然数. 将$\hat{N}$本征值为0的本征态记为$\ket{\phi_0}$, 
        此时如果再用$\hat{b}$作用, 则得到零向量. 所以, 可以用本征值作为下标来标记本征态:
        \[ \hat{N}|\phi_n \rangle = n|\phi_n\rangle \]
        故将$\hat{N}$称为\textbf{数目算符}.

    \subsection{一维谐振子的求解(2)}
        一维谐振子的能量本征态在通过$\hat{b}$算符的作用时, $\hat{N}$的本征值下降1, 
        故把$\hat{b}$称为\textbf{下降算符}. 最终本征值下降到0时, 应有: 
        \[ \hat{b}|\phi_0 \rangle = 0 \]
        可以推出:
        \[ \hat{N}|\phi_0 \rangle = 0 \]
        求解:
        \[ \langle x|\hat{b}|\phi_0 \rangle = 0\]
        将$\hat{b}$的定义代入, 得到:
        \begin{equation}\begin{aligned}
            \langle x|\sqrt{\frac 12}\bigg(\sqrt{\frac {m\omega}{\hslash}}\hat{x}+ \frac {\mathrm{i}\hat{p}}{\sqrt{m\omega\hslash}}\bigg)|\phi_0 \rangle = 0
        \end{aligned}\end{equation}
        解得:
        \[ \langle x|\phi_0\rangle = \bigg(\frac {m\omega}{\pi\hslash} \bigg)^{\frac 14} \mathrm{e}^{-\frac {m\omega}{2\hslash}x^2} \]
        求出基态的能量:
        \[ \hat{H}|\phi_0 \rangle = \frac 12 \hslash \omega |\phi_0 \rangle \]
        基态也有一定的能量, 称为\textbf{零点能}.

        现在研究一下$\hat{b}^\dagger$作用于$\hat{N}$的本征态.
        \begin{equation}\begin{aligned}
            \hat{N}\hat{b}^\dagger |\phi_n \rangle = (\hat{b}^\dagger \hat{b})\hat{b}^\dagger |\phi_n \rangle = \hat{b}^\dagger (\hat{b}^\dagger\hat{b}+\hat{I})|\phi_n \rangle = (\lambda_n+1) \hat{b}^\dagger |\phi_n \rangle
        \end{aligned}\end{equation}
        所以, $\hat{b}^\dagger$作用在$\hat{N}$的本征态上还会得到$\hat{N}$的本征态, 
        会使得$\hat{N}$的本征值上升1, 于是将$\hat{b}^\dagger$称为\textbf{上升算符}.

        同理可以得到:
        \[ \hat{b}^\dagger|\phi_n\rangle = \sqrt{\lambda_n+1}|\phi_m\rangle \]
        如果想要得到各个激发态的波函数, 可以通过用$\hat{b}^\dagger$不断作用在基态的
        波函数上面: 
        \begin{equation}
            \begin{split}
                \phi_{n}(x) &= \braket{x|\phi_n}\\
                 &= \braket{x|\frac{(\hat{b}^{\dagger})^n}{\sqrt{n}}|\phi_0}\\
                 &= \frac{1}{\sqrt{2^n n!}}\braket{x|\left(\sqrt{\frac{m\omega}{\hslash}}\hat{x} - \frac{\ii\hat{p}}{\sqrt{m\omega\hslash}}\right)|\phi_0}\\
                 &= \left(\frac{m\omega}{\pi\hslash}\right)^{\frac{1}{4}}\frac{1}{\sqrt{2^n n!}}\left(\sqrt{\frac{m\omega}{\hslash}}x - \sqrt{\frac{\hslash}{m\omega}}\dv{}{x}\right)\exp\left(-\frac{m\omega}{2\hslash}x^2\right)
            \end{split}
        \end{equation}
        这样我们就得到了除了无限深势阱本征态函数之外的另一个本征函数集,可以更方便地用于
        数值计算无界空间中的问题. 
        \begin{asg}
            第6次作业第4题: 求出一维谐振子的第$n$个能级的波函数及其势能.
        \end{asg}

    \subsection{时间演化问题}
        有了一维谐振子的解, 我们可以拓展到多维, 类比之前的简谐振动分析, 得到能量的本征值为
        \[ E = \sum_{j=1}^F \bigg(n_j+\frac 12\bigg)\hslash \omega_j \]
        在简正坐标下的波函数为
        \[ \langle \bm{Q}|\psi \rangle = \prod_{j=1}^F \langle Q_j | \phi_{n_j} \rangle \]
        其中
        \[ \langle Q_j|\phi_{n_j} \rangle = \bigg(\frac {\omega}{\pi \hslash}\bigg)^{\frac 14} \mathrm{e}^{-\frac {\omega_j}{2\hslash} Q_j^2} \]
        注意此处没有质量, 因为它被概率在简正坐标变换时引入的Jacobi行列式约掉了.

        \splitline

        如果Hamilton函数为两个Hamilton函数之和
        \footnote{注意,这里只是一种形象的说法,
        我们真正关心的是怎么从小系统的Hamilton量出发构建大系统的Hamilton量,怎么将小空间的Hamilton
        算符定义在更大的空间上}
        : 
        \begin{equation}
            \begin{split}
            \hat{H} = \hat{H}_1 + \hat{H}_2 = \hat{H}_1\otimes\mb{I_2} + \mb{I_1}\otimes\hat{H}_2
            \end{split}
        \end{equation}

        设它们本征态为$\phi_{n_1}^{(1)}$和$\phi_{n_2}^{(2)}$
        于是:
        \[\hat{H} |\phi_{n_1}^{(1)}\rangle\otimes\ket{\phi_{n_2}^{(2)}} = \epsilon_{n_1}|\phi_{n_1}^{(1)}\rangle\otimes\ket{\phi_{n_2}^{(2)}} + \epsilon_{n_2}|\phi_{n_1}^{(1)} \rangle \otimes \ket{\phi_{n_2}^{(2)}}  = (\epsilon_{n_1}+\epsilon_{n_2})|\phi_{n_1}^{(1)} \rangle \otimes\ket{\phi_{n_2}^{(2)}} \]
        更普遍地, 对于多维谐振子, 应有:
        \[ \hat{H} = \sum_{j=1}^F \hat{H}_j \]
        其中 :
        \[ \hat{H}_j = \frac 12 \hat{P}_j^2 + \frac 12 \omega_j^2 Q_j^2 \]
        如果我们已知了:
        \[ \hat{H}|\phi_n\rangle = \epsilon_n|\phi_n\rangle \]
        那么:
        \[ \mathrm{e}^{-\frac {\mathrm{i}\hat{H}t}{\hslash}}|\phi_n\rangle = \mathrm{e}^{-\frac {\mathrm{i}\epsilon_n t}{\hslash}}|\phi_n\rangle \]
        含时的Schrodinger方程
        \footnote{可能前面并没有提到,含时Schrodinger方程也是量子力学中的基本假设}
        为:
        \[ \mathrm{i}\hslash \frac {\partial}{\partial t}|\psi(t)\rangle = \hat{H} |\psi(t)\rangle \]
        注意时间在量子力学中并没有算符, 而是一个参量. 由含时的Schrodinger方程可以推出
        \footnote{下面形式的时间演化算符要求Hamilton量不显含时间}
        :
        \[ \mathrm{e}^{-\frac {\mathrm{i}\hat{H}t}{\hslash}}|\psi(0)\rangle = |\psi(t) \rangle \]
        所以把$\mathrm{e}^{-\frac {\mathrm{i}\hat{H}t}{\hslash}}$称为
        \textbf{时间演化算符}.可以得到任意物理量在时间$t$的平均值:
        \[ \langle \hat{B}(t) \rangle = \langle \psi(t)|\hat{B} | \psi(t) \rangle = \langle \psi(0) |\mathrm{e}^{\frac {\mathrm{i}\hat{H}t}{\hslash}} \hat{B} \mathrm{e}^{-\frac {\mathrm{i}\hat{H}t}{\hslash}}|\psi(0) \rangle \]
        定义$\mathrm{e}^{\frac {\mathrm{i}\hat{H}t}{\hslash}} \hat{B} \mathrm{e}^{-\frac {\mathrm{i}\hat{H}t}{\hslash}}$
        为算符$\hat{B}$的\textbf{Heisenberg算符},
         与考虑系统对应的态随时间演化的Schrodinger绘景不同, 
         在Heisenberg绘景中描述系统的态并不发生变化, 而是算符随时间演化, 
         运动方程可以写为:  
        \begin{equation}
            \dv{\hat{B}(t)}{t} = \frac{1}{\ii \hslash}[\hat{B}(t), \hat{H}]
        \end{equation}
        这个方程称为Heisenberg方程. 
        时间演化算符可以用能量本征态表示:
        \[ \mathrm{e}^{-\frac {\mathrm{i}\hat{H}t}{\hslash}} = \sum_n \mathrm{e}^{-\frac {\mathrm{i}\epsilon_n t}{\hslash}} |\phi_n \rangle \langle\phi_n| \]
        代入, 得到:
        \begin{equation}\begin{aligned}
            \langle \hat{B}(t) \rangle &= \langle \psi(t)|\hat{B} | \psi(t) \rangle\\
            &= \langle \psi(0) |\mathrm{e}^{\frac {\mathrm{i}\hat{H}t}{\hslash}} \hat{B} \mathrm{e}^{-\frac {\mathrm{i}\hat{H}t}{\hslash}}|\psi(0) \rangle\\
            &= \langle \psi(0) |\sum_n \mathrm{e}^{\frac {\mathrm{i}\epsilon_n t}{\hslash}} |\phi_n \rangle \langle\phi_n|\hat{B}|\sum_m \mathrm{e}^{-\frac {\mathrm{i}\epsilon_m t}{\hslash}} |\phi_m \rangle \langle\phi_m|\psi(0)\rangle\\
            &= \sum_{m,n} \langle \psi(0)|\phi_n\rangle \langle \phi_n|\hat{B}|\phi_m \rangle \langle \phi_m|\psi(0)\rangle \mathrm{e}^{\frac {\mathrm{i}(\epsilon_n-\epsilon_m)t}{\hslash}}
        \end{aligned}\end{equation}
        \begin{asg}
            第7次作业第1题: 高维谐振子$t$时刻的物理量
        \end{asg}

    \subsection{经典和量子情形的比较}
    
    \begin{asg}
        第7次作业第2题: 高维谐振子的时间自关联函数计算
    \end{asg}

    比较一下谐振子的经典和量子描述.经典配分函数为:
    \[ Z_\mathrm{cl} = \int \mathrm{e}^{-\beta H(x,p)}\frac {\mathrm{d}x\mathrm{d}p}{2\pi\hslash} = \frac 1{\beta\hslash\omega} \]
    量子体系的配分函数为:
    \[ Z_\mathrm{Q} = \mathrm{Tr} \ \mathrm{e}^{-\beta \hat{H}} = \sum_n \mathrm{e}^{-\beta\epsilon_n} = \frac 1{2\sinh{\frac {\beta\hslash\omega}2}} \]
    这两个结果在$\beta\hslash\omega \to 0$时是一致的.如果:
    $\beta \to 0$则对应高温极限; $\hslash \to 0$对应经典极限; 
    $\omega \to 0$对应能级差很小, 这三种情形都使得量子力学情形趋近于经典情形.
    根据$m\omega^2 = k$, 增大约化质量会使得$\omega$减小, 这就是同位素效应.

    有了配分函数就可以求出各个热力学函数. 定义: 
    \[ u = \beta\hslash\omega \]
    自由能为:
    \begin{equation}\begin{aligned}
        F_\mathrm{cl} &= -\frac 1{\beta} \ln{Z_\mathrm{cl}} = \frac 1{\beta} \ln{u}\\
        F_\mathrm{Q} &= -\frac 1{\beta} \ln{Z_\mathrm{Q}} = \frac 1{\beta}\ln{\sinh{\frac u2}}+ \frac {\ln{2}}{\beta}
    \end{aligned}\end{equation}
    熵为:
    \begin{equation}\begin{aligned}
        S_\mathrm{cl} &= -\bigg(\frac {\partial F_{\mathrm{cl}}}{\partial T}\bigg)_V = -k_B \ln{u} + k_B\\
        S_\mathrm{Q} &= -\bigg(\frac {\partial F_{\mathrm{Q}}}{\partial T}\bigg)_V = -k_B \ln{\sinh{\frac u2}} + \frac {k_Bu}2 \coth{\frac u2} - k_B \ln{2}
    \end{aligned}\end{equation}
    内能为:
    \begin{equation}\begin{aligned}
        U_\mathrm{cl} &= -\frac {\partial}{\partial \beta}\ln{Z_\mathrm{cl}} = \frac 1{\beta}\\
        U_\mathrm{Q} &= -\frac {\partial}{\partial \beta}\ln{Z_\mathrm{Q}} = \frac {u}{2\beta} \coth{\frac u2}
    \end{aligned}\end{equation}
    热容为:
    \begin{equation}\begin{aligned}
        C_{V\mathrm{cl}} = -k_B \beta^2 \bigg(\frac {\partial U_\mathrm{cl}}{\partial T}) &= k_B\\
        C_{V\mathrm{Q}} = -k_B \beta^2 \bigg(\frac {\partial U_\mathrm{Q}}{\partial T}) &= k_B \frac {(\frac u2)^2}{\sinh^2{\frac u2}}
    \end{aligned}\end{equation}
    定义\textbf{量子校正因子}:
    \[ Q(\frac u2) = \frac u2 \coth{\frac u2} \]
    于是 :
    \[ \langle \hat{H} \rangle = U = \frac {Q(\frac u2)}{\beta} \]
    当$u \to 0$, $Q(\frac u2) \to 1$, 接近经典结果; 当$u \to +\infty$, $\frac {Q(\frac u2)}{\frac u2} \to 1$, 于是
    \[U \to \frac {\hslash \omega}2 \]
    能量即为零点能, 即系统几乎完全分布在基态上.
    对于热容, 如果$u \to +\infty$则有$C_V \to 0$。

    \subsection{谐振子运动的量子对应:相干态}

    将一个谐振子从平衡位置拉开,它才能进行简谐振动。在量子力学中,我们也希望找到这样的对应。
    为此,我们就利用平移算符将一个谐振子从平衡位置平移,得到一个态,我们称之为\(\ket{\alpha}\):

    \[
        \ket{\alpha} = \hat{D}(x_0)\ket{0}
    \]

    如果我们认为谐振子基态的平衡位置是0,那么经过平移以后的波函数将\textbf{不再是谐振子的基态},
    甚至不再是谐振子的本征态。因此,有必要对其波函数进行一定的展开:

    \[
        \braket{x | \hat{D}(x_0) | 0 } = \sum_n \braket{x | n} \braket{n | \hat{D}(x_0) | 0}
    \]

    为了更进一步,我们将平移算符显式地表达出来,并对于谐振子体系,改用升降算符表示:
    \[
        \begin{aligned}
        \ket{\alpha} &= \e^{-\frac{\ii \hat{p} x_0}{\hbar}} \ket{0}
        \\ &= \e^{-\frac{\ii x_0}{\hbar} (-\ii)\sqrt{\frac{\hbar m \omega}{2}} (\hat{a} - \hat{a}^\dagger)} \ket{0}
        \\ &= \e^{-\frac{x_0}{\sqrt{2}\sigma} (\hat{a}^\dagger - \hat{a})} \ket{0}
        \\ &= \e^{-\frac{x_0^2}{4\sigma^2}} \e^{\frac{x_0}{\sqrt{2}\sigma} \hat{a}^\dagger} \e^{-\frac{x_0}{\sqrt{2}\sigma} \hat{a}} \ket{0}
        \end{aligned}
    \]
    这里\(\sigma = \sqrt{\frac{\hbar}{m\omega}}\)。
    %其中,最后一个等式用到了关系\( \e^{\hat{A} + \hat{B}} = \e^\hat{A} \e^\hat{B} \e^{ \frac{1}{2} [\hat{A}, \hat{B}] } \),
    %它是Glauber公式,可以通过向\(\hat{A}, \hat{B}\)前乘以系数\(\lambda\)求导证明,详情可以参考曾谨言《量子力学(卷II)》(第四版)108页。

    由于\[\e^{\frac{x_0}{\sqrt{2}\sigma} \hat{a}} \ket{0} = \sum_n \frac{1}{n!} \left(\frac{x_0}{\sqrt{2}\sigma}\right)^n \hat{a}^n \ket{0} = \ket{0} \]
    所以
    \[
        \begin{aligned}
        \ket{\alpha} &= \e^{-\frac{x_0^2}{4\sigma^2}} \e^{-\frac{x_0}{\sqrt{2}\sigma} \hat{a}^\dagger} \ket{0}
        \\ &= \sum_n \frac{1}{n!} \e^{-\frac{x_0^2}{4\sigma^2}} \left(\frac{x_0}{\sqrt{2}\sigma}\right)^n (\hat{a}^\dagger)^n \ket{0}
        \\ &= \sum_n \frac{1}{\sqrt{2^n n!}}\exp\left(-\frac{x_0^2}{4\sigma^2}\right) \left(\frac{x_0}{\sigma}\right)^n \ket{n}
        \end{aligned}
    \]

    引入简写\(\alpha = x_0 / \sqrt{2}\sigma\),则有
    \begin{equation}\label{expansion}
        \braket{x | \alpha} = \sum_n \frac{1}{\sqrt{n!}}\exp\left(-\frac{\alpha^2}{2}\right) \alpha^n \braket{x|n}
    \end{equation}

    通过以上的计算,我们得到了态\(\ket{\alpha}\)在谐振子本征态上的表示。现在,继续考虑它的时间演化:

    \[
        \begin{aligned}
            \hat{U}(t)\ket{\alpha} &= \sum_n \frac{1}{\sqrt{n!}}\exp\left(-\frac{\alpha^2}{2}\right) \alpha^n 
            \e^{-\frac{\ii E_n t}{\hbar}}\ket{n}
            \\ &= \sum_n \frac{1}{\sqrt{n!}}\exp\left(-\frac{\alpha^2}{2}\right) \alpha^n 
            \e^{-\ii n\omega t}\e^{-\ii \omega t / 2} \ket{n}
            \\ &= \e^{-\ii \omega t / 2}\sum_n \frac{1}{\sqrt{n!}}\exp\left(-\frac{|\alpha\e^{-\ii\omega t}|^2}{2}\right) (\alpha\e^{-\ii\omega t})^n
            \\ &= \e^{-\ii \omega t / 2}\ket{\alpha(t)}
        \end{aligned}
    \]
    其中\(\alpha(t) = \alpha\e^{-\ii\omega t}\)。从而可以看出,\(\ket{\alpha}\)态含时演化的结果是得到\(\ket{\alpha \e^{-\ii\omega t}}\)态。
    当然,前面应当还有一个\(\e^{-\ii \omega t / 2}\)作为相位因子。
    这说明,谐振子相干态不论演化到何时,其永远是相干态。

    然后再考察波函数的变化。由于
    \[
        \braket{x | \alpha} = \sqrt{\frac{1}{\sigma\sqrt{\pi}}} \e^{-\left(\frac{x - x_0}{\sqrt{2}\sigma}\right)^2}
    \]
    且\(x_0(t) = \sqrt{2}\sigma \alpha(t)\),所以
    \[
        \braket{x | \alpha(t)} = \e^{-\ii \omega t / 2} \sqrt{\frac{1}{\sigma \sqrt{\pi}}} 
        \exp \left[-\left(\frac{x - x_0\e^{-\ii\omega t}}{\sqrt{2}\sigma}\right)^2\right]
    \]

    概率密度
    \[
        |\braket{x | \alpha(t)}|^2 = \frac{1}{\sigma\sqrt{\pi}}
        \exp\left[-\frac{(x - x_0\cos\omega t)^2}{\sigma^2}\right] 
    \]
    从这个结果中可以看到,谐振子相干态在运动过程中,概率密度始终是一个Gauss波包,且波包不展宽;
    其平衡位置\(x_0\)的周期性变化就像经典谐振子一样。
    由于谐振子基态满足不确定性原理\(\braket{(\Delta x)^2} \braket{(\Delta p)^2} \geq \frac{\hbar^2}{4}\) 的等号(读者可以自行计算验证),因此
    相干态在运动过程中始终满足最小的不确定性关系。

    使用Heisenberg表象可以更方便地进行期望值\(\braket{x(t)}, \braket{p(t)}\)的计算,可参考《中级物理化学》第五章有关习题,请读者自行完成。

    谐振子相干态有若干性质:例如其是湮灭算符\(\hat{a}\)的本征态,对应的本征值即为复数\(\alpha\);
    所有的\(\ket{\alpha}\)可以作为一个超完备的基组组成相干态表象,等等。相关内容可以参考曾谨言《量子力学》第五版卷II。


    \bibliographystyle{plain}
    \bibliography{ref_quantum}


    \chapter{相空间}
    \section{重新审视Hamilton方程}
    \subsection{将Hamilton方程写为对称的形式}
    由于Hamilton方程是一阶微分方程组,其变量$\bm{x},\bm{p}$拥有比牛顿方程中位置和速度更加平等的地位
    \footnote{正则变量之间平等地位更应该通过\cite{Goldstein2000Classical}来说明},
    所以我们希望能将Hamilton方程写为对称的形式.对于一个$n$自由度的力学系统
    \footnote{
        即$\bm{x}$是一个n维向量
    }
    ,考虑引入一个新变量$\bm{\eta}:= (\bm{x}, \bm{p})$
    ,那么正则方程可以写为:
    \begin{equation}
        \left\{
        \begin{split}
            \dot{\eta}_{i} &= \dot{x}_i = \pdv{H}{p_i} = \pdv{H}{\eta_{i+n}}\\
            \dot{\eta_{i+n}} &= \dot{p}_i = - \pdv{H}{x_i} = -\pdv{H}{\eta_{i}}
        \end{split}
        \right.
        \quad\quad\quad i = 1,\cdots, n
    \end{equation}
    引入$2n\times 2n$方阵$\bm{J}$:
    \begin{equation}
        \bm{J} = 
        \begin{bmatrix}
            \bm{0} & \bm{I}_n\\
            -\bm{I}_n & \bm{0}
        \end{bmatrix}
    \end{equation}
    那么Hamilton方程可以写为:
    \begin{equation}
        \dot{\bm{\eta}} = \bm{J}\pdv{H}{\bm{\eta}}
        \label{canonical equation}
    \end{equation}
    \subsection{动力系统的概念}
    这一节讨论一下比Hamilton方程更加普遍的情况.考虑如下微分方程组(Hamilton方程显然是下面所述
    微分方程的一个特例):
    \begin{equation}
        \dv{\bm{x}}{t} = \bm{v}(\bm{x})
        \label{dynamical system}
    \end{equation}
    其中$\bm{v}$是一个$\mathbb{R}^{n}\to\mathbb{R}^{n}$的连续可微映射.这个方程可以看作
    一个在给定速度场中运动的粒子的运动方程.对于任何给定的初始条件:
    \begin{equation}
        \bm{x}(t_0) = \bm{x_0}
        \label{initial condition}
    \end{equation}
    方程满足微分方程解的存在唯一性定理\cite{丁同仁2004常微分方程教程},因此对于初始条件有唯一解:
    \begin{equation}
        \bm{x} = \bm{\phi}(t;t_0, \bm{x}_0)
    \end{equation}
    我们将$\bm{\phi}$在给定$t_0,\bm{x}_0$时作为$t$的函数的“函数图像”
    \footnote{即集合$\{(\bm{\phi}(t;t_0,\bm{x}_0), t)|t>t_0\}$}称为积分曲线.
    其中$\bm{x}$取值的空间$\mathbb{R}^n$称为\textbf{相空间}.给定初始条件之后微分方程的解
    给出了相空间中一条以$t$为参数并且在$t_0$时间通过$\bm{x}_0$的曲线,称之为\textbf{轨线}.
    从直观上讲,这个曲线就是我们能看到的在速度场中运动的粒子所走过的轨迹.一般称这样的方程为动力系统
    (注意速度场并不显含时间).
    \par 下面介绍几个动力系统的基本性质,有助于我们理解本节讲授的Liouville定理.
    \par (1) 积分曲线的平移不变性.考虑$\bm{x} = \bm{\phi}(t;t_0,\bm{x}_0)$是初始条件\ref{initial condition}
    对应的解,那么$\bm{x} = \bm{\phi}(t;t_0 + C,\bm{x}_0)$依然满足方程(但是不再满足初始条件),
    这个结论的正确性是由速度场不含时间保证的,可以直接将解带入微分方程进行验证.容易想象,时间平移之后
    相空间中的轨线完全没有变化,可以形象地理解,在不含时速度场中粒子运动的轨迹只取决于粒子的初始位置
    而与粒子开始运动的时间无关.
    \par (2) 过相空间每一点轨线的唯一性. 这是动力系统(同样是Hamilton方程)的一个重要性质,
    它说明了从相空间中不同点出发的轨线不可能相交.考虑两个不同初始条件的积分曲线
    $\bm{\phi}(t;t_1,\bm{x}_1),\, \bm{\phi}(t; t_2, \bm{x}_2)$在相空间中相交于同一点$\bm{x}'$,
    那么可以对其中一个曲线进行时间平移,使得在某个时刻$t'$两个积分曲线相交于$(\bm{x}', t')$,由微分
    方程解的存在唯一性定理,这两个积分曲线必须完全重合,那么它们对应的相空间中的轨线也必须完全重合,
    这说明了通过相空间中每一点有且只有一条轨线.
    \par (3) 相流.考虑将初始条件设为$t=0,\, \bm{x}(0) = \bm{x}_0$,定义
    $\mathbb{R}^n \to \mathbb{R}^n$的映射:
    \begin{equation}
        \phi^{t}: \bm{x}_{0} \to \bm{\phi}(t; 0, \bm{x}_0)
    \end{equation}
    这个映射将$t=0$时粒子在相空间中的位置映射为粒子沿着轨线运动$t$时间后粒子在相空间中的位置.由轨线
    的唯一性可知,对于$\forall t$映射$\phi^t$是一个双射.更进一步,由于已经假设了$\bm{v}(\bm{x})$
    是一个连续可微映射,因此$\bm{\phi}(t; 0, \bm{x}_0)$对于初值$\bm{x}_0$也是连续可微的
    \cite{丁同仁2004常微分方程教程2},那么映射$\phi^t$是$\mathbb{R}^n$上的一个微分同胚.事实上,
    容易根据微分方程解的存在唯一性证明$\forall s, t\in \mathbb{R}$:
    \begin{equation}
        \phi^{s + t} = \phi^s\circ\phi^{t}
        \label{def of phase flow}
    \end{equation}
    这样$\phi^t$组成的集合就有了群的结构
    \footnote{严格来讲,应该验证存在单位元、逆元等,使这个集合满足群的定义}
    ,将这样的一个单参数微分同胚群称为\textbf{相流}.

    \subsection{Hamilton系统的稳定性矩阵}
    在这一节中我们讨论改变初值对于轨线的影响.固定初始时刻$t=0$,初始位置为$(\bm{x}_0, \bm{p}_0)$,
    将$t$时刻系统在相空间中的位置写成初始条件的函数(这个函数就是Hamilton方程的解):
    \begin{equation}
        \begin{split}
            \bm{x}_t &= \bm{x}_t(\bm{x}_0, \bm{p}_0)\\
            \bm{p}_t &= \bm{p}_t(\bm{x}_0, \bm{p}_0)
        \end{split}
    \end{equation}
    若初始位置偏离$(\mathrm{d}\bm{x}_0,\mathrm{d}\bm{p}_0)$(假设偏离量是小量),那么
    \begin{equation}
        \begin{split}
        \bm{x}_t(\bm{x}_0+\mathrm{d}\bm{x}_0, \bm{p}_0+\mathrm{d}\bm{p}_0) &= \bm{x}_t(\bm{x}_0, \bm{p}_0) + \frac {\partial \bm{x}_t}{\partial \bm{x_0}} \dd \bm{x}_0 + \frac {\partial \bm{x}_t}{\partial \bm{p}_0} \dd \bm{p}_0\\
        \bm{p}_t(\bm{x}_0+\mathrm{d}\bm{x}_0, \bm{p}_0+\mathrm{d}\bm{p}_0) &= \bm{p}_t(\bm{x}_0, \bm{p}_0) + \frac {\partial \bm{p}_t}{\partial \bm{x_0}} \dd \bm{x}_0 + \frac {\partial \bm{p}_t}{\partial \bm{p}_0} \dd \bm{p}_0
        \end{split}
    \end{equation}
    这里只考虑了Taylor展开到一阶的结果.或者写成
    \begin{equation}
        \begin{split}
            \mathrm{d}\bm{x}_t &= \frac {\partial \bm{x}_t}{\partial \bm{x_0}} \dd \bm{x}_0 + \frac {\partial \bm{x}_t}{\partial \bm{p}_0} \dd \bm{p}_0\\
            \mathrm{d}\bm{p}_t &= \frac {\partial \bm{p}_t}{\partial \bm{x_0}} \dd \bm{x}_0 + \frac {\partial \bm{p}_t}{\partial \bm{p}_0} \dd \bm{p}_0
        \end{split}
        \end{equation}
    现在已经将$t$时刻的位置偏离表示成了0时刻位置偏离的函数\footnote{初始时刻只有无穷小偏离,因此只考虑线性近似}
    ,但是由于Hamilton方程的解是未知的,系数矩阵没有办法直接求出来.为了获取系数矩阵的一些信息,我们尝试对时间求导:
    \begin{equation}
        \begin{split}
            \frac{\mathrm{d}}{\mathrm{d}t} \bigg(\frac {\partial \bm{x}_t}{\partial \bm{x}_0}\bigg)_{\bm{p}_0} &= \bigg(\frac {\partial}{\partial \bm{x}_0} \frac {\mathrm{d}\bm{x}_t}{\mathrm{d}t}\bigg)_{\bm{p}_0} = \bigg(\frac {\partial}{\partial \bm{x}_0} \bigg(\frac {\partial H}{\partial \bm{p}_t}\bigg)_{\bm{x}_t}\bigg)_{\bm{p}_0} = \bigg(\frac {\partial^2 H}{\partial \bm{x}_t \partial \bm{p}_t}\bigg) \bigg(\frac {\partial \bm{x}_t}{\partial \bm{x}_0}\bigg)_{\bm{p}_0}+ \bigg(\frac {\partial^2H}{\partial \bm{p}_t^2}\bigg)_{\bm{x}_t} \bigg(\frac {\partial \bm{p}_t}{\partial \bm{x}_0}\bigg)_{\bm{p}_0}\\
            \frac{\mathrm{d}}{\mathrm{d}t} \bigg(\frac {\partial \bm{x}_t}{\partial \bm{p}_0}\bigg)_{\bm{x}_0} &= \bigg(\frac {\partial}{\partial \bm{p}_0} \frac {\mathrm{d}\bm{x}_t}{\mathrm{d}t}\bigg)_{\bm{x}_0} = \bigg(\frac {\partial}{\partial \bm{p}_0} \bigg(\frac {\partial H}{\partial \bm{p}_t}\bigg)_{\bm{x}_t}\bigg)_{\bm{x}_0} = \bigg(\frac {\partial^2 H}{\partial \bm{x}_t \partial \bm{p}_t}\bigg) \bigg(\frac {\partial \bm{x}_t}{\partial \bm{p}_0}\bigg)_{\bm{x}_0} + \bigg(\frac {\partial^2H}{\partial \bm{p}_t^2}\bigg)_{\bm{x}_t} \bigg(\frac {\partial \bm{p}_t}{\partial \bm{p}_0}\bigg)_{\bm{x}_0}\\
            \frac{\mathrm{d}}{\mathrm{d}t} \bigg(\frac {\partial \bm{p}_t}{\partial \bm{x}_0}\bigg)_{\bm{p}_0} &= \bigg(\frac {\partial}{\partial \bm{x}_0} \frac {\mathrm{d}\bm{p}_t}{\mathrm{d}t}\bigg)_{\bm{p}_0} = -\bigg(\frac {\partial}{\partial \bm{x}_0} \bigg(\frac {\partial H}{\partial \bm{x}_t}\bigg)_{\bm{p}_t}\bigg)_{\bm{p}_0} = -\bigg(\frac {\partial^2 H}{\partial \bm{x}_t^2}\bigg)_{\bm{p}t} \bigg(\frac {\partial \bm{x}_t}{\partial \bm{x}_0}\bigg)_{\bm{p}_0}- \bigg(\frac {\partial^2H}{\partial \bm{p}_t \partial \bm{x}_t}\bigg) \bigg(\frac {\partial \bm{p}_t}{\partial \bm{x}_0}\bigg)_{\bm{p}_0}\\
            \frac{\mathrm{d}}{\mathrm{d}t} \bigg(\frac {\partial \bm{p}_t}{\partial \bm{p}_0}\bigg)_{\bm{x}_0} &= \bigg(\frac {\partial}{\partial \bm{p}_0} \frac {\mathrm{d}\bm{p}_t}{\mathrm{d}t}\bigg)_{\bm{x}_0} = -\bigg(\frac {\partial}{\partial \bm{p}_0} \bigg(\frac {\partial H}{\partial \bm{x}_t}\bigg)_{\bm{p}_t}\bigg)_{\bm{x}_0} = -\bigg(\frac {\partial^2 H}{\partial \bm{x}_t^2}\bigg)_{\bm{p}t} \bigg(\frac {\partial \bm{x}_t}{\partial \bm{p}_0}\bigg)_{\bm{x}_0}- \bigg(\frac {\partial^2H}{\partial \bm{p}_t \partial \bm{x}_t}\bigg) \bigg(\frac {\partial \bm{p}_t}{\partial \bm{p}_0}\bigg)_{\bm{x}_0}    
        \end{split}
    \end{equation}
    由此可以得到一个微分方程组:
    \begin{equation}
        \displaystyle
        \begin{split}
            \frac {\mathrm{d}}{\mathrm{d}t}
            \begin{bmatrix}
                \frac {\partial \bm{x}_t}{\partial \bm{x}_0} & \frac {\partial \bm{x}_t}{\partial \bm{p}_0}\\
                \frac {\partial \bm{p}_t}{\partial \bm{x}_0} & \frac {\partial \bm{p}_t}{\partial \bm{p}_0}
            \end{bmatrix}
            =
            \begin{bmatrix}
                \frac {\partial^2 H}{\partial \bm{x}_t \partial \bm{p}_t} & (\frac {\partial^2 H}{\partial \bm{p}_t^2})_{\bm{x}_t}\\
                -(\frac {\partial^2 H}{\partial \bm{x}_t^2})_{\bm{p}_t} & - \frac {\partial^2 H}{\partial \bm{x}_t \partial \bm{p}_t}
            \end{bmatrix}
            \begin{bmatrix}
                \frac {\partial \bm{x}_t}{\partial \bm{x}_0} & \frac {\partial \bm{x}_t}{\partial \bm{p}_0}\\
                \frac {\partial \bm{p}_t}{\partial \bm{x}_0} & \frac {\partial \bm{p}_t}{\partial \bm{p}_0}
            \end{bmatrix}
        \end{split}
    \end{equation}
    等式右边由Hamilton量二阶偏导数组成的矩阵称为Hamilton系统的\textbf{稳定性矩阵},其决定了在相空间
    某一点附近,初值改变引起的\textbf{短时间内}轨线改变的趋势,下面给以简单说明.考虑一般的微分方程组
    \ref{dynamical system},初始条件分别为$t=0, \bm{x}(0) = \bm{x}_0;\, t=0, \bm{x}(0) = \bm{x}_0 + \delta\bm{x}_0$
    ,写出对应的解:
    \begin{equation}
        \begin{split}
            \dot{\bm{\phi}}(t, \bm{x}_0) &= \bm{v}(\bm{\phi}(t, \bm{x}_0))\\
            \dot{\bm{\phi}}(t, \bm{x}_0 + \delta\bm{x}_0) &= \bm{v}(\bm{\phi}(t, \bm{x}_0 + \delta\bm{x}_0))
        \end{split}
    \end{equation}
    将上两式相减,得到改变初值引起的积分曲线的变化所满足的方程:
    \begin{equation}
        \dot{\delta\bm{\phi}} = \left.\pdv{\bm{v}}{\bm{x}}\right|_{\bm{x = \bm{\phi(t, \bm{x}_0)}}} \delta\bm{\phi}
    \end{equation}
    上式是一个线性的微分方程,其系数矩阵显然是时间的函数,如果我们认为在较短的时间内系数矩阵不随时间变化,
    而且取初始位置$\bm{x}_0$的值,那么上述方程在短时间内的解近似为:
    \begin{equation}
        \delta \bm{\phi} \approx \exp\left(\left.\pdv{\bm{v}}{\bm{x}}\right|_{\bm{x} = \bm{x}_0}\right) \delta\bm{x}_0
    \end{equation}
    可以看到短时间内矩阵$\displaystyle\left.\pdv{\bm{v}}{\bm{x}}\right|_{\bm{x} = \bm{x}_0}$的性质决定了$\delta\bm{\phi}$
    的行为(主要是其特征值实部的正负\footnote{若实部最大的特征值实部若大于0,则在该点邻域内初值小的偏移
    都会引起轨线以指数增长的偏移,称这样的区域为轨道的不稳定区域;若特征值实部全部小于0,那么初值小的偏移
    引起的轨线偏移是指数衰减的,称这样的区域为轨道的稳定区域}).对于Hamilton系统,$\displaystyle\bm{v} = \bm{J}\pdv{H}{\bm{\eta}}$,
    容易验证$\displaystyle\pdv{\bm{v}}{\bm{x}}$就是之前定义的Hamilton系统的稳定性矩阵.
    \section{Liouville定理}
    \subsection{Liouville定理的提出}
    前一节已经指出了$\phi^t$是一个$\mathbb{R}^n\to\mathbb{R}^n$的微分同胚,其意义是将0时刻相空间
    中的点映射到$t$时刻相空间中的点.我们关心这个映射的性质,对于Hamilton系统\ref{canonical equation}
    一个重要的性质是$\phi^t$是一个保持体积的映射,具体来说,假设$V(D)$表示
    $D\in\mathbb{R}^n$(前提是$D$的体积有定义)的体积,那么:
    \begin{equation}
        V(\phi^t(D)) = V(D)
    \end{equation}
    这个结论并不需要用到Hamilton系统的全部性质,可以期待$\phi^t$保持了相空间中更多的量.同时也可以探究
    一般的动力系统\ref{dynamical system}中相体积在$\phi^t$下的变化.
    \subsection{Liouville定理的证明}
    $t$时刻$\phi^t(D)$的体积用积分表示为:
    \begin{equation}
        V(\phi^t(D)) = \int_{\phi^t D}\d x_1\d x_2\cdots\d x_n
    \end{equation}
    考虑到重积分的换元公式,可以将$\phi^t(D)$的体积写为:
    \begin{equation}
        V(\phi^t(D)) = \int_{D}\left|\pdv{\phi^t(\bm{x}_0)}{\bm{x}}\right|\d x_1\d x_2\cdots\d x_n
        \label{volum of D}
    \end{equation}
    那么只用知道映射$\phi^t$的Jacobi行列式$\displaystyle\left|\pdv{\phi^t(\bm{x}_0)}{\bm{x}}\right|$的表达式就可以计算出
    $\phi^t$映射下体积的变化.但是这个行列式本身的性质无法直接看出,我们对其求导:
    \begin{equation}
        \begin{split}
            \dv{}{t}\pdv{\phi^t(\bm{x}_0)}{\bm{x}_0} &= \pdv{}{\bm{x}_0}\dv{\bm{\phi}(t, \bm{x}_0)}{t}\\
            & = \pdv{\bm{v}(\bm{\phi}(t, \bm{x}_0))}{\bm{x}_0}\\
            & = \left.\pdv{\bm{v}(\bm{x})}{\bm{x}}\right|_{\bm{x}=\bm{\phi}(t, \bm{x}_0)}\cdot
            \pdv{\phi^t(\bm{x}_0)}{\bm{x}_0}
        \end{split}
        \label{the equation of jacobian}
    \end{equation}
    我们希望通过上面得到的等式来计算Jacobi行列式$\displaystyle\pdv{\phi^t(\bm{x}_0)}{\bm{x}_0}$.
    \par 
    首先给出一个利用矩阵函数性质的证明.考虑方阵$\bm{A}$:
    \begin{equation}
        \bm{A} = 
        \begin{pmatrix}
            a_{11} & \cdots & a_{1n}\\
            \vdots & & \vdots\\
            a_{n1} & \cdots & a_{nn}\\
        \end{pmatrix}
    \end{equation}
    它的行列式可以表示为(行列式按行展开):
    \begin{equation}
        \det{\bm{A}} = \sum_{j=1}^na_{ij} A_{ij}^* \quad\quad i = 1, 2, \cdots, n
    \end{equation}
    其中,$A_{ij}^*$表示$a_{ij}$的代数余子式
    \footnote{方阵$\bm{A}\,$$i,j$元$a_{ij}$余子式$M_{ij}$定义为将方阵$\bm{A}$的i行j列去掉后组成的方阵的行列式,
    容易通过行列式的多重线性性和交错对称性证明:
    \begin{equation}
        \det \bm{A} = \sum_{j = 1}^{n}(-1)^{i + j}a_{ij}\cdot M_{ij}\quad \quad i = 1, 2, \cdots, n
    \end{equation}
    $a_{ij}$的代数余子式$A_{ij}^*$定义为$A_{ij}^* := (-1)^{i+j}M_{ij}$
    }
    .定义$\bm{A}$的\textbf{伴随矩阵}$\bar{\bm{A}}$为:
    \begin{equation}
        \bar{\bm{A}}_{ij} = A_{ji}^* 
    \end{equation}
    矩阵$\bm{A}$的逆矩阵(假设可逆)可以用伴随矩阵$\bar{\bm{A}}$表示为
    \footnote{可以通过矩阵乘法与行列式的性质验证下面所表示的矩阵就是$\bm{A}$的逆矩阵}:
    \begin{equation}
        \bm{A}^{-1} = \frac{1}{\det{\bm{A}}}\bm{\bar{A}}
    \end{equation}
    \par 
    假设方阵$\bm{A}$中的每个元素$a_{ij}$均是$t$的函数,可以想象对于方阵$\bm{A}$求导即是对于每个
    元素求导组成的方阵,但是对$\det \bm{A}$求导并不是这样.利用行列式是每一行(每一列)的多重线性
    函数这个性质
    \footnote{
        设$T(v_1,\cdots,v_n)$是$V\times\cdots\times V \to \mathbb{R}$的多重线性映射,那么:
        \begin{equation}
            \dv{T(v_1,\cdots,v_n)}{t} = T\left(\dv{v_1}{t}, v_2, \cdots, v_n\right) + T\left(v_1, \dv{v_2}{t}, \cdots, v_n\right)
             + \cdots + T\left(v_1, v_2, \cdots, \dv{v_n}{t}\right)
        \end{equation}
        有很多与之相关的例子,比如对n个函数乘积的求导(莱布尼茨法则),对两个向量内积的求导、
        对$\mathbb{R}^3$中向量积的求导、对对易子的求导等等;证明的思路与莱布尼茨法则证明的思路类似.
    },可以给出对行列式求导的结果:
    \begin{equation}
        \frac {\mathrm{d}}{\mathrm{d}t} \det{\bm{A}} = \sum_i \det{\dot{\bm{A}_{i}}}
    \end{equation}
    其中,$\dot{\bm{A}_{i}}$是只对第$i$行的所有元素对时间求导,其他元素不变得到的矩阵.进一步得到
    \begin{equation}
        \begin{split}
            \frac {\mathrm{d}}{\mathrm{d}t} \det{\bm{A}} &= \sum_i \det{\dot{\bm{A}_i}} = \sum_{i=1}^n \sum_{j=1}^n \frac {\mathrm{d}a_{ij}}{\mathrm{d}t} \bm{A}_{ij}^*\\
        &= \mathrm{Tr} \bigg(\frac {\mathrm{d}\bm{A}}{\mathrm{d}t} \bar{\bm{A}}\bigg) = \mathrm{Tr} \bigg(\frac {\mathrm{d}\bm{A}}{\mathrm{d}t} \bm{A}^{-1}\bigg) \det{\bm{A}}
        \end{split}
    \end{equation}
    将两边同时除以$\bm{A}$的行列式,得到一个重要的公式:
    \begin{equation}
        \frac {\mathrm{d}}{\mathrm{d}t} \ln{\det{\bm{A}}} = \mathrm{Tr} \bigg(\frac {\mathrm{d}\bm{A}}{\mathrm{d}t} \bm{A}^{-1}\bigg)
    \end{equation}
    假设$\bm{A}$满足(就是$\phi^t$的Jacobi行列式满足的条件):
    \begin{equation}
        \frac {\mathrm{d}}{\mathrm{d}t} \bm{A} = \bm{M}\cdot\bm{A}
        \label{equation of matrix}
    \end{equation}
    就有:
    \begin{equation}
        \frac {\mathrm{d}}{\mathrm{d}t} \ln{\det{\bm{A}}} = \mathrm{Tr}\ \bm{M}
    \end{equation}
    将这个结论应用于$\phi^t$满足的方程\ref{the equation of jacobian},可以得到:
    \begin{equation}
        \dv{}{t}\ln \left|\pdv{\phi^t(\bm{x}_0)}{\bm{x}_0}\right| = \mr{Tr}\left(\pdv{\bm{v}}{\bm{x}}\right)_{\bm{x} = \bm{\phi}(t, \bm{x}_0)}
    \end{equation}
    将上式两边对时间积分,同时考虑到初始条件$\displaystyle\pdv{\phi^0}{\bm{x}_0} = 1$,可以给出$t$时刻Jacobi矩阵
    的表达式:
    \begin{equation}
        \left|\pdv{\phi^t(\bm{x}_0)}{\bm{x}_{0}}\right| = \exp\left(\int_{0}^{t}\mr{Tr}\left.
        \pdv{\bm{v}}{\bm{x}}\right|_{\bm{x} = \bm{\phi}(u, \bm{x}_0)}\d u\right)
        \label{expression of jacobian}
    \end{equation}
    将上式带入\ref{volum of D}就可以得到理论上的体积公式.可以看出,要计算出Jacobi行列式必须要方程的显式解,
    这在大多数情况下并不能够得到满足,但是当$\displaystyle\mr{Tr}\pdv{\bm{v}}{\bm{x}}$
    与时间无关时,就容易计算出Jacobi行列式.有一种特殊的情况:
    \begin{equation}
        \mr{Tr}\pdv{\bm{v}}{\bm{x}} = 0
    \end{equation}
    此时$\displaystyle\left|\pdv{\phi^t(\bm{x}_0)}{\bm{x}_0}\right| = 1,\, \forall t$,
    满足这个条件的系统的相流中的任何一个元素$\phi^t$均是保持相体积不变的映射,
    这个结论被称为Liouville定理.对于Hamilton系统的相流中某个映射$\phi^t$的Jacobi行列式有
    (参考\ref{eqaution of matrix}中的定义):
    \begin{equation}
        \bm{M} = \bm{J} \pdv{^2 H}{\bm{\eta}^2}=
        \begin{pmatrix}
        \frac {\partial^2 H}{\partial \vec{x}_t \partial \vec{p}_t} & (\frac {\partial^2 H}{\partial \vec{p}_t^2})_{\vec{x}_t}\\
        -(\frac {\partial^2 H}{\partial \vec{x}_t^2})_{\vec{p}_t} & - \frac {\partial^2 H}{\partial \vec{x}_t \partial \vec{p}_t}
        \end{pmatrix}
    \end{equation}
    显然这个矩阵的迹为0,所以
    \begin{equation}
        \frac {\mathrm{d}}{\mathrm{d}t} \det \bigg|\frac {\partial (\bm{x}_t,\bm{p}_t)}{\partial (\bm{x}_0,\bm{p}_0)} \bigg| = 0
    \end{equation}
    所以Jacobi行列式恒为1.由之前得到的结果\ref{volum of D}就可以证明Hamilton相流中任一映射均是保持相体积的映射.
    为了区分0时刻与$t$时刻的相空间,通常将0时刻相空间中的坐标记为$(\bm{x}_0, \bm{p}_0)$,$t$时刻记为
    $(\bm{x}_t,\bm{p}_{t})$\footnote{这样的记法所表达的真正含义是$\phi^t(\bm{x}_0, \bm{p}_0) = (\bm{x}_t, \bm{p}_t)$},
    那么两个由映射$\phi^t$联系起来的空间的体积元之间满足:
    \begin{equation}
        \mathrm{d}\bm{x}_t \mathrm{d}\bm{p}_t = \left|\pdv{\phi^t(\bm{x}_0)}{\bm{x}_0}\right|\mathrm{d}\bm{x}_0 \mathrm{d}\bm{p}_0 = \dd \bm{x}_0\dd \bm{p}_0
    \end{equation}
    上面的等式完全可以看作Liouville定理的另一种表述形式.
    \paragraph{}
    还有另外的证明Liouville定理的方法,将方程\ref{the equation of jacobian}看作一个线性微分方程组,将$\displaystyle\pdv{\phi^t(\bm{x}_0)}{\bm{x}_0}$
    的每一列都看作方程的一个解,那么$\displaystyle\left|\pdv{\phi^t(\bm{x}_0)}{\bm{x}_0}\right|$就是这个微分方程组的
    Wronsky行列式,根据线性微分方程组的Liouville公式
    \footnote{直接对Liouville公式求导并带入微分方程组就可以证明这个结论}
    可以直接得到\ref{expression of jacobian}式的结论.
    \paragraph{}
    还可以根据Hamilton系统的性质,给出一个比较巧妙的证明
    \footnote{
        这里证明了Hamilton相流给出的相空间之间的变换是一个辛变换;相流给出的变换
        是一种正则变换.
    }.首先约定一些记号:
    \begin{equation}
        \begin{split}
            \bm{M}(t) &:= \pdv{\phi^t(\bm{\eta}_0)}{\bm{\eta}_0}\\
            \bm{A}(t) &:= \bm{M}(t)^t\bm{J}\bm{M}(t)
        \end{split}
    \end{equation}
    其中$\dps\bm{M}(t)$就是Hamilton相流代表的变换的Jacobi行列式(定义参见\ref{def of phase flow})
    希望证明$\bm{A}(t)$是一个不随时间变化的矩阵,对其求导,同时带入Hamilton方程\ref{canonical equation}:
    \begin{equation}
        \begin{split}
            \dot{\bm{A}}(t) &= \dot{\bm{M}}(t)^t\bm{J}\bm{M}(t) + \bm{M}(t)^t\bm{J}\dot{\bm{M}}(t)\\
            &= \bm{M}(t)^t\pdv{^2 H}{\bm{\eta}^2}^t \bm{J}^t\bm{J}\bm{M}(t) + \bm{M}(t)^t\bm{J}\bm{J}
            \pdv{^2 H}{\bm{\eta}^2}\bm{M}(t)\\
            &= 0
        \end{split}
    \end{equation}
    这说明$\bm{A}(t)$是一个不随时间变化的矩阵,考虑到$t=0$时$\phi^0$是恒等映射,对应的Jacobi行列式
    为单位矩阵,所以:
    \begin{equation}
        \bm{A}(t) = \bm{A}(0) = \bm{J}
    \end{equation}
    那么对于Hamilton相流$\{\phi^t\}$\footnote{这里表示所有映射$\phi^t$的集合},下式恒成立:
    \begin{equation}
        \bm{J} = \left[\pdv{\phi^t(\bm{\eta}_0)}{\bm{\eta}_0}\right]^t \bm{J} \left[\pdv{\phi^t(\bm{\eta}_0)}{\bm{\eta}_0}\right]
    \end{equation}
    在上式两边取行列式,容易得到$\dps\det\bm{M}(t) = \pm 1$,如果考虑到$t\to\bm{M}(t)$是一个连续的映射
    \footnote{这里的连续性只是修改者的一个感觉,}
    而行列式也是连续的映射,可以根据$t=0$时$\det\bm{M}(0) = 1$确定下$\det\bm{M}(t) = 1$.
    
    \section{Liouville方程}
    \subsection{有关系综的概念}\footnote{这一节的更多内容可以参考\cite{Tuckerman2010Statistical}
    和刘川老师的平衡态统计物理讲义第二章开头}
    统计物理的目标是建立宏观物理量和微观运动规律之间的联系,之前讨论的Hamilton方程可以用来描述经典意义下
    微观系统的运动.某个微观系统的运动状态(某一时刻各个粒子的广义坐标和广义动量)对应相空间中一个点(
    称为系统的\textbf{代表点}),系统随时间的演化等价于代表点在相空间中按照Hamilton方程决定的轨线运动.Maxwell
    与Boltzmann认为,对于宏观系统的测量发生在一段时间$(t_0, t_0 + \tau)$内,其中$\tau$是一个
    宏观短(指系统的宏观物理量没有发生可观测的变化)、微观长(指在这段时间内系统的代表点在相空间中发生了
    很明显的移动)的时间段,而真正观测到的物理量$A(t_0)$是对应微观物理量$a(x,p)$的\textbf{时间平均}:
    \begin{equation}
        A(t_0) := \frac{1}{\tau}\int_{t_0}^{t_0+\tau}a(x(t), p(t))\dd x\dd p
    \end{equation}
    但是我们无法求解由大量粒子(~$10^{23}$)构成的复杂系统的运动方程,无法给出系统的相轨道,
    因此上面的定义难以给出有意义的结论.
    \par 
    Boltzmann通过提出\textbf{各态历经假设}
    \footnote{从数学上讲各态历经假设并不是对于任意力学系统都成立}
    来解决上面遇到的困难,Boltzmann认为对于能量守恒的系统,
    经过足够长(微观上)时间演化后,系统的代表点在等能面上每一个点邻域内都停留相同长的时间.
    利用这种思想,我们可以定义系统在宏观短微观长的时间内在相空间中每一点附近出现的概率$\rho(x, p, t)$,
    那么就能将物理量对时间的平均转化到对相空间的平均:
    \begin{equation}
        A(t_0) = \int\rho(x, p, t_0)a(x, p)\dd x\dd p
    \end{equation}
    随着我们引入相空间中的概率密度函数$\rho(x, p, t)$,我们已经更换了思考问题的角度.我们不再考虑\textbf{一个
    系统}长时间演化过程中在相空间中的分布,而是直接考虑以同样的密度函数分布在相空间中的\textbf{大量完全相同的系统},
    待求的宏观物理量是对应的微观物理量在这些系统上的平均值,可以想象,如果各态历经假设是成立的,那么这两种
    平均应该给出相同的结果.
    \par 
    有了上面的讨论,就将求出任意时刻热力学量的问题转化为了求出任意时刻密度函数$\rho(x, p, t)$的问题,
    这个密度函数被称为\textbf{系综密度函数},下面给出系综的准确定义.
    满足同样宏观条件(如能量、体积、粒子数)的系统有可能处于不同的微观运动状态、对应相空间中不同的代表点
    ,将具有\textbf{相同宏观条件}的所有系统
    \footnote{这些系统必须是“相同的系统”,即是用Hamilton方程所描述的系统,形象地说有着相同数量、种类
    的粒子,并且相互作用也相同;但是这里的相同并不是说运动状态也相同,它们对应着相空间中不同的代表点
    }
    的集合称为此宏观条件下的一个\textbf{系综}.可以定义系综内系统代表点在相空间的归一化密度函数$\rho(x, p, t)$,
    这个函数代表了这个系综内系统的代表点在$(x, p)$邻域出现的概率,
    满足:
    \begin{equation}
        \begin{split}
            &\rho(x, p, t) \geq 0\\
            &\int\rho(x, p, t)\dd x\dd p = 1
        \end{split}
        \label{ensemble density}
    \end{equation}
    宏观系统(宏观物理量)的时间演化用系综随时间的演化来描述:系综中的每一个系统代表点都按照Hamilton方程
    在相空间中运动,代表点会随着时间在相空间中重新分布,对应的系综密度函数也会随时间变化
    \footnote{既然系综内系统的代表点按照Hamilton方程运动,这样引起的系综密度的变化是完全确定的,
    可以用下一节将要介绍的Liouville方程描述}
    ,任意时刻的物理量按照对应时刻的系综密度函数计算.
    \subsection{Liouville方程}
    \footnote{这一节的更多内容可以参考\cite{Tuckerman2010Statistical2}}
    考虑给定初始时刻$t=0$大量系统的代表点在相空间中的一个分布
    \footnote{这样的分布可能是某个系综对应的分布,也有可能只是任意给定的分布,不代表真实的系综}
    ,这个分布可以用一个归一化的密度函数$\rho(x, p, 0)$(满足\ref{ensemble density})来描述,
    这些系统的代表点将在相空间中按照Hamilton方程演化,我们想知道$t$时间后相空间中代表点的密度函数,
    即希望给出$\rho(x, p, 0)$满足的方程.
    \par
    假设给定的分布在相空间中代表点的总数目为$N$,对于0时刻任意相空间中的区域$D$
    \footnote{在我们的讨论中总假设$D$是有体积的},其中代表点的数目$n(D)$为:
    \begin{equation}
        n(D) = N\int_{D}\rho(x, p, 0)\dd x\dd p
    \end{equation}
    考虑$D$内所有点都按照Hamilton方程在相空间中运动,那么$t$时刻$D$将演化为$\phi^t(D)$
    \footnote{$\phi^t$是Hamilton相流中的一个元素,具体定义参见前一章关于相流的讨论},
    在演化过程中,$D$内的代表点不会从中“跑出”,也不会有新的代表点进入,更不会凭空消失,
    因此$\phi^t$内的代表点数目维持不变,即:
    \begin{equation}
        n(\phi^t(D)) = N\int_{\phi^t}\rho(x, p, t)\dd x\dd p = n(D)
    \end{equation}
    那么可以得到:
    \begin{equation}
        \int_{D}\rho(x, p, 0)\dd x\dd p = \int_{\phi^t(D)}\rho(x, p, t)\dd x\dd p
    \end{equation}
    对等式右边应用重积分的换元,利用$\phi^t$将$\phi^t(D)$变换为0时刻的区域$D$(参考\ref{volum of D}),
    同时利用Liouville定理,上面的等式可转化为:
    \begin{equation}
        \int_{D}\rho(x_0, p_0, 0)\dd x_0\dd p_0 = \int_{D}\rho(x_t(x_0, p_0), p_t(x_0, p_0), t)\dd x_0 \dd p_0
    \end{equation}
    上式中$(x_t, p_t)$的精确含义是$(x_t, p_t) = \phi^t(x_0, p_0)$,表示$(x_t, p_t)$是由
    $(x_0, p_0)$按照Hamilton方程演化$t$时间后到达的点.由于上面的区域$D$是任意给定的,那么可以得到:
    \begin{equation}
        \rho(x_t(x_0, p_0), p_t(x_0, p_0), t) = \rho(x_0, p_0, 0)
        \label{Liouville equation Lagrange}
    \end{equation}
    对上面的等式对时间求导\footnote{这里的求导是沿着轨线进行的}:
    \begin{equation}
        \frac{\mathrm{d}\rho}{\mathrm{d}t} = \frac {\partial \rho}{\partial t} + \frac {\partial \rho}{\partial \bm{x}_t}\dot{\bm{x}}_t  + \frac {\partial \rho}{\partial \bm{p}_t} \dot{\bm{p}}_t = 0
    \end{equation}
    再利用正则方程,得到
    \begin{equation}
        - \frac {\partial \rho}{\partial t} = \bigg(\frac {\partial \rho}{\partial \bm{x}}\bigg)^\mathrm{t} \frac {\partial H}{\partial \bm{p}} - 
        \bigg(\frac {\partial \rho}{\partial \bm{p}}\bigg)^\mathrm{t} \frac {\partial H}{\partial \bm{x}}
        \label{Liouville equation Euler}
    \end{equation}
    这个方程被称为Liouville方程.
    \par 
    考虑定义在相空间中的两个函数$A(\bm{x}, \bm{p}),B(x ,p)$,定义这两个函数的\textbf{Poisson括号}
    \footnote{
        可以使用更紧凑的形式表达Poisson括号.考虑Hamilton方程\ref{canonical equation},其中将$(\bm{x}, \bm{p})$
        合并为正则变量$\bm{\eta}$,基于这种考虑,可以将Poisson括号写为:
        \begin{equation}
            \{A, B\} = \left(\pdv{A}{\bm{\eta}}\right)^\mr{t}\bm{J}\left(\pdv{B}{\bm{\eta}}\right)
        \end{equation}
        从这种形式的Poisson括号更容易看出其是正则变换下的不变量.
    }
    :
    \begin{equation}
        \{A, B\} := \left(\pdv{A}{\bm{x}}\right)^\mr{t}\left(\pdv{B}{\bm{p}}\right) - \left(\pdv{A}{\bm{p}}\right)^\mr{t}\left(\pdv{B}{\bm{x}}\right)
    \end{equation}
    利用Poisson括号重写Liouville方程:
    \begin{equation}
        - \frac {\partial \rho}{\partial t} = \{ \rho, H\}
    \end{equation}
    得到Liouville方程的另一种表述形式.如果Hamilton函数满足形式:
    \begin{equation}
        H(\bm{x},\bm{p}) = \frac{1}{2}\bm{p}^\mathrm{t} \bm{M}^{-1} \bm{p} + V(\bm{x})
    \end{equation}
    那么这个系统的Liuville方程就可以写为:
    \begin{equation} 
        - \frac {\partial \rho}{\partial t} = \bigg(\frac {\partial \rho}{\partial \bm{x}}\bigg)^\mathrm{t} \bm{M}^{-1} \bm{p}
         - \bigg(\frac {\partial \rho}{\partial \bm{p}}\bigg)^\mathrm{t} \frac {\partial V}{\partial \bm{x}}
    \end{equation}
    \par
    这里给出一种常见的分布——\textbf{Boltzmann分布}\footnote{这是正则系综(NVT系综)的系综密度函数}:
    \begin{equation}
        \rho(\bm{x},\bm{p}) \propto \mathrm{e}^{-\beta H(\bm{x},\bm{p})}
    \end{equation}

    如果一个分布满足:
    \begin{equation}
        \frac {\partial \rho}{\partial t} = 0
    \end{equation}
    那么可以看出在相空间中任何一点$(\bm{x}_1, \bm{p}_1)$的系综密度$\rho(\bm{x}_1, \bm{p}_1, t)$
    是一个与时间无关的常量(此时系综密度函数不显含时间),在这种情况下,对于任意微观量$a(\bm{x}, \bm{p})$
    其系综平均值为:
    \begin{equation}
        A(t) = \int\rho(\bm{x}, \bm{p})a(\bm{x}, \bm{p})\dd \bm{x}\dd \bm{p}
    \end{equation}
    $A$是一个不随时间变化的微观量,那么称这样的分布为\textbf{稳态分布}.

    \section{求解Liouville方程}
    一般我们遇到的是给定初始密度分布,求解$t$时刻密度分布的问题,即如下一阶偏微分方程的初值问题
    \footnote{
        这里有一个有趣的问题,给定初始密度分布$f(x, p)$是满足归一化条件的,那么能否证明按照Liouville方程
        演化的密度分布在任意时刻满足归一化条件呢?是可以的,假设0时刻密度函数分布在区域$D$内,考虑密度函数
        在相空间上的积分对时间的导数:
        在$t$时刻时:
        \begin{equation}
            \begin{split}
                \frac{\mathrm{d}}{\mathrm{d}t}\int_{\phi^{t}(D)}\rho(x_{t}, p_{t},t)\mathrm{d}x_{t}\mathrm{d}p_{t} &= \frac{\mathrm{d}}{\mathrm{d}t}\int_{D}\rho[x_{t}(x_{0},p_{0},t), p_{t}(x_{0},p_{0},t),t]\frac{\partial(x_{t},p_{t})}{\partial(x_{0},p_{0})}\mathrm{d}x_{0}\mathrm{d}p_{0}\\
                &= \int_{D}\left.\frac{\partial \rho[x_{t}(x_{0},p_{0},t), p_{t}(x_{0},p_{0},t),t]}{\partial t}\right|_{x_{0},p_{0}}\mathrm{d}x_{0}\mathrm{d}p_{0}\\
                &= \int_{D}\left[\left.\frac{\partial \rho}{\partial t}\right|_{x_{t},p_{t}} + \left\{\rho, H\right\}_{x_{t},p_{t}}\right]\mathrm{d}x_{0}\mathrm{d}p_{0}\\
                &=0 
            \end{split}
            \label{conservation of density}
        \end{equation}
        其中同时使用了Liouville定理和Liouville方程.  
    }
    (为了方便书写假设为一维系统):
    \begin{equation}
        \left\{
        \begin{split}
            &-\pdv{\rho}{t} = \pdv{\rho}{x}\pdv{H}{p} - \pdv{\rho}{p}\pdv{H}{x}\\
            &\rho(x, p, 0) = f(x, p)
        \end{split}
        \right.
        \label{Liouville equation Cauchy problem}
    \end{equation}
    在求解这个问题时很难使用解析的方法,一般都是利用数值解法求解.针对Liouville方程的特点,我们
    从两个视角出发,给出两种基本的解法.
    从方程\ref{Liouville equation Euler}出发,可以研究给定区域(给定点附近)内系综密度随时间的变化,
    这称为\textbf{Euler图象}
    \footnote{这个名称来源于流体力学,Euler通过描述空间内的\textbf{流速场}来描述流体的运动;Lagrange通过描述
    每一个质点的\textbf{运动轨迹}来描述流体的运动.}
    .从方程\ref{Liouville equation Lagrange}出发,可以追踪相空间内按照Hamilton方程运动的点处的系综密度
    ,轨线上任意一点的系综密度值都可以通过初值确定,这称为\textbf{Lagrange图象}.

    \subsection{从Euler图像求解任意时刻的密度分布}
    这里给出一个具体的问题.
    回顾第一章中使用经典力学描述HCl分子的振动,在那里只讨论了谐振子的情形,通常使用Morse势来
    更精确地描述这个振动,Morse势被定义为:
    \begin{equation}
        V(x) = D_e (1- \mathrm{e}^{-a(r-r_\mathrm{eq})})^2 = D_e(1-\mathrm{e}^{-ax})^2
    \end{equation}
    其中$a>0$, 在平衡位置附近可以使用谐振子近似(Taylor展开到二阶).写出谐振子近似下的Boltzmann分布:
    \begin{equation}
        \rho(x,p,0) = \frac {1}{Z} \mathrm{e}^{-\beta (\frac {p^2}{2m} + \frac 12 m\omega^2 x^2)} 
    \end{equation}
    其中谐振子的参数$\omega$后面给出.
    上面给出的系综密度函数不满足归一化条件,只有归一化之后才会是严格意义上的系综密度函数,
    显然归一化常数$Z$为:
    \begin{equation}
        Z = \int\ee^{-\beta H(x, p)}\dd x\dd p
    \end{equation}
    这个归一化常数被称为\textbf{正则配分函数}.
    谐振子近似下的配分函数是
    \begin{equation}
        Z = \int \mathrm{e}^{-\beta (\frac {p^2}{2m} + \frac 12 m\omega^2 x^2)} \mathrm{d}x\mathrm{d}p = \frac {2\pi}{\beta \omega}
    \end{equation}
    从量纲上分析,在配分函数中多出了$\mathrm{d}x\mathrm{d}p$的量纲(而且这么定义的配分函数的量纲
    会随着系统的维数变化).应该除以$2\pi\hslash$, 相当于对相空间做了量子化.于是
    \begin{equation}
        Z = \frac 1{\beta \hslash \omega}
    \end{equation}
    就是无量纲的配分函数.
    \par 
    回到用Morse势描述HCl的振动的问题,Morse势的常数$a$可以用谐振子近似的$\omega$进行估计.
    令$x \to 0 $,对$V(x)$在平衡位置附近作Taylor展开,展开到二阶:
    \begin{equation}
        V(x) = D_e a^2 x^2 + o(x^2)
    \end{equation}
    势能在平衡位置Taylor展开的系数决定了谐振子近似中谐振子的参数:
    \begin{equation}
        \frac 12 m\omega^2x^2 = D_e a^2 x^2
    \end{equation}
    于是:
    \begin{equation}
        \omega = \sqrt{\frac {2D_ea^2}m}
        \label{omega of Morse}
    \end{equation}
    这样就可以在知道振动频率(通过光谱数据)后构造双原子分子的Morse势.
    \par 
    现在我们考虑氢分子在Morse势下的振动,计算在Morse势下给定初始系综密度随时间的演化.
    Hamilton量为:
    \begin{equation}
        H(x, p) = \frac{p^2}{2\mu} + D_e(1-\ee^{-ax^2})^2
    \end{equation}
    初始的系综密度给定:
    \begin{equation}
        \rho(x, p, 0) = \frac{1}{Z}\exp\left[-\beta\left(\frac{p^2}{2m} + \frac{1}{2}m\omega^2x^2\right)\right]
    \end{equation}
    这样就可以考虑初值问题\ref{Liouville equation Cauchy problem}.
    与数值求解常微分方程的思路类似,我们希望用差分来代替方程中的偏微分.这样就需要把空间
    \footnote{
        很显然,只能划分有限空间上的网格,但是给出的密度函数并不是有限空间的函数,因此要做出取舍,
        要看系综密度主要分布在什么位置,将会演化到什么位置.
    }
    划分成规则的(按照坐标线划分的)矩形网格,通过函数在网格点上的值来不断递推下一个时刻(时间也是离散的)
    网格点上的函数值,对于边界值需要额外讨论
    \footnote{
        一般而言,有界区域上的偏微分方程都会给定边界条件,但是在处理我们的问题时人为选择了有界的区域,
        边界条件需要自己给定,要怎么选择?
    }
    .
    在$x$取值范围等距插入$M-1$个点,将其最小值与最大值分别记为$x_0, x_M$,令$\Delta x = x_{j+1} - x_j$
    \footnote{
        这里有一个问题,如何选择合适的空间步长$\Delta x$?
    }
    ;在$p$取值的范围等距插入
    $N-1$个点,将其最小值与最大值分别记为$p_0, p_N$,令$\Delta p = p_{j + 1} - p_j$.
    这样就构造了\textbf{某个时刻}相空间中的格点
    ;将初始时刻记作$t_0$,时间步长设为$\Delta t$,
    如此也构造了离散的时间点$t_n$.解偏微分方程相当于通过前一个时刻空间中函数值推出下一个时刻
    空间中函数值的过程,而这里通过将空间格点化的方法数值求解的过程相当于通过$t_j$时刻对应相空间格点上
    的函数值递推$t_{j+1}$时刻相空间格点上函数值的过程.下面给出基本的递推方案,为了方便起见,首先引入一些
    记号(对于格点的编号):
    \begin{equation}
        \rho_{i, j}^{n} := \rho(x_i, p_j, t_n)
    \end{equation}
    那么$\rho$对时间的偏导数可以表示为:
    \begin{equation}
        \pdv{\rho(x_i, p_j, t_n)}{t} \approx  \frac{\rho_{i, j}^{n + 1} - \rho_{i, j}^{n}}{\Delta t}
    \end{equation}
    其中$\rho_{i, j}^{n+1}$是待求的量.同样也可以通过差分表示$\rho$对$x, p$的偏微分:
    \begin{equation}
        \begin{split}
            \pdv{\rho(x_i, p_j, t_n)}{x} &\approx \frac{\rho_{i + 1, j}^{n} - \rho_{i, j}^n}{\Delta x}
             \approx \frac{\rho_{i + 1, j}^{n} - \rho_{i-1, j}^n}{2\Delta x}\\
             \pdv{\rho(x_i, p_j, t_n)}{p} &\approx \frac{\rho_{i, j+1}^{n} - \rho_{i, j}^n}{\Delta p}
             \approx \frac{\rho_{i, j+1}^{n} - \rho_{i, j+1}^n}{2\Delta p}
        \end{split}
    \end{equation}
    注意到上面使用了\textbf{中心差分}作为一种数值微分方法,中心差分更加对称,一般而言比不对称差分
    (对事件求导的差分)拥有更高的精度.使用这样的差分方案将Liouville方程改为差分方程:
    \begin{equation}
        -\frac{\rho_{i, j}^{n+1} - \rho_{i, j}^{n}}{\Delta t} = \frac{p_j}{\mu}
        \frac{\rho_{i+1, j}^{n} - \rho_{i-1, j}^{n}}{\Delta x} - V_x(x_i)
        \frac{\rho_{i, j+1}^{n} - \rho_{i, j-1}^{n}}{\Delta p}
    \end{equation}
    这个方案被称为显式的差分方案,因为$t_{n+1}$时刻格点上的函数值可以直接通过上面的式子显式地表达:
    \begin{equation}
        \rho_{i, j}^{n+1} = \rho_{i, j}^{n} + \frac{\Delta t}{2\Delta p}V_{x}(x_i)(\rho_{i, j+1}^{n} - \rho_{i, j-1}^{n})
        - \frac{p_j\Delta t}{2 \mu \Delta x}(\rho_{i+1, j}^{n} - \rho_{i-1, j}^{n}) 
    \end{equation}
    可以看出根据前一个时刻5个格点的数据可以递推出下一个时刻一个格点的函数值.这样看起来已经很好地解决了
    数值演化系综密度的问题,理论上只要将时间步长和空间步长取得足够小,就可以获得任意精确的解,但是实际
    并不是这样,真实计算中无法将步长取的任意小(更小的时间步长会带来更大的计算量,同时也受到
    计算机处理浮点数精度的影响)所以我们获取的数值解并不一定准确
    \footnote{
        至于“有限差分带来的误差到底有多大”,“这样的误差怎么传递”,“误差会不会累积”这样的问题就不是本课程
        能够讨论的内容了,应该查阅数值偏微分方程相关的书籍.
    }
    .
    \par 
    经过实践\footnote{是修改笔记者的实践,情况仅供参考},这种显式的差分方案数值上不稳定(长时间演化会带来
    函数值的发散,尤其集中在选定区域的边界).一般而言,隐式的差分方案在数值上相较于显式方案更加稳定,
    在隐式方案中,对于空间的微分要通过下一个时刻格点上的函数值计算:
    \begin{equation}
        \pdv{\rho(x_i, p_j, t_n)}{x} \approx \frac{\rho_{i + 1, j}^{n+1} - \rho_{i, j}^{n+1}}{\Delta x}
             \approx \frac{\rho_{i + 1, j}^{n+1} - \rho_{i-1, j}^{n+1}}{2\Delta x}
    \end{equation}
    下面给出一个笔记修改者使用过的隐式差分方案:
    \begin{equation}
        \begin{split}
            \frac{\rho_{i,j}^{n+0.5} - \rho_{i,j}^{n}}{\Delta t/2} &= V_{x}(x_{i})\left[\frac{\rho_{i,j+1}^{n+0.5} - 
            \rho_{i,j-1}^{n+0.5}}{2\Delta p}\right] - \frac{p_{j}}{\mu}\left[\frac{\rho_{i+1,j}^{n} - 
            \rho_{i-1,j}^{n}}{2\Delta x}\right]\\
            \frac{\rho_{i,j}^{n+1} - \rho_{i,j}^{n+0.5}}{\Delta t/2} &= V_{x}(x_{i})\left[\frac{\rho_{i,j+1}^{n+0.5} -
            \rho_{i,j-1}^{n+0.5}}{2\Delta p}\right] - \frac{p_{j}}{\mu}\left[\frac{\rho_{i+1,j}^{n+1} - 
            \rho_{i-1,j}^{n+1}}{2\Delta x}\right]
        \end{split}
    \end{equation}
    这是一种半隐式的方案,一个时间步长分为两小步演化,每一个方程中有一个变量的微分通过隐式计算、另一个通过显式计算
    \footnote{
        这样的目的是为了让待求解的线性方程组更加简单(是一个三对角矩阵方程),从原理上讲可以采用全隐式
        差分方案,但是得到的线性方程组是更高阶的,会使得整体计算的复杂度更高.
    }
    .
    引入记号$r:=\frac{\Delta t}{\Delta p},\, s:=\frac{\Delta t}{\Delta x}$,上面的两个差分方程可以写为如下形式:
    \begin{equation}
        \begin{split}
            &\frac{V_{x}(x_{i})}{4}r\cdot\rho_{i,j-1}^{n+0.5} + \rho_{i,j}^{n+0.5} - \frac{V_{x}(x_{i})}{4}r\cdot\rho_{i,j+1}^{n+0.5} =
             \frac{p_{j}}{4\mu}s\cdot\rho_{i-1,j}^{n} + \rho_{i,j}^{n} - \frac{p_{j}}{4\mu}s\cdot\rho_{i+1,j}^{n} \\
            &- \frac{p_{j}}{4\mu}s\cdot\rho_{i-1,j}^{n+1} + \rho_{i,j}^{n+1} + \frac{p_{j}}{4\mu}s\cdot\rho_{i+1,j}^{n+1} =
            -\frac{V_{x}(x_{i})}{4}r\cdot\rho_{i,j-1}^{n+0.5} + \rho_{i,j}^{n+0.5} + \frac{V_{x}(x_{i})}{4}r\cdot\rho_{i,j+1}^{n+0.5}
        \end{split}
    \end{equation}
    可以看出上面两个方程都是三对角的矩阵方程
    \footnote{
        三对角矩阵方程指的是线性方程组的系数矩阵除了主对角线和两个次对角线以外,
        其余元素均为零的线性方程组,求解这样的方程组有特殊的算法(追赶法),为$O(n^2)$复杂度
        ,显著快于求解一般线性方程组的Gauss消元法($O(n^3)$复杂度)
    }
    ,可以每次求解n个这样的三对角矩阵方程来得到下一个时刻格点上的函数值.经过实践,这个隐式的差分方案
    在较大的空间步长上都是数值稳定的,但是计算量较大.
    \par 
    这里还留有一个问题,根据前文的讨论\ref{conservation of density},在密度函数随时间演化的过程中
    其始终是归一化的,但是我们这里讨论的数值解法是否可以保证这一点呢?能不能发展出保持这个守恒量
    的差分格式呢?

    \subsection{从Lagrange图像求解任意时刻密度分布}
    前一节中使用格点化相空间的方法数值求解了Liouville方程,这种方法在相空间维数很高的时候计算量急剧上升
    (主要是因为格点的数量随着维度指数上升),
    导致这种方法无法适用于高维系统的计算.除了用Euler图象来演化密度函数以外,也可以用Lagrange图象来演化密度函数.
    考虑Liouville方程的另一种形式:
    \begin{equation}
        \frac {\mathrm{d}}{\mathrm{d}t} \rho(x_t(x_0, p_0, t),p_t(x_0, p_0, t),t) = 0
    \end{equation}
    可以\textbf{形式上}得到$t$时刻的概率密度为:
    \begin{equation}
        \rho(x,p,t) = \int \rho(x_0,p_0,0)\delta(x-x_t(x_0,p_0)) \delta(p-p_t(x_0,p_0)) \mathrm{d}x_0\mathrm{d}p_0
        \label{formal solution}
    \end{equation}
    这里使用了$\delta$函数
    \footnote{
        严格来讲$\delta$函数并不是普通意义上的函数,而是广义函数,有关$\delta$函数的严格理论
        这里不做讨论,只是从物理含义(直观)上给出一个概念
    }
    .$\delta$函数满足:
    \begin{equation}
        \begin{split}
        &\delta(x-x_0) = 0, \ \forall \ x \neq x_0\\
        &\int_{-\infty}^{+\infty} \delta(x-x_0) \mathrm{d}x = 1\\
        &\int_{-\infty}^{+\infty} f(x)\delta(x-x_0) \mathrm{d}x = f(x_0)
        \end{split}
        \label{delta function}
    \end{equation}
    \footnote{$\delta$函数的严格定义是从上面性质的第三条出发的,$\delta$函数被
    定义为试验函数空间上的满足性质3的连续线性泛函}基于上面的性质,$\delta$函数
    在物理中经常被用来描述一些集中在某一点但积分有限的量(比如带有一定电量的点电荷,具有质量的质点).
    现在希望给$\delta$函数给一个形式
    \footnote{
        可以证明$\delta$函数并不能表示为普通函数的形式,但是可以用普通函数序列来逼近.
    }
    ,让它和上面满足的性质自洽:
    可以利用Fourier变换
    \footnote{注意这里Fourier变换的定义,为了对称通常将系数写为$\frac{1}{\sqrt{2\pi}}$}
    及其逆变换的性质给出一个积分形式的$\delta$函数(形式上的):
    \begin{equation}
        \begin{split}
            F(k) &:= \frac 1{\sqrt{2\pi}} \int_{-\infty}^{+\infty} f(x)\mathrm{e}^{-\mathrm{i}kx}\mathrm{d}x\\
            f(x) &= \frac 1{\sqrt{2\pi}} \int_{-\infty}^{+\infty} F(k)\mathrm{e}^{\mathrm{i}kx}\mathrm{d}k
        \end{split}
        \label{fourier transform}
    \end{equation}
    于是有:
    \begin{equation}
        \begin{split}
            f(x_0) &= \frac 1{\sqrt{2\pi}} \int_{-\infty}^{+\infty}\left[\frac 1{\sqrt{2\pi}} 
            \int_{-\infty}^{+\infty} f(x)\mathrm{e}^{-\mathrm{i}kx}\mathrm{d}x\right] \mathrm{e}^{\mathrm{i}kx_0}\mathrm{d}k\\
            &= \frac 1{2\pi} \int_{-\infty}^{+\infty}f(x)\left[\int_{-\infty}^{+\infty}\ee^{-\ii k(x - x_0)}\dd k\right]\dd x
        \end{split}
    \end{equation}
    在上式中第二步交换了两个广义积分的顺序,其中的\textbf{收敛性}值得仔细考虑,在这里交换后积分并不收敛,
    但是可以形式上定义$\delta$函数为(为了方便使用):
    \begin{equation}
        \delta(x-x_0) = \frac 1{2\pi} \int_{-\infty}^{+\infty} \mathrm{e}^{\mathrm{i}k(x-x_0)}\mathrm{d}k
        \label{integral formation of delta function}
    \end{equation}
    讨论完了$\delta$函数这个数学工具,我们回到使用Lagrange图像解决密度函数演化的问题.直接利用:
    \begin{equation}
        \rho(x_t(x_0, p_0), p_t(x_0, p_0), t) = \rho(x_0, p_0, 0)
    \end{equation}
    给定相空间中的初始点$(x_0, p_0)$,只要数值求解Hamilton方程,就可以得到
    $(x_t(x_0, p_0), p_t(x_0, p_0))$处的系综密度值,那么可以抽取初始时刻相空间中
    大量的点(比如一个矩形的网格),同时按照Hamilton方程演化这些点,就可以得到这些点$t$
    时刻时在相空间中的分布,进而得出这些点的系综密度值.这种方法只用求解常微分方程组(前面章节中
    讨论过数值解法),计算量比较小而且适用于维数较高的情况.除此之外还有一个明显的优点,我们可以
    在初始时刻时就去关注那些有着明显密度分布的点.
    \par 
    上述的思想可以进一步推广,去解决更普遍的一些一阶偏微分方程,考虑如下方程:
    \begin{equation}
        \pdv{f(\bm{x}, t)}{t} + \bm{g}(\bm{x}, t) \cdot \pdv{f(\bm{x}, t)}{\bm{x}} = 0
    \end{equation}
    其中$\bm{g}(\bm{x}, t)$是一个给定的向量值函数,可以看作$\mathbb{R}^n$上的速度场,
    为了形象起见,考虑一个在$\mathbb{R}^n$中沿着速度场运动的粒子,那么它满足:
    \begin{equation}
        \dv{\bm{x}}{t} = \bm{g}(\bm{x}, t)
        \label{character line}
    \end{equation}
    容易看出沿着这个粒子的运动轨迹,函数$f$的值不变,这样可以仿照Lagrange图像求解密度函数的方法来求解
    任意时刻$f$的数值.将这个粒子的运动轨迹\ref{character line}称为这个一阶偏微分方程的特征线,
    Hamilton方程决定的曲线就是Liouville方程的特征线,因此这个方法也被称为特征线方法.
    \par
    物理量$B(x, p)$的期望定义为:
    \begin{equation}
        \langle B(t) \rangle = \int \rho(x,p,t) B(x,p) \mathrm{d}x\mathrm{d}p
    \end{equation}
    回到用Morse势描述HCl的振动的问题,初始时刻为谐振子的Boltzmann分布时,计算位置、位置平方的期望和位置涨落:
    \begin{equation}
        \begin{split}
            \langle x \rangle &= 0\\
            \langle x^2 \rangle &= \frac 1{\beta m \omega^2}\\
            \Delta x &= \sqrt{\langle x^2 \rangle - \langle x \rangle ^2} = \frac 1{\sqrt{\beta m \omega^2}}
        \end{split}
    \end{equation}
    但是上面这些量会随着时间演化(原因是系综密度会随时间变化),上文中已经给出了任意时刻系综密度的计算方法
    ,理论上可以计算任意时刻的上述物理量,但是实际上会有相当的麻烦
    \footnote{
        根据笔记修改者的经验,如果要计算系综平均值必须依赖于任意时刻格点上的系综密度,将这些数据存储
        起来可能是一笔不小的开销(很有可能会导致内存溢出);另外,
        通过Lagrange图像求解系综密度一般来说不能得到矩形格点上的系综密度值,会造成数值积分的困难.
    }
    .我们可以有更好的方法来计算这些物理量,利用Liouville方程的形式解\ref{formal solution},
    可以将$t$时刻物理量的期望表示为
    \footnote{
        如果觉得使用包含$\delta$函数的形式解计算数学上不够“严格”,也可以利用0时刻与$t$时刻之间粒子位置的映射(严格讲
        是Hamilton相流)$\phi^t$进行积分换元,然后使用Liouville定理和Liouville方程,得到完全相同的结果.
    }
    :
    \begin{equation}
        \begin{split}
            \langle B(t) \rangle &= \int \rho(\bm{x},\bm{p},t) B(\bm{x},\bm{p}) \mathrm{d}\bm{x}\mathrm{d}\bm{p}\\
            &= \iint \rho(\bm{x}_0,\bm{p}_0,0)\delta(\bm{x}-\bm{x}_t(\bm{x}_0,\bm{p}_0)) \delta(\bm{p}-\bm{p}_t(\bm{x}_0,\bm{p}_0)) \mathrm{d}\bm{x}_0\mathrm{d}\bm{p}_0 B(\bm{x},\bm{p}) \mathrm{d}\bm{x}\mathrm{d}\bm{p}\\
            &= \iint \delta(\bm{x}-\bm{x}_t(\bm{x}_0,\bm{p}_0)) \delta(\bm{p}-\bm{p}_t(\bm{x}_0,\bm{p}_0)) B(\bm{x},\bm{p}) \mathrm{d}\bm{x}\mathrm{d}\bm{p} \rho(\bm{x}_0,\bm{p}_0,0) \mathrm{d}\bm{x}_0\mathrm{d}\bm{p}_0\\
            &= \int B(\bm{x}_t,\bm{p}_t) \rho(\bm{x}_0,\bm{p}_0,0)\mathrm{d}\bm{x}_0\mathrm{d}\bm{p}_0
        \end{split}
    \end{equation}
    这意味着,只用初始概率密度也可以得到$t$时刻的物理量的期望,这使得数值计算变得十分方便.

    在经典的情形下,由于不考虑不确定性关系,因此这里可以任意交换两个delta函数的次序。
    后面我们将看到,在量子相空间中,存在算符的delta函数,因此不能随意交换两个delta函数的次序,否则会导致相空间定义的改变。

    \section{应用:多原子分子的振动配分函数}

    考虑NVT系综中的水分子。利用简正坐标变换,单个水分子的系综密度函数可以表示为:
    \begin{equation}
        \begin{split}
            \rho(\bm{x}, \bm{p}) &= \frac{1}{Z}\exp\left[-\beta H(\bm{x}, \bm{p})\right]
            = \frac{1}{Z}\exp\left[-\beta\left(\frac{1}{2}\bm{p}^{\mr{t}}\mb{M}^{-1}\bm{p} + V(\bm{x})\right)\right]\\
            Z &= \frac{1}{(2\pi\hslash)^{3N}}\int\exp\left[-\beta H(\bm{x}, \bm{p})\right]\dd \bm{x}\dd \bm{p}
        \end{split}
    \end{equation}
    如果可以知道$V(\bm{x})$的具体形式
    \footnote{可以从文献中查找到描述水分子不同性质、不同精度的势能面}
    , 就可以计算出配分函数$Z$
    \footnote{即使在$V(\bm{x})$形式已知的情况下, 计算配分函数也并不是简单的事情.
    之前提到了在Morse势下计算\ce{HCl}分子物理量的系综平均值, (只考虑径向)
    这个二体系统的相空间只有2维, 可以采用划分网格等传统数值积分方法计算.
    \textbf{但是}, 对于水分子, 相空间有9维, 如果采用网格法计算这样的积分, 那么
    格点数量会随系统的维数指数增加.为了计算这类高维积分, 人们发展了\textbf{分子动力学}
    与\textbf{Monte Carlo}等方法, 不再均匀地从相空间中取点, 而是在
    $\exp\left[-\beta H(\bm{x}, \bm{p})\right]$取值比较大的地方采更多的
    点.}
    , 进一步获得水分子的热力学性质.这里为了计算方便, 考虑对势能面使用小振动近似
    \footnote{
        容易想象, 当系统温度较低时, 系综密度只分布在平衡点附近, 此时小振动近似是比较准确
        的.(当然, 这里是经典谐振子, 如果使用量子力学框架, 需要另行讨论)
    }
    , 那么Cartesian坐标下的系综密度应该写为:
    \begin{equation}
        \begin{split}
            \rho(\bm{x}, \bm{p}) = \frac{1}{Z}\exp\left[-\beta\left(
                \frac{1}{2}\bm{p}^{\mr{t}}\mb{M}^{-1}\bm{p} + \frac{1}{2}\bm{x}^{\mr{t}}
                \mb{V}^{(2)}\bm{x}
            \right)\right]
        \end{split}
    \end{equation}
    在简正坐标下小振动近似的Hamilton量有着简单的形式, 因此我们非常希望能用简正坐标表示
    系综密度函数(这相当于在讨论坐标变换之下系综密度的变换规则).假设坐标变换为:
    $\sigma: (\bm{x}, \bm{p})\mapsto(\bm{Q}, \bm{P})$, 
    系综密度函数为$\tilde{\rho}(\bm{Q}, \bm{P})$, 考虑相空间内的一块体积$D$, 无论
    使用哪个坐标表示, 这块体积包含的系综密度是相同的, 有:
    \begin{equation}
        \begin{split}
            \int_{D(\bm{x},\bm{p})}\rho(\bm{x}, \bm{p})\dd \bm{x}\dd \bm{p}
             = \int_{D(\bm{Q}, \bm{P})}\tilde{\rho}(\bm{Q}, \bm{P})\dd \bm{Q}\dd \bm{P}
        \end{split}
    \end{equation}
    其中$D(\bm{x},\bm{p})$为体积$D$在原坐标下的表示, 而$D(\bm{Q}, \bm{P})$为体积$D$在
    新坐标下的表示, 对于任意的体积$D$上式成立, 那么利用重积分的换元公式, 可以得到:
    \begin{equation}
        \tilde{\rho}(\bm{Q}, \bm{P}) = \rho(\bm{x}, \bm{p})\left|\pdv{(\bm{x}, \bm{p})}{(\bm{Q}, \bm{P})}\right|
    \end{equation}
    (上式中的$(\bm{x}, \bm{p})$要用$(\bm{Q}, \bm{P})$表示).考虑简正坐标变换:
    \begin{equation}
        \begin{aligned}
            \left|\frac {\partial (\bm{x,p})}{\partial (\bm{Q,,P})}\right| &= \det
            \begin{pmatrix}
                \mb{M}^{-\frac{1}{2}}\mb{S} & 0\\
                0 & \mb{M}^{\frac{1}{2}}\mb{S}
            \end{pmatrix}
            = 1
        \end{aligned}
    \end{equation}
    这里用到了正交矩阵的行列式为1.这样得到了简正坐标下的系综密度:
    \begin{equation}
        \tilde{\rho}(\bm{Q}, \bm{P}) = \rho(\bm{x}(\bm{Q}, \bm{P}), \bm{p}(\bm{Q}, \bm{P})) 
        = \frac{1}{Z}\exp\left[-\beta\sum_{j=1}^{N}\left(\frac{1}{2}P_{j}^{2} + \frac{1}{2}\omega_{j}^2 Q_{j}^{2}\right)\right] 
    \end{equation}
    可以看到使用小振动近似后, 系统的系综密度函数在简正坐标下具有Gauss函数乘积的形式, 具体
    来说, 可以定义每个简正坐标自由度的“系综密度”:
    \begin{equation}
        \tilde{\rho}_{j}(Q_j, P_j) = \frac {1}{\beta\hslash\omega_j} \exp\left[-\frac {\beta}2 (P_j^2 + \omega_j^2 Q_j^2)\right]
    \end{equation}
    系统的系综密度为各个自由度“系综密度”的乘积:
    \begin{equation}
        \tilde{\rho}(\bm{Q}, \bm{P}) = \prod_{j=1}^{N}\tilde{\rho}_{j}(Q_j, P_j)
    \end{equation}
    系综密度可以表示为Gauss函数乘积的这个特点可以极大地方便数值计算, 考虑系统的物理量$B$
    , 将其写为简正坐标的函数$B(\bm{Q}, \bm{P})$, 那么平均值为:
    \begin{equation}
        \begin{split}
            \left<B\right> = \int B(\bm{Q}, \bm{P})\prod_{j=1}^{N}\tilde{\rho}_j\dd \bm{Q}\dd \bm{P}
        \end{split}
    \end{equation}
    分别在每个自由度上按照对应的密度函数采样
    \footnote{
        如果想了解怎么按照Gauss函数对应的密度分布抽样, 可以参考Box-Muller sampling
    }
    , 将这些坐标组合为相空间中点的坐标, 这样得到
    对于相空间中密度函数的抽样, 根据这个抽样可以计算$\left<B\right>$, 这个方法被称为
    “重要性抽样”.


    \begin{asg}
        尝试用不同的方法证明Liouville定理.
    \end{asg}
    \begin{asg}
        Boltzmann分布是否为稳态分布?
    \end{asg}
    \begin{asg}
        查阅\ce{H2}分子的红外光谱数据,构造\ce{H2}分子的Morse势表达式.
    \end{asg}
    \begin{asg}
        以谐振子的Boltzmann分布为初始分布,在Morse势,Euler图象下演化\ce{H2}的$t$时刻的分布.
    \end{asg}
    \begin{asg}
        以谐振子的Boltzmann分布为初始分布,在Morse势,Lagrange图象下演化\ce{H2}的$t$时刻的分布.
    \end{asg}

    \bibliographystyle{plain}
    \bibliography{ref_phasespace}
    \chapter{量子力学的相空间形式}
    \section{统计力学基础回顾}

        在量子统计力学中对于任意一个算符 $\hat{A}$
        \begin{equation}
            \langle {\hat{A}} \rangle = \sum_i \left\langle {\psi^{i}(t)} \middle| {\hat{A}} \middle| {\psi^{i}(t)} \right\rangle P_{i}
        \end{equation}
        其中$P_i$是体系处于$\psi^{i}(t)$的态上的概率, 且$\sum_i P_i = 1$. 
        这里实际采取了两重平均: 其一是量子力学上的平均, 即物理量在某个态下的期望
        $\left\langle {\psi^{i}(t)} \middle| {\hat{A}} \middle| {\psi^{i}(t)} \right\rangle$; 
        另个一重是热力学上的平均, 表现为某个态出现的概率 $P_{i}$. 

        设$\left. \middle| {\phi_{i}} \right\rangle$ 是一组完备基, 则
        \begin{equation}\begin{aligned}
            \langle {\hat{A}} \rangle
            &= \sum_i \left\langle {\psi^{i}(t)} \middle| {\hat{A}} \middle| {\psi^{i}(t)} \right\rangle P_{i}
            \\
            &= \sum_{imn} \left\langle {\psi^{i}(t)} \middle| {\phi_{n}} \right\rangle \left\langle {\phi_{n}} \middle| {\hat{A}} \middle| {\phi_{m}} \right\rangle \left\langle {\phi_{m}} \middle| {\psi^{i}(t)} \right\rangle P_{i}
            \\
            &= \sum_{imn} \left\langle {\phi_{n}} \middle| {\hat{A}} \middle| {\phi_{m}} \right\rangle 
            \left\langle {\phi_{m}} \middle| {\psi^{i}(t)} \right\rangle P_{i}
            \left\langle {\psi^{i}(t)} \middle| {\phi_{n}} \right\rangle
        \end{aligned}\end{equation}
        从中我们可以定义\textbf{统计算符}
        \begin{equation}
            \hat{\rho} = \sum_i P_{i} \left. \middle| {\psi^{i}(t)} \right\rangle \left\langle {\psi^{i}(t)} \middle| \right.
        \end{equation}
        则有
        \begin{equation}
            \langle {\hat{A}} \rangle = \mathrm{Tr}(\hat{\rho}\hat{A})
        \end{equation}

        如果在某种表象下, 存在某一个态$\psi^{n}(t)$对任意的$i$都有$P_i = \delta_{ni}$, 则称这个系统处于一个\textbf{纯态}; 否则称这个体系处于\textbf{混合态}.
        纯态的密度算符可以写为$\hat{\rho} = \left. \middle| {\psi^{n}(t)} \right\rangle \left\langle {\psi^{n}(t)} \middle| \right.$. 

        统计算符有如下性质
        \begin{equation}
            \mathrm{Tr}\hat\rho = 1, \quad
            \hat\rho^{\dagger} = \hat\rho, \quad
            \mathrm{Tr}\hat\rho^{2} \leq 1
        \end{equation}
        当且仅当密度矩阵对应的是纯态时, 最后一个式子取等号. 这一性质使用Cauchy不等式可以证明. 

        \splitline

        根据我们熟悉的经典统计物理, 物理量的系综平均为
        \begin{equation}
            \langle A \rangle = \int \rho(x,p) A(x,p) \mathrm{d}x\mathrm{d}p
        \end{equation}
        其中
        \begin{equation}\begin{aligned}
            \rho &= \frac 1Z \mathrm{e}^{-\beta H(x,p)}\\
            Z &= \int \frac 1{2\pi \hbar} \mathrm{e}^{-\beta H(x,p)} \mathrm{d}x \mathrm{d}p
        \end{aligned}\end{equation}

    \section{寻找量子相空间的定义}

        能否将量子统计力学中求物理量期望值的公式写成与经典统计物理相同的形式? 
        这个问题粗看似乎不太可能: 要把算符写成$x$与$p$的函数似乎就意味着在指定$x$与$p$下得到算符的"值", 
        而在量子力学框架下我们无法同时确定粒子的位置与动量. 
        
        不过,让我们从量子配分函数的定义出发:
        \[
            Z = \Tr{\e^{-\beta \hat{H}}}
        \]
        对照经典的配分函数:
        \[
            Z = \int \frac {1}{2\pi \hbar} \mathrm{e}^{-\beta H(x,p)} \mathrm{d}x \mathrm{d}p
        \]

        在经典情形下,系综密度函数可以向上述经典配分函数中乘上一个\(\delta\)函数得到。
        比如说:
        \[
            \begin{gathered}
                \rho_X(x') = \int \frac 1{2\pi \hbar} \mathrm{e}^{-\beta H(x,p)} \delta(x - x') \mathrm{d}x \mathrm{d}p
                \\
                \rho_P(p') = \int \frac 1{2\pi \hbar} \mathrm{e}^{-\beta H(x,p)} \delta(p - p') \mathrm{d}x \mathrm{d}p
            \end{gathered}
        \]
        同时,也有
        \[
            \rho(x', p') = \int \frac 1{2\pi \hbar} \mathrm{e}^{-\beta H(x,p)} \delta(x - x')\delta(p - p') \mathrm{d}x \mathrm{d}p
        \]
        这两个\(\delta\)函数的次序可以交换,因为此时\(x, p\)都是数或向量。

        因此,在量子情形下,我们可以对位置和动量分别干类似的事情,只不过将数或向量替换为算符即可:
        \[
            \begin{gathered}
                \rho_X(x') = \Tr{\e^{-\beta\hat{H}}\delta(\hat{x} - x')}
                \\
                \rho_P(p') = \Tr{\e^{-\beta\hat{H}}\delta(\hat{p} - p')}
            \end{gathered}
        \]
        其中,算符的\(\delta\)函数应该在该算符本征态上求迹的意义下理解:
        \[
            \begin{aligned}
                \rho_X(x') &= \Tr{\e^{-\beta\hat{H}}\delta(\hat{x} - x')} 
            \\ &= \int \dd x \braket{x | \e^{-\beta\hat{H}}\delta(\hat{x} - x') | x}
            \\ &= \braket{x' | \e^{-\beta\hat{H}} | x'}
            \end{aligned}
        \]

        但是,如果要定义联合的密度分布,则此时会遇到麻烦。两个不对易的算符,其\(\delta\)函数此时是无法交换的:
        \[
            \Tr{\e^{-\beta\hat{H}}\delta(\hat{x} - x')\delta(\hat{p} - p')} \neq \Tr{\e^{-\beta\hat{H}}\delta(\hat{p} - p')\delta(\hat{x} - x')}
        \]
        这是因为,如果我们将\(\delta\)函数以其Fourier展开的形式写出来:
        \[
            \delta(\hat{x} - x')\delta(\hat{p} - p') = \frac{1}{(2\pi)^2} \int \dd k \dd \xi \e^{\ii k(\hat{x} - x')}\e^{\ii \xi (\hat{p} - p')}
        \]
        会发现,由Glauber公式,有
        \[
            \e^{\ii k(\hat{x} - x')}\e^{\ii \xi (\hat{p} - p')} = \e^{\ii k(\hat{x} - x') + \ii \xi (\hat{p} - p')} \e^{\frac{-\ii\hbar k\xi}{2}}
        \]
        而
        \[
            \e^{\ii \xi (\hat{p} - p')}\e^{\ii k(\hat{x} - x')} = \e^{\ii k(\hat{x} - x') + \ii \xi (\hat{p} - p')} \e^{\frac{\ii\hbar k\xi}{2}}
        \]
        说明这两者是不相等的。因此,似乎对于这个量子相空间的密度函数,出现了不同的定义。
        事实上我们还可以定义密度函数如下:
        \[
            \Tr{\e^{-\beta\hat{H}}\delta(\hat{x} - x', \hat{p} - p')}
        \]
        其中用到了简记
        \[
            \delta(\hat{x} - x', \hat{p} - p') = \frac{1}{(2\pi)^2} \int \dd k \dd \xi
            \e^{\ii k(\hat{x} - x') + \ii \xi (\hat{p} - p')}
        \]

        因此,似乎密度函数的定义不是唯一的。这种不唯一性来自于位置和动量算符的不对易性。
        由于不对易关系的存在,不能找到一个唯一的由物理量到算符的映射。

        将这个问题稍加推广,量子相空间的定义问题可以拓展如下:
        考虑算符空间到物理量空间的任意映射核\(K(\hat{x} - x', \hat{p} - p')\),其是函数\(f(k, \xi)\)的类似Fourier变换的形式:
        \[
            K(\hat{x} - x', \hat{p} - p') = \frac{1}{(2\pi)^2} \int \dd k \dd \xi \e^{\ii k(\hat{x} - x') + \ii \xi (\hat{p} - p')} f(k, \xi)
        \]
        如果它能够生成合适的相空间,需要让这个映射核作用在Boltzmann算符上以后,求迹后积分得到原来的配分函数\(Z\)。
        \[
            \int \dd x' \dd p' \Tr{\e^{-\beta\hat{H}}K(\hat{x} - x', \hat{p} - p')} = \Tr{\e^{-\beta\hat{H}}} = Z
        \]
        这也就要求了
        \[
            \int \dd x' \dd p' K(\hat{x} - x', \hat{p} - p') = \hat{I}
        \]
        
        我们希望把上述对\(K\)的限制转化为对\(f(k, \xi)\)的限制。
        作为本课程的一道作业题,请读者利用\(\delta\)函数的Fourier变换表示,完成:

        \begin{asg}
            证明:文中\(f(k, \xi)\)应当满足的条件是:\(f(0, 0) = 1\)。这说明了量子相空间的定义不是唯一的。
        \end{asg}

        \splitline
        
    对Boltzmann算符讨论的结果可以扩展到任意情形。
    上文中对Boltzmann算符,我们找到了映射关系\(\rho(x, p)\leftarrow \hat{\rho}\),使得对于量子情形有
    \[
        Z = \int \dd x \dd p \rho(x, p) = \Tr{\rho}
    \]
    
    对算符\(\hat{\rho}\)的讨论可以推广到任意算符\(\hat{A}\)上。定义映射:
    \[
        A(x, p) = \Tr{\hat{A} \hat{K}(x, p)} = \Tr{
            \hat{A} \int \frac{\dd k \dd \xi}{(2\pi)^2}  \e^{\ii k(\hat{x} - x') + \ii \xi (\hat{p} - p')} f(k, \xi)
            }
    \]
    则此时下式即可成立:
    \[
        \Tr{\hat{A}} = \int \dd x \dd p A(x, p)
    \]

    然而我们不满足于此。为了使得量子相空间具有物理意义,我们必须设法对物理量的期望值的表示:
    \[
        \braket{A} = \frac{1}{Z} \Tr{\hat{\rho}\hat{A}}
    \]
    也找到相应的相空间对应。因此必须考虑两个算符的迹如何表示的问题。
    
    可以证明,对于算符\(\hat{A}, \hat{B}\)及其进行上述映射得到的函数\(A(x, p)\)和\(B(x, p)\),
    \[\Tr{\hat{A}\hat{B}} = \int \dd x \dd p A(x, p) B(x, p)\]成立当且仅当\(f(k, \xi) \equiv 1\)。
    以\(f(k, \xi) \equiv 1\)定义的量子相空间密度函数:
    \[
        \rho(x, p) = \Tr{
            \e^{-\beta\hat{H}} \int \frac{\dd k \dd \xi}{(2\pi)^2}  \e^{\ii k(\hat{x} - x') + \ii \xi (\hat{p} - p')}
            }
    \]
    称为\textbf{Wigner 函数}。对于\(f(k, \xi)\)不恒等于1的情形,有
    \[
        \Tr{\hat{A}\hat{B}} = \int \dd x \dd p ~ A(x, p) \tilde{B}(x, p)
    \]
    其中定义\(\tilde{B}(x, p)\)的量子相空间是定义\(A(x, p)\)的\textbf{对偶空间},需要满足
    \[
        f(k, \xi) \tilde{f}(-k, -\xi) = 1
    \]

    其他常见的量子相空间密度函数的定义包括 Mehta/Kirkwood 空间、P/Q 空间等。
    前面定义的\(\Tr{\e^{-\beta\hat{H}}\delta(\hat{x} - x')\delta(\hat{p} - p')}\)
    和\(\Tr{\e^{-\beta\hat{H}}\delta(\hat{p} - p')\delta(\hat{x} - x')}\)
    就是Mehta/Kirkwood空间,其对应的\(f(k, \xi)\)分别为\(\e^{-\frac{\ii\hbar k\xi}{2}}\)和\(\e^{\frac{\ii\hbar k\xi}{2}}\)。
    P/Q空间则对于谐振子尤为有用,是通过上升、下降算符定义的。感兴趣的读者可以阅读文献\cite{quantumphasespace}。

    \section{Wigner函数的解析求解}

    本节以谐振子体系为例,展示Wigner函数的解析求解方法。

    不妨从简单的情形出发,先考虑边缘密度\(\rho_X(x_0) = \Tr{ \e^{-\beta \hat{H}}\delta(\hat{x} - x_0)}\)的求算。
    这个求迹运算
    如果在位置的本征空间上展开,会得到传播子\(\braket{x_0 | \e^{-\beta \hat{H}} | x_0}\),需要用虚时路径积分的方法求解。
    不过谐振子是一个特殊的体系,我们可以直接在能量的本征空间上展开:
    \[
        \rho_X(x_0) = \sum_n \braket{n | \e^{-\beta \hat{H}} \delta(\hat{x} - x_0) | n}
    \]
    展开后,有两种思路。一种思路是再插入\(\hat{I} = \dd x \ket{x}\bra{x}\)得到波函数形式,利用Mehler求和公式求算。
    Mehler求和公式是:
    \[
        \sum_{n=0}^{\infty} \frac{\mm{H}_{n}(x) \mm{H}_{n}(y)}{n !}\left(\frac{1}{2} w\right)^{n}
        =\left(1-w^{2}\right)^{-1 / 2} \exp \left[\frac{2 x y w-\left(x^{2}+y^{2}\right) w^{2}}{1-w^{2}}\right]
    \]
    但这种思路需要对特殊函数的性质具有较好的了解,并不方便,并且难以推广到求解联合密度的情形中去。

    因此我们采取第二种思路,即将\(\delta(\hat{x} - x_0)\)展开写,并将e指数上的\(\hat{x}\)用升降算符表示。
    \begin{equation}\label{trace}
        \begin{aligned}
            \Tr{\e^{-\beta \hat{H}}\delta(\hat{x} - x_0)} &= \sum_n \braket{n | \e^{-\beta \hat{H}}\delta(\hat{x} - x_0) | n}
            \\ &= \frac{1}{2\pi} \int_{-\infty}^{\infty} \dd k \sum_n \braket{n |  \e^{-\beta\hbar\omega(n+\frac{1}{2})} \e^{\ii k \left[\frac{\hbar}{2m\omega}(\hat{a}^\dagger + \hat{a}) - x_0\right]}  | n}
            \\ &= \frac{1}{2\pi} \int_{-\infty}^{\infty} \dd k ~  \e^{-\ii k x_0 + \frac{\hbar k^2}{4m\omega}} \sum_n\braket{n | \e^{-\beta\hbar\omega(n+\frac{1}{2})} \e^{c\hat{a}} \e^{c\hat{a}^\dagger}  | n}
        \end{aligned}
    \end{equation}
    
    注意到
    \[
        \e^{c\hat{a}^\dagger}\ket{n} = \sum_{m=0}^{\infty} \frac{c^m}{m!}\sqrt{\frac{(n+m)!}{n!}}\ket{n+m}
    \]
    所以
    \[
        \e^{-\beta\hbar\omega(n+\frac{1}{2})}~ \e^{c\hat{a}}~ \e^{c\hat{a}^\dagger} \ket{n}
         = \sum_{l=0}^{\infty} \sum_{m=0}^{\infty} \frac{c^l}{l!}\sqrt{\frac{(n+m)!}{(n+m-l)!}} \frac{c^m}{m!}\sqrt{\frac{(n+m)!}{n!}} \ket{n+m-l}
    \]
    两边同乘以\(\bra{n}\),利用谐振子数目本征态的正交性,即可得到
    \begin{equation}\label{series}
        \sum_n \braket{n | \e^{-\beta\hbar\omega(n+\frac{1}{2})} \e^{c\hat{a}} \e^{c\hat{a}^\dagger}  | n} = \e^{-\frac{1}{2}\beta\hbar\omega} \sum_{m=0}^\infty \sum_{n=0}^\infty \frac{(n+m)!}{n!(m!)^2} c^{2m} \e^{-n\beta\hbar\omega}
    \end{equation}
    
    在这里需要用到一个结论:函数\((1-z)^{-m}\)在常点\(z = 0\)附近的邻域内的Taylor展开可以通过\((1-z)^{-1} = \sum_{n = 0}^{\infty} z^n\)两端求导得到:
    \[
        \frac{\dd^m}{\dd z^m}(1-z)^{-1} = (-1)^m m! (1-z)^{-m-1} = \sum_{n = m}^\infty \frac{n!}{(n-m)!}z^{n-m} = \sum_{n=0}^\infty \frac{(n+m)!}{n!}z^n
    \]
    所以\[(1-z)^{-m-1} = (-1)^m \sum_{n=0}^{\infty} \frac{(n+m)!}{n!~m!}z^n\]
    所以式(\ref{series})可以写作
    \[
        \e^{-\frac{1}{2}\beta\hbar\omega} \sum_{m=0}^{\infty} (1 - e^{-\beta\hbar\omega})^{-m-1} (-1)^m c^{2m} = \frac{1}{2\sinh \frac{\beta\hbar\omega}{2}} \exp\left[-\frac{\hbar k^2}{2m\omega(1-\e^{-\beta\hbar\omega})}\right]
    \]
    将上述结论代入式(\ref{trace})中,发现结果转化成为高斯积分:
    \[
        \begin{aligned}
        \Tr{\e^{-\beta \hat{H}}\delta(\hat{x} - x_0)} 
        &= \frac{1}{2\pi} \int_{-\infty}^{\infty} \dd k ~\frac{1}{2\sinh \frac{\beta\hbar\omega}{2}} \exp \left[-\ii kx_0 + \frac{\hbar k^2}{4m\omega} - \frac{\hbar k^2}{2m\omega(1-\e^{-\beta\hbar\omega})} \right]
        \\ &=  \frac{1}{2\pi} \int_{-\infty}^{\infty} \dd k ~\frac{1}{2\sinh \frac{\beta\hbar\omega}{2}} \exp \left[-\frac{\hbar k^2}{4m\omega \tanh \frac{\beta\hbar\omega}{2}} - \ii kx_0\right]
        \\ &= \left[\frac{m\omega}{2\pi\hbar\sinh(\beta\hbar\omega)}\right]^{\frac{1}{2}}\exp \left[-\frac{m\omega x_0^2}{\hbar}\tan{\frac{\beta\hbar\omega}{2}}\right]
        \end{aligned}
    \]

    这个方法可以方便地移植到联合密度分布上去。
    \[
    \begin{aligned}
        &\quad \rho_W(x, p) = \Tr{\e^{-\beta \hat{H}}\delta(\hat{x} - x_0, \hat{p} - p_0)} 
        \\ &= \sum_n \braket{n | \e^{-\beta \hat{H}}\delta(\hat{x} - x_0, \hat{p} - p_0) | n}
        \\ &= \frac{1}{4\pi^2} \iint \dd k ~\dd \xi ~\sum_n \braket{n |  \e^{-\beta\hbar\omega(n+\frac{1}{2})} \exp\left\{\ii k \left[\frac{\hbar}{2m\omega}(\hat{a}^\dagger + \hat{a}) - x_0\right] + \ii \xi \left[\ii \sqrt{\frac{\hbar\omega}{2}} (\hat{a}^\dagger - \hat{a}) - p_0 \right]\right\}  | n}
        \\ &= \frac{1}{4\pi^2} \iint \dd k ~\dd \xi ~\exp\left[-\ii k x_0 + \frac{\hbar k^2}{4m\omega} + \frac{\hbar m\omega \xi^2}{4}\right] \sum_n\braket{n | \e^{-\beta\hbar\omega(n+\frac{1}{2})} \e^{c_2\hat{a}} \e^{c_1\hat{a}^\dagger}  | n}
    \end{aligned}
    \] 
    其中\[c_1 = \ii k \sqrt{\frac{\hbar}{2m\omega}} - \xi \sqrt{\frac{\hbar\omega}{2}},\quad c_2 = \ii k \sqrt{\frac{\hbar}{2m\omega}} + \xi \sqrt{\frac{\hbar\omega}{2}}\]

    请读者补足后续计算,验证
    \[
    \rho_W (x, p)
    = \frac{1}{2\pi \hbar \cosh (\frac{\beta\hbar\omega}{2})} \exp\left[-\frac{\beta}{Q}\left(\frac{p_0^2}{2m} + \frac{m\omega^2x_0^2}{2}\right)\right]
    \]
    其中\(Q = \frac{\beta\hbar\omega}{2\tanh \frac{\beta\hbar\omega}{2}}\)是我们引入的量子校正因子。
    可以看到,这个结果从形式上相当于两个边缘分布直接相乘。
    经典情形下,直接取\(\beta\)或者\(\omega\)非常小的极限,这样
    \(\tanh \frac{\beta\hbar\omega}{2}\rightarrow \frac{\beta\hbar\omega}{2}\),上式的指数项可以直接化为经典情形;
    归一化因子的分母中的\(\cosh (\frac{\beta\hbar\omega}{2}) \rightarrow 1\),这也将趋近于经典情形。


    如果对\(\dd x_0 \dd p_0\)进行积分,则很容易发现能够得到配分函数:
    \[
    \begin{aligned}
    \iint \dd x_0 \dd p_0 \Tr{\e^{-\beta \hat{H}}\delta(\hat{x} - x_0)\delta(\hat{p} - p_0)}
    & = \frac{1}{2\pi \hbar \cosh (\frac{\beta\hbar\omega}{2})} \left(\frac{2\pi mQ}{\beta}\right)^{\frac{1}{2}} \left(\frac{2\pi Q}{\beta m \omega^2}\right)^{\frac{1}{2}}
    \\ &= \frac{1}{2\sinh \frac{\beta\hbar\omega}{2}}
    = Z
    \end{aligned}
    \]

    利用Wigner函数,可以方便地解决量子情形下的某些实际问题。
    一般来说,将基于经典密度函数的采样改成基于Wigner函数的采样即可。
    
    \begin{asg}
        数值计算水分子在300 K下的量子振动配分函数,以及其某些物理化学性质,比如键长、键角的期望和涨落等。
        (这是期末大作业的题目之一。提示:利用简正坐标变换,从势能面的Hessian矩阵提取出水分子的简正振动模式。
        平动和转动自由度已经删去。)
    \end{asg}

    \section{Wigner函数与传播子}

    Wigner函数的另一种表达形式如下:
    \[
        \rho_W(x, p) = \int \dd \Delta \left< x + \frac{\Delta}{2} | \e^{-\beta\hat{H}} | x - \frac{\Delta}{2} \right> \e^{\ii p\Delta / \hbar}
    \]
    
    \begin{asg}
        证明这个表达形式和\[\rho_W(x, p) = \Tr{\e^{-\beta \hat{H}}\delta(\hat{x} - x_0, \hat{p} - p_0)}\]的定义是等价的。
        (提示:将这个广义\(\delta\)函数展开成Fourier变换的形式,从中你可以看到平移算符!)
    \end{asg}

    因此,在上式中只要对\(\rho_W(x, p)\)进行Fourier变换,即可求出
    \[
        \left< x + \frac{\Delta}{2} | \e^{-\beta\hat{H}} | x - \frac{\Delta}{2} \right>
    \]
    的结果。令\(x_a = x + \frac{\Delta}{2}, x_b = x - \frac{\Delta}{2}\),即可求得
    \(
        \braket{x_a | \e^{-\beta \hat{H}} | x_b}
    \)
    的形式。
    
    \begin{asg}
        对\(\rho_W(x, p)\)进行Fourier变换,验证
        \[
            \braket{x_a | \e^{-\beta\hat{H}} | x_b} = 
            \sqrt{\frac{m\omega}{2\hbar\sinh(\beta\hbar\omega)}} 
            \exp\left\{\frac{m\omega}{2\hbar\sinh(\beta\hbar\omega)} 
            \left[\cosh(\beta\hbar\omega)(x_a^2 + x_b^2) - 2x_a x_b\right]\right\}
        \]
    \end{asg}
    
    作变换\(\beta \rightarrow -\ii t / \hbar\),即可得到谐振子的\textbf{传播子}。
    传播子的物理意义将在下一章揭晓。


    

    \bibliographystyle{plain}
    \bibliography{ref_quantum_phase_space}
    \chapter{Lagrange力学和量子力学的路径积分形式}

    \section{最小作用量原理与Lagrange力学}        

        在经典力学中, 除了可以用基于
         哈密顿量的Hamilton力学描述力学体系, 还可以使用Lagrange力学作为
         一种完全等效的描述手段. 在探索量子力学的其他的表示方法之前, 
         先回顾一下经典力学中除Hamilton力学之外其他的两种描述力学系统的手段:
          Lagrange力学, 或者Hamilton-Jacobi方程. 
        
        在本讲义的最开始,就已经给出了Hamilton力学中的Hamilton正则方程, 
        并且省略了推导过程. 这里将给出其推导.

        首先需要给出描述经典力学中最为重要的力学原理, \textbf{最小作用量原理}. 
        它完全概括了经典力学体系的运动规律.

        \begin{law}[最小作用量原理]
            力学体系有一个与其运动相关的物理量称为\textbf{作用量}$S$, 
            它是一个洛伦兹标量. 如果一个力学体系在给定时刻$t_1$和$t_2$分别由给定的
            广义坐标$q^{(1)}$与$q^{(2)}$描写, 则该力学系统的作用量$S$可以表达为
            联结这两个位型之间各种可能轨迹的\textbf{泛函}. 
            \begin{equation}
                S[\bm q(t)] = \int_{t_1}^{t_2} L( \bm q(t), \dot{\bm q}(t), t ) \dd t
            \end{equation}
            这里的函数$L(\bm q,\dot{\bm q},t)$称为系统的\textbf{拉格朗日量}, 该力学体系在时刻$t_1$和$t_2$之间联结广义坐标$\bm q^{(1)}$与$\bm q^{(2)}$的真实运动轨迹就是使作用量$S$的一阶变分$\delta S = 0$的轨迹\cite{刘川理力}.
        \end{law}
        这是一个原理, 这意味着它不是通过推到得出来的. 就如同电磁学中的麦克斯韦尔方程式, 
        统计热力学中的各态历经原理一样, 它们的正确性是由它们推论的正确性保证的. 

        接着来讨论有关与泛函和变分的问题. 泛函可以认为是一个函数空间到$\mathbb{R}$的映射
        , 它的自变量是一个函数
        \footnote{
            在这里我并不想给出泛函与泛函微分在数学上严格的定义, 
            读者可以自行寻找相关参考书籍.
            }. 
        假设系统真实运动轨迹是$\bm q_\mathrm{c}(t)$, 显然有
        $\bm q_\mathrm{c}(t_1) = \bm q^{(1)}$, $\bm q_\mathrm{c}(t_2) =\bm q^{(2)}$. 
        考虑一个对真实运动轨迹的一个微小偏离$\delta \bm q(t)$, 
        且在这里我们要求这个微小的偏离满足$\delta \bm q(t_1)=\delta \bm q(t_2)=0$. 
        这个对真实轨迹的微小偏离在数学上称为变分.
        最小作用量原理要求$\delta S = 0$, 那么首先根据定义写下$\delta S$: 
        \begin{equation}
            \delta S = \int_{t_1}^{t_2} L( \bm q_\mathrm{c} + \delta \bm q(t_1) , \dot{\bm q}_\mathrm{c} + \delta \dot{\bm q}(t_1), t ) \dd t - \int_{t_1}^{t_2} L( \bm q_\mathrm{c} , \dot{\bm q}_\mathrm{c}, t ) \dd t
        \end{equation}
        使得$\delta \bm q$为无穷小量, 则可以得到
        \footnote{
            这里使用了爱因斯坦求和规定, 即隐含一个对相同指标(这里是$i$)的求和号. 
            笔者认为这种表达比使用向量表示更不容易出错, 缺点在于指标太多的时候可能没法
            很好地把式子整理得很好看. (但以笔者的微レ存的数理基础, 用向量表示可能会
            推错式子)
            }
        \begin{equation}
            \delta S = \int_{t_1}^{t_2} \dd t \left( \frac{\partial L}{\partial q_i} \delta q_i + \frac{\partial L}{\partial \dot q_i} \delta \dot q_i \right)
        \end{equation}
        注意到, 通过分步积分:
        \begin{equation}
            \frac{\partial L}{\partial \dot q_i} \delta \dot q_i = \frac{\mathrm{d}}{\mathrm{d} t} \left( \frac{\partial L}{\partial \dot q_i} \delta q_i \right) - \left(\frac{\mathrm{d} }{\mathrm{d} t} \frac{\partial L}{\partial \dot q_i} \right) \delta q_i
        \end{equation}
        可以得到:
        \begin{equation}\label{eq:8-1-1}
            \delta S = \left. \frac{\partial L}{\partial \dot q_i} \delta q_i \right|_{t_1}^{t_2} + \int_{t_1}^{t_2} \dd t \left( \frac{\partial L}{\partial q_i} - \frac{\mathrm{d} }{\mathrm{d} t} \frac{\partial L}{\partial \dot q_i} \right) \delta q_i = 0
        \end{equation}
        在端点处我们要求过$\delta \bm q(t_1)=\delta \bm q(t_2)=0$, 
        所以上式中第一项为0. 由于第二项对于任意的$\delta q_i(t)$都必须成立, 
        而且各个$\delta q_i(t)$都是完全独立的变分, 所以唯一的可能就是上式中的括号等于0
        . 于是我们推出了在Lagrange力学中描述运动的方程, Euler-Lagrange方程
        \footnote{
            更广泛地讲, 这个方程给出了很大一类泛函极值问题的解, 其存在是变分原理保证的, 
            不受经典力学框架的限制. 
        }
        :
        \begin{equation}
            \frac{\partial L}{\partial q_i} - \frac{\mathrm{d} }{\mathrm{d} t} \frac{\partial L}{\partial \dot q_i} = 0
            \qquad i = 1,2,\cdots,f.
        \end{equation}
        其中$f$为系统的自由度. 

        拉格朗日量在Lagrange力学中处于中心地位, 它描述了体系全部的经典力学性质. 
        对于非相对论的力学体系, 拉格朗日量可以写为动能减去势能$L = K - V$. 
        在一些坐标系下动能可能有些非常繁杂的表达式, 而势能仅依赖于广义坐标与时间. 
        拉格朗日量的形式如下: 
        \begin{equation}
            L =  \frac{M_{ij}(\bm q)}{2} \dot q_i \dot q_j  - V(\bm q, t)
        \end{equation}
        并定义广义动量为: 
        \begin{equation}
            p_i =\frac{\partial L}{\partial \dot q_i} = M_{ij}(\bm q) \dot q_j
        \end{equation}

        \splitline

        考虑拉格朗日量对时间的全导数(即沿着某条经典运动路径的导数):
        \begin{equation}
            \frac{\mathrm{d} L}{\mathrm{d} t} = \frac{\partial L}{\partial t} + p_i \ddot q_i + \dot p_i \dot q_i = \frac{\partial L}{\partial t} +  \frac{\mathrm{d} (p_i \dot q_i)}{\mathrm{d} t}
        \end{equation}
        如果拉格朗日量不显含时间, 那么定义系统的守恒量$E = p_i \dot q_i - L$是能量. 
        受到系统能量的启发, 对拉格朗日量进行勒让德变换
        \footnote{这里不打算从数学上介绍勒让德变换, 形象地讲这个变换将Lagranage量的一个独立变量
        $\bm{\dot{q}}$替换为了$\bm{p}:=\pdv{L}{\bm{\dot{q}}}$, 这个变换的存在性与动能项的正定性有关. }
        , 定义哈密顿量为$H = p_i \dot q_i - L$. 
        由这个定义可以得出:
        \begin{equation}
            \mathrm{d} H = \dot q_i \mathrm{d} p_i - \dot p_i \mathrm{d} q_i - \frac{\partial L}{\partial t} \mathrm{d} t
        \end{equation}
        与哈密顿量的全微分进行比较, 可以得出:
        \begin{equation}
            \dot{q_{i}} = \pdv{H}{p_i};
            \qquad
            \dot{p_{i}} = -\pdv{H}{q_i}
        \end{equation}
        即哈密顿正则方程.

        \section{Hamilton-Jacobi方程}
        我们有了作用量$S$这个新物理量, 由于最小作用量原理的存在(这意味着这里我们只考虑经典轨迹
        的作用量), $S$只能是初始时刻、位置和结束时刻、位置的函数, 即:
        \begin{equation}
            S_{cl} = S(\bm{p}_0, t_0; \bm{p}_1, t_1)
        \end{equation}
        可以研究一下其对广义坐标$\bm q$的偏导、
        对时间$t$的偏导与全导. 这里的偏导数比较多, 注意弄清楚每个偏导中那些变量是不变的
        . 

        考虑对于广义坐标的$q$的偏导, 这里实际上考虑的是对初始位置$\bm q^{(1)}$与结束为止$\bm q^{(2)}$的偏导, 并且假设联结这两个点的轨迹是经典的真实运动轨迹. 通过式\ref{eq:8-1-1}, 第一项中的$\delta q_i$就相当于微元, 第二项积分为0. 可以得到
        \begin{equation}
            \left( \frac{\partial S}{\partial q_i^{(1)}} \right)_{t, q_i^{(2)}} = - \frac{\partial L}{\partial \dot q_i};
            \qquad
            \left(\frac{\partial S}{\partial q_i^{(2)}}\right)_{t, q_i^{(1)}} = \frac{\partial L}{\partial \dot q_i}
        \end{equation}
        一般选取末端点的偏导, 并且注意到拉格朗日量对广义速度的偏导数就是广义动量. 因此在不引起歧义时一般写为
        \begin{equation}
            \frac{\partial S}{\partial q_i} = p_i
        \end{equation}

        实际上作用量$S$形式上是一个变上限定积分, 因此全导非常容易得到
        \begin{equation}
            S[\bm q(t)] = \int_{0}^{t} L( \bm q(\tau), \dot{\bm q}(\tau), \tau ) \dd \tau
            \qquad \implies \frac{\mathrm{d} S}{\mathrm{d} t} = L
        \end{equation}
        考虑到全导与偏导的关系, 有
        \begin{equation}
            \frac{\mathrm{d} S}{\mathrm{d} t} = \frac{\partial S}{\partial t} + \frac{\partial S}{\partial q_i} \dot{q}_i = \frac{\partial S}{\partial t} + p_i\dot{q}_i
        \end{equation}
        那么根据哈密顿量的定义$H = p_i \dot q_i - L$, 可以看出
        \begin{equation}\label{eq:8-1-2-1}
            \frac{\partial S}{\partial t} = - H
        \end{equation}

        基于上面的结论, 我们发现只使用作用量$S$及其偏导也可以表述经典力学
        \begin{equation}
            \frac{\partial S}{\partial t} + H(\bm q, \frac{\partial S}{\partial \bm q}, t) = 0
        \end{equation}
        进一步展开可以写为
        \begin{equation}\label{eq:8-1-2-2}
            \frac{\partial S}{\partial t} + \frac{M^{-1}_{ij}(\bm q)}{2} \frac{\partial S}{\partial q_i}\frac{\partial S}{\partial q_j} + V(\bm q, t) = 0
        \end{equation}
        $M^{-1}_{ij}$是质量矩阵的逆. 

        以上\ref{eq:8-1-2-1}至\ref{eq:8-1-2-2}三个方程都可以被称为Hamilton-Jacobi方程. 它提供了一种只使用作用量$S$的偏导数表达经典力学体系的运动规律的方法.

    \section{量子力学的路径积分形式}

        \subsection{传播子与路径积分}

        在前文中, 已经通过含时薛定谔方程得到了时间演化算符. 当不同时刻的哈密顿算符对易且不显含时间时
        \begin{equation}
            | \psi(t) \rangle = \mathrm{e}^{-\frac {\mathrm{i}\hat{H}t}{\hslash}} | \psi(0) \rangle
        \end{equation}
        插入位置本征态
        \begin{equation}
            \langle x | \psi(t) \rangle = \int \mathrm{d} y \langle x | \mathrm{e}^{-\frac {\mathrm{i}\hat{H}t}{\hslash}} | y \rangle  \langle y | \psi(0) \rangle
        \end{equation}
        可以看到,$\langle x | \exp[-\mathrm{i}\hat{H}t / \hslash] | y \rangle$具有状态转移概率的物理含义。
        因此,将其定义为\textbf{传播子}。如果求出传播子, 那么就可以求解含时Schrodinger方程.

        首先研究$\mathrm{e}^{\lambda \hat{A}}\mathrm{e}^{\lambda \hat{B}}$和$\mathrm{e}^{\lambda (\hat{A}+\hat{B})}$的关系. 根据Baker–Campbell–Hausdorff公式, 应有
        \begin{equation}
            \mathrm{e}^{\lambda \hat{A}} \mathrm{e}^{\lambda \hat{B}} = \mathrm{e}^{\lambda (\hat{A}+\hat{B}) + \frac 12 \lambda^2 [\hat{A},\hat{B}] + O(\lambda^2)}
        \end{equation}
        如果$\lambda \to 0$, 可以忽略二阶无穷小量, 则有
        \begin{equation}
            \mathrm{e}^{\lambda \hat{A}} \mathrm{e}^{\lambda \hat{B}} = \mathrm{e}^{\lambda (\hat{A}+\hat{B})}
        \end{equation}
        将时间平均分为$N$份, 令$\lambda = \frac tN$, 并令$N \to \infty$. 所以
        \begin{equation}
            \mathrm{e}^{-\frac {\mathrm{i}\hat{H}t}{N\hslash}} = \mathrm{e}^{-\frac {\mathrm{i}\hat{K}t}{N\hslash}} \mathrm{e}^{-\frac {\mathrm{i}\hat{V}t}{N\hslash}}
        \end{equation}
        代入传播子, 得到
        \begin{equation}\begin{aligned}
            \langle x | \mathrm{e}^{-\frac {\mathrm{i}\hat{H}t}{N\hslash}} | y \rangle &= \langle x | \mathrm{e}^{-\frac {\mathrm{i}t}{N\hslash}\frac {\hat{p}^2}{2m}} 
            \mathrm{e}^{-\frac {\mathrm{i}t}{N\hslash}\hat{V}} |y \rangle\\
            &= \langle x | \mathrm{e}^{- \frac{\mathrm{i}t}{N\hslash} \frac{\hat{p}^2}{2m}} |y \rangle \mathrm{e}^{-\frac {\mathrm{i}t}{N\hslash}V(y)} \\
            &= \mathrm{e}^{-\frac {\mathrm{i}t}{N\hslash}V(y)} \int \langle x | \mathrm{e}^{-\frac {\mathrm{i}t}{N\hslash}\frac {\hat{p}^2}{2m}} |p \rangle \langle p |y \rangle \mathrm{d}p \\
            &= \mathrm{e}^{-\frac {\mathrm{i}t}{N\hslash}V(y)} \int \langle x|p \rangle \langle p|y \rangle \mathrm{e}^{-\frac {\mathrm{i}t}{N\hslash}\frac {p^2}{2m}} \mathrm{d}p\\
            &= \frac 1{2\pi \hslash} \mathrm{e}^{-\frac {\mathrm{i}t}{N\hslash}V(y)} \int \mathrm{e}^{\frac {\mathrm{i}(x-y)p}{\hslash}} \mathrm{e}^{-\frac {\mathrm{i}t}{N\hslash}\frac {p^2}{2m}} \mathrm{d} p
        \end{aligned}\end{equation}
        根据Gauss积分
        \begin{equation}
            \int_{-\infty}^{+\infty} \mathrm{e}^{-ax^2+bx} \mathrm{d}x = \sqrt{\frac {\pi}a} \mathrm{e}^{\frac {b^2}{4a}}
        \end{equation}
        由此得到$t/N \to 0$时的传播子为
        \begin{equation}
            \langle x | \mathrm{e}^{-\frac {\mathrm{i}\hat{H}t}{N\hslash}} | y \rangle = \sqrt{\frac {mN}{2\pi\mathrm{i} \hslash t}} \mathrm{e}^{\mathrm{i}\frac {mN(x-y)^2}{2\hslash t}}\mathrm{e}^{-\frac {\mathrm{i}t}{N\hslash} V(y)}
        \end{equation}
        前两项来自于动能算符, 第三项来自于势能算符. 动能算符和势能算符虽然不对易, 但是在$t/N \to 0$时可以得到这个结果. 

        \splitline

        一般情况下的传播子可以通过对无穷短时间的传播子积分得到
        \begin{equation}
            \langle x_0 | \mathrm{e}^{-\frac {\mathrm{i}\hat{H}t}{\hslash}} | x_N \rangle = \left(\frac{mN}{2\pi\mathrm{i}\hslash t}\right)^{N/2} \left(\prod_{n = 1}^{N-1}\int 
            \mathrm{d} x_n \right) \mathrm{e}^{ \frac{\mathrm{i}t}{\hslash N} \sum_{n = 1}^{N} \left[ \frac{m}{2} \left(\frac{x_{n-1} - x_n}{t/N}\right)^2 - V(x_n)\right] }
        \end{equation}
        在$t/N \to 0$的情形下$N(x_{n-1} - x_n)/t = \dot x_n$, 且可将求和号换为积分号. 那么指数上可写为
        \begin{equation}\begin{aligned}
            \frac{\mathrm{i}t}{\hslash N} \sum_{n = 1}^{N} \left[ \frac{m}{2} \left(\frac{x_{n-1} - x_n}{t/N}\right)^2 - V(x_n)\right] 
            &= \frac{\mathrm{i}}{\hslash} \int_0^t \mathrm{d}t \left[ \frac{1}{2} m \dot x^2 - V(x) \right]\\
            &= \frac{\mathrm{i}}{\hslash} \int_0^t \mathrm{d}t L = \frac{\mathrm{i}}{\hslash} S[x(t)]
        \end{aligned}\end{equation}
        作为一种表示方法, 有\cite{费曼量子力学与路径积分}
        \begin{equation}\label{eq:8-2-1}
            \langle x_0 | \mathrm{e}^{-\frac {\mathrm{i}\hat{H}t}{\hslash}} | x_N \rangle 
            = \int_{x_0}^{x_N} \mathcal{D}[x(t)] \mathrm{e}^{ \frac{\mathrm{i}}{\hslash} S[x(t)] }
        \end{equation}
        其中$\mathcal{D}[x(t)]$表示对所有可能的路径进行积分, 同时将归一化因子作为测度包含在内(这个归一化因子应当是与路径无关的). 
        虽然上面似乎给出了归一化因子的一个形式, 但很明显它对于$N \to \infty$不是一个好的定义. 
        实际上不同体系归一化因子并不是一样的, 需要通过实际计算确定. 

        可以发现传播子其实是一种对所有路径的加权平均, 其中所有路径权重的模相等, 但具有与作用量相关的相位. 
        由于作用量的形式与经典作用量一致, 只要能算得出这个积分就能求出传播子. 
        请注意, 这并不意味着$S[x(t)]$的"结果"也会与经典作用量积分的结果$S[\bar x(t)]$一致(用$\bar x(t)$表示经典轨迹), 
        也就是说一般不能算出轨迹的经典作用量直接带入路径积分表达式.

        \subsection{路径积分的准经典近似}

        经典轨迹作用量$S[\bar x(t)]$的表达式相对容易求得, 它反映了经典粒子的运动状态. 
        而在求传播子的时候, 对所有路径求$S[x(t)]$是一件非常困难的事情. 
        如果能直接将传播子表达式中对所有可能的路径积分换成对经典路径的积分, 将会是一个很有用的近似. 
        我们将这个近似称为\textbf{准经典近似}, 用方程表达出来即为下式(式\ref{eq:8-2-1.1}). 
        \begin{equation}\label{eq:8-2-1.1}
            \langle x_i | \mathrm{e}^{-\frac {\mathrm{i}\hat{H}t}{\hslash}} | x_f \rangle \simeq f(t) \mathrm{e}^{ \frac{\mathrm{i}}{\hslash} S[\bar x(t)] } 
        \end{equation}
        问: 准经典近似在什么条件下成立?

        \splitline

        现在我们证明, 对形式如下的拉格朗日量而言, 准经典近似是精确成立的. 
        \begin{equation}\label{eq:8-2-1.2}
            L(x, \dot x, t) = a(t)\dot x^2 + b(t) \dot x x + c(t) x^2 + d(t) \dot x + e(t) x + f(t)
        \end{equation}
        设粒子从点$(x_1, t_1)$运动到$(x_2, t_2)$, 经典轨迹为$\bar x(t)$. 设计一个新变量$y(t) = x(t) - \bar x(t)$, 有$y(t_1) = y(t_2) = 0$. 
        由于$\bar x(t)$是确定的轨迹, 有$\mathcal{D}[x(t)] = \mathcal{D}[y(t)]$. 作用量的积分为
        \begin{equation}
            S[\bar x(t) + y(t)] = S[\bar x(t)] + \int_{t_1}^{t_2} [a(t)\dot y^2 + b(t) \dot y y + c(t) y^2 + \cdots] \mathrm{d}t
        \end{equation}
        其中对$y$或者$\dot y$的一次项放在了省略号里, 这些项积分为0. 传播子可以写为
        \begin{equation}
            \langle x_1 | \mathrm{e}^{-\frac {\mathrm{i}\hat{H}t}{\hslash}} | x_2 \rangle 
            = \mathrm{e}^{ \frac{\mathrm{i}}{\hslash} S[\bar x(t)] } \int_0^0 \mathcal{D}[y(t)] \mathrm{e}^{ 
                \frac{\mathrm{i}}{\hslash} \int_{t_1}^{t_2} [a(t)\dot y^2 + b(t) \dot y y + c(t) y^2] \mathrm{d}t 
                }
        \end{equation}
        现在注意到积分号中的被积函数与经典轨迹$\bar x(t)$无关, 所以其路径积分应当与$x_1$或$x_2$无关; 且轨迹$y(t)$起始并终止于$y=0$, 因此这个路径积分只能是时间的函数. 这意味着传播子可以写为
        \begin{equation}\begin{aligned}
            \langle x_1 | \mathrm{e}^{-\frac {\mathrm{i}\hat{H}t}{\hslash}} | x_2 \rangle = f(t_1, t_2) \mathrm{e}^{ 
                \frac{\mathrm{i}}{\hslash} S[\bar x(t)] 
                } 
        \end{aligned}\end{equation}
        可以看到, 在差一个$t_1$和$t_2$的函数的意义下, 通过准经典近似能准确求得体系的传播子. 
        而一般不管使用的是何种方法, 归一化系数都是通过间接的方法求出来的. 准经典近似非常有用, 只要拉格朗日量具有式\ref{eq:8-2-1.2}的形式, 即使对多粒子体系, 或是拉格朗日量含时的情况, 也能相对容易地算出传播子的具体形式. 

        \subsection{自由粒子的传播子}

        自由粒子体系的势能为0, 所以可以不需要把时间分成$N$份, 而是直接对整个传播子来计算. 把$V=0,N=1$代入上式, 即得到
        \begin{equation}\label{eq:8-2-2}
            \langle x_0 | \mathrm{e}^{-\frac {\mathrm{i}\hat{H}t}{\hslash}} | y_0 \rangle 
            = \sqrt{\frac {m}{2\pi\mathrm{i} \hslash t}} \mathrm{e}^{-\mathrm{i}\frac {m(x_0 - y_0)^2}{2\hslash t}}
        \end{equation}
        如果推广到$F$维体系, 则有
        \begin{equation}
            \langle \bm{x_0} | \mathrm{e}^{-\frac {\mathrm{i}\hat{H}t}{\hslash}} |\bm{y_0} \rangle 
            = \left(\frac {1}{2\pi\mathrm{i} \hslash t} \right)^{\frac F2} |\mb{M}|^{\frac 12} \mathrm{e}^{-\mathrm{i}\frac {(\bm{x_0}-\bm{y_0})^{\mathrm{T}} \mb{M} (\bm{x_0}-\bm{y_0})}{2\hslash t}}
        \end{equation}

        \splitline

        考虑直接通过路径积分得到自由粒子的传播子. 在经典情况下写出自由粒子作用量
        \begin{equation}
            S[x(t)] = \int_0^t \mathcal{L}(x,\dot{x},t') \mathrm{d}t' = \frac 12 \int_0^t m\dot{x}^2 \mathrm{d}t'
        \end{equation}
        Lagrange函数会满足Euler-Lagrange方程. 而对于自由粒子, Lagrange函数不显含坐标, 所以
        \begin{equation}
            \frac {\mathrm{d}}{\mathrm{d}t} (m \dot x) = 0
        \end{equation}
        由此可见, 速度不随时间变化, 且
        \begin{equation}
            \dot{x} = \frac{y_0 - x_0}t
        \end{equation}
        所以, 上述作用量积分的结果为
        \begin{equation}
            S(x_0, y_0) = \frac{m(y_0 - x_0)^2}{2t}
        \end{equation}
        已证得对于自由粒子准经典近似是精确成立的, 利用传播子的准经典近似(式\ref{eq:8-2-1.1})很容易算出传播子的表达式
        \begin{equation}
            \langle x_0 | \mathrm{e}^{-\frac {\mathrm{i}\hat{H}t}{\hslash}} | y_0 \rangle 
            = \int \mathcal{D}[x(t)] \mathrm{e}^{ \frac{\mathrm{i}}{\hslash} S[x(t)] } = C(t) \mathrm{e}^{\frac {\mathrm{i}}{\hslash} \frac{m(y_0 - x_0)^2}{2t}}
        \end{equation}
        其中$C(t)$是归一化因子. 现在希望能把$C(t)$求出, 给定初始条件
        \begin{equation}
            t \to 0,~~~~~\langle x_0|y_0\rangle = \delta(y_0-x_0)
        \end{equation}
        计算出
        \begin{equation}
            \int_{-\infty}^{+\infty} \mathrm{e}^{\frac {\mathrm{i}}{\hslash} \frac {m(y_0 - x_0)^2}{2t}}\mathrm{d}y_0 = \sqrt{\frac {2\pi\mathrm{i}\hslash t}m}
        \end{equation}
        于是
        \begin{equation}\label{eq:8-2-3}
            C(t) = \sqrt{\frac m{2\pi\mathrm{i}\hslash t}}
        \end{equation}
        我们得出了$t \to 0$时的归一化因子. 为了验证在$t \neq 0$ 时归一化系数是正确的, 设传播子为$C(t) D(t) \mathrm{e}^{\frac{\mathrm{i}}{\hslash}S}$, 带回含时薛定谔方程验证, 其中$D(0) = 1$.
        \begin{equation}
            -\mathrm{i}\hslash \frac {\partial}{\partial t} \langle y_0 | \mathrm{e}^{-\frac {\mathrm{i}\hat{H}t}{\hslash}} |x_0 \rangle 
            =  \langle y_0 | \hat{H} \mathrm{e}^{-\frac {\mathrm{i}\hat{H}t}{\hslash}} |x_0 \rangle 
            = -\frac {\hslash^2}{2m} \frac {\partial^2}{\partial y_0^2} \langle y_0 | \mathrm{e}^{-\frac {\mathrm{i}\hat{H}t}{\hslash}} |x_0 \rangle
        \end{equation}
        展开之后可以得到\(D'(t) = 0\), 所以\(D(t)=1\)是一个常数函数. 
        \begin{equation}\begin{aligned}
            \left[i\hslash D'(t) C(t) - \frac{i\hslash}{2t} D(t) C(t) + \frac{m(x_0-y_0)^2}{2t^2} D(t) C(t) \right] \mathrm{e}^{\frac {\mathrm{i}}{\hslash} S} \\
            = \left[- \frac{i\hslash}{2t} D(t) C(t) + \frac{m(x_0-y_0)^2}{2t^2} D(t) C(t)\right] \mathrm{e}^{\frac {\mathrm{i}}{\hslash} S}
        \end{aligned}\end{equation}
        由此可以确定自由粒子的传播子, 结果与式\ref{eq:8-2-2}相吻合. 

        \subsection{其他体系的传播子}

        即使不是自由体系也可以利用式\ref{eq:8-2-1}进行计算, 但能计算出解析形式传播子的体系非常少. 
        目前, 传播子可以写出解析表达式的束缚态有: 自由粒子、无限深方势阱、谐振子、Morse势、类氢原子等\cite{Grosche_HFPI}\cite{Kleinnert_PIQMSPP}. 

        谐振子体系是一个可以求出解析形式传播子的体系, 读者可以尝试计算一下谐振子的传播子作为练习\footnote{用准经典近似很简单, 读者也可以尝试下不用准经典近似怎么做}. 结果为
        \begin{equation}
            \langle x | \mathrm{e}^{-\frac {\mathrm{i}}{\hslash}\hat{H}_\mathrm{HO}t} | y \rangle 
            = \sqrt{\frac{m\omega}{2\mathrm{i}\pi\hslash\sin(\omega t)}} \exp\left\{ \frac{\mathrm{i}}{\hslash} \frac{m\omega}{2\sin(\omega t)}\left[ (x^2 + y^2)\cos(\omega t) - 2xy \right] \right\} \\
        \end{equation}

        有关这方面的内容, 可以参考\cite{费曼量子力学与路径积分}\cite{谷村吉隆化学物理入门}.

    \section{虚时路径积分}

        \subsection{路径积分分子动力学}

        注意到量子体系下的Boltzmann分布为
        \begin{equation}
            \mathrm{e}^{-\beta \hat{H}} = \sum_n \mathrm{e}^{-\beta E_n} |\phi_n \rangle \langle \phi_n|
        \end{equation}
        类似地, 可以作和
        \begin{equation}
            \mathrm{e}^{-\frac {\mathrm{i}\hat{H}t}{\hslash}} = \sum_n \mathrm{e}^{-\frac {\mathrm{i}E_n t}{\hslash}} |\phi_n \rangle \langle \phi_n|
        \end{equation}
        我们发现上述的两个式子有一定的相似性. 两者的区别在于指数上与哈密顿量相乘的标量一个是虚数一个是实数. 
        如果将$\beta$对应成时间$t$, 则可称为\textbf{虚时间}. 对应关系为
        \begin{equation}
            t \leftrightarrow -\mathrm{i}\hslash \beta
        \end{equation}
        显然地, 高温对应虚时间的短时, 低温对应虚时间的长时. 同样可求出虚时间下的传播子
        \begin{equation}
            \langle x|\mathrm{e}^{-\beta \hat{H}}|y\rangle
        \end{equation}
        求出配分函数
        \begin{equation}\begin{aligned}
            Z &= \mathrm{Tr}(\mathrm{e}^{-\beta \hat{H}}) \\
            &= \sum_n \langle n| \mathrm{e}^{-\beta \hat{H}}|n\rangle \\
            &= \int \langle x|\mathrm{e}^{-\beta \hat{H}}|x\rangle \,\mathrm{d}x
        \end{aligned}\end{equation}
        在路径积分的语言下可以放弃态的概念, 也不需要有波函数, 只要有传播子就有配分函数, 有了配分函数就得到了体系所有的热力学性质. 
        要求这个积分中的传播子, 将$\beta$分为$P$份. 在$P \to \infty$时, 有
        \begin{equation}\begin{aligned}
            Z &= \int \mathrm{d} x^{(1)} \cdots \mathrm{d} x^{(P)} \left(\frac{mP}{2\pi \hslash^2 \beta}\right)^{P/2} \\ & \qquad\times\left.
            \exp\left( -\beta \sum_{k=1}^{P} \left[\frac{mP(x^{(k+1)}-x^{(k)})^2}{2\hslash^2 \beta^2}+\frac{1}{P}V(x^{(k)})\right]\right)\right|_{x_1 = x_{P+1}}
        \end{aligned}\end{equation}
        其中$x_1 = x_{P+1}$. 发现动能项的形式类似于一个谐振子, 可以定义$\omega_P = \frac P{\hslash \beta}$. 
        指数上的部分看作$P$个点组成的环两两用弹簧连接, 且每个点都额外受外力作用. 它可以写成一个等效势能$V_\mathrm{eff}$:
        \begin{equation}
            Z = \langle x|\mathrm{e}^{-\beta \hat{H}}|x\rangle = C(\beta) \int \mathrm{e}^{-\beta V_\mathrm{eff}(\bm{x})} \mathrm{d}\bm{x}
        \end{equation}
        这无疑是一个高维积分, 可以通过Monte-Carlo算法计算, 这种做法称为\textbf{路径积分蒙特卡洛(Path integral Monte Carlo, PIMC)}. 
        除此之外也可以在这里插入一个关于“动量”的积分
        \begin{equation}\begin{aligned}
            Z &= \left(\frac{mP}{2\pi \hslash^2 \beta}\right)^{P/2} \left(\frac{\beta}{2\pi m}\right)^{P/2} \left( \prod_{k=1}^{P} \int \mathrm{d} x^{(k)} \mathrm{d} p^{(k)} \right) \\ & \qquad\times\left.
            \exp\left( -\beta \sum_{k=1}^{P} \left[\frac{(p^{(k)})^2}{2m}+\frac{mP(x^{(k+1)}-x^{(k)})^2}{2\hslash^2 \beta^2}+\frac{1}{P}V(x^{(k)})\right]\right)\right|_{x_1 = x_{P+1}} \\
            &= \left( \prod_{k=1}^{P} \int \frac{\mathrm{d} x^{(k)} \mathrm{d} p^{(k)}}{2\pi\hslash}  \right) \mathrm{e}^{-\beta H_\mathrm{cl}}
        \end{aligned}\end{equation}
        这里我们得到了一个有效哈密顿量$H_\mathrm{cl}$, 这个哈密顿量描述一个经典体系. 该体系的动量是人为的、没有意义
        \footnote{这意味你可以任意设定动量项中的质量, 质量矩阵没有必要是对角的, 更没有必要等于$m$. 当然让它等于$m$会让公式显得简洁一些}, 
        但可以通过其位型空间的分布得到对应量子体系的配分函数. 
        \begin{equation}
            H_\mathrm{cl} = \left. \sum_{k=1}^{P} \left[\frac{(p^{(k)})^2}{2m}+\frac{mP(x^{(k+1)}-x^{(k)})^2}{2\hslash^2 \beta^2}+\frac{1}{P}V(x^{(k)})\right] \right|_{x_1 = x_{P+1}}
        \end{equation}

        因此,此时体系满足的运动方程为
        \begin{equation}
            \begin{gathered}
            \dot{x}_{k}=\frac{p_{k}}{m^{\prime}} \\
            \dot{p}_{k}=-m \omega_{p}^{2}\left(2 x_{k}-x_{k+1}-x_{k-1}\right)-\frac{1}{P} \frac{\partial V}{\partial x_{k}}
            \end{gathered}
        \end{equation}

        因此,可以通过分子动力学方法来有效地对其位型空间的分布进行采样\footnote{
            实际上上述方程在应用时会出现难以收敛的问题。需要对坐标进行staging transform以进行运动的解耦合。
            详见Tuckerman 第4章、第12章。
        }, 这种方法称为
        \textbf{路径积分分子动力学(Path Integral Molecular Dynamics, PIMD)}. 
        如果$P=1$, 上式得到的结果就是经典统计力学的结果, 这体现了量子力学与经典统计力学的同构.

        特别的, 如果一个算符$\hat{A} = A(\hat x)$只是位置的函数, 那么
        \begin{equation}\begin{aligned}
            \langle {\hat A} \rangle = \left( \prod_{k=1}^{P} \int \frac{\mathrm{d} x^{(k)} \mathrm{d} p^{(k)}}{2\pi\hslash}  \right)  \left[ \frac1P\sum_{k=1}^{P}A(x^{(k)}) \right] \mathrm{e}^{-\beta H_\mathrm{cl}}
        \end{aligned}\end{equation}
        上式实际上就是$H_\mathrm{cl}$描述的经典体系中$A(x)$平均值的系综平均.

        \subsection{量子Monte-Carlo}

        考虑含时Schodinger方程:
        \begin{equation}
            \mathrm{i}\hslash \frac {\partial}{\partial t} | \psi(t) \rangle = \hat{H}|\psi(t) \rangle
        \end{equation}
        将其改写为关于$\beta$的方程:
        \begin{equation}
            -\frac {\partial}{\partial \beta} |\psi(\beta) \rangle = \hat{H} |\psi(\beta) \rangle
        \end{equation}

        任意给定一个初态, 可以写成Hamilton函数的本征函数的线性组合
        \begin{equation}
            |\psi(0) \rangle = \sum_n c_n |\phi_n\rangle
        \end{equation}
        并且
        \begin{equation}
            |\psi(\beta) \rangle = \mathrm{e}^{-\beta \hat{H}} |\psi(0) \rangle = \sum_n c_n\mathrm{e}^{-\beta E_n}|\phi_n \rangle 
            = \mathrm{e}^{-\beta E_0} \sum_n c_n \mathrm{e}^{-\beta(E_n-E_0)} |\phi_n \rangle
        \end{equation}
        因此, 当$\beta \to \infty$时, 得到的就是基态$|\phi_0 \rangle$. 基于此开发出了\textbf{量子蒙特卡罗(Quantum Monte Carlo, QMC)}算法
        \footnote{QMC实际上是一系列量子力学和随机数方法结合产生的算法, 这里提到的算法应该叫做扩散蒙特卡罗(Diffusion Monte Carlo, DMC)算法}. 
        如果想求第一激发态, 只需要设计投影算符$\hat P = \hat I - |\phi_0 \rangle\langle \phi_0|$在迭代过程中不断将基态除去, 迭代就会收敛到第一激发态上. 

    \bibliographystyle{plain}
    \bibliography{ref_lagrange}
    \chapter{量子力学的轨线形式——Bohm动力学}


    我们已经讨论过经典力学中的Hamilton力学和Lagrange力学,但是经典力学的基石仍然是Newton力学。
    Newton力学是基于轨线\(x = x(t)\)的力学形式。
    在量子力学中,我们已经找到了与Hamilton力学、Lagrange力学和经典统计力学的对应,
    它们分别是量子力学的算符形式(包括Schrodinger、Heisenberg表象)、Feynman路径积分形式和量子相空间形式(Wigner等)。
    我们现在希望找到某种量子力学的轨线形式。

    \section{连续性方程}

        类比经典情况, 若存在守恒量
        \begin{equation}
            \int \rho (\bm{x},t) \mathrm{d} \bm{x} = \text{const.}
        \end{equation}
        且局部没有粒子的产生与湮灭, 则有连续性方程
        \begin{equation}
            \frac {\partial \rho}{\partial t} + \bm{\nabla} \cdot (\rho \bm{v}) = 0
        \end{equation}
        可以将守恒量的方程两侧对时间求导, 可以得到一个在积分号下的连续性方程. 
        \begin{equation}\begin{aligned}
            \frac{\mathrm{d}}{\mathrm{d}t} \int_V \rho (\bm{x}_t, t) \mathrm{d} \bm{x}_t
            & = \int_V \bigg( \frac{\mathrm{d}\rho}{\mathrm{d}t} \mathrm{d} \bm{x}_t + \rho \frac{\mathrm{d}}{\mathrm{d}t} \mathrm{d} \bm{x}_t\bigg) \\
            & = \int_V \bigg( \frac{\partial \rho}{\partial t} + \frac{\partial \rho}{\partial \bm{x}_t}\bm{\dot{x}}_t + \rho \frac{\partial \bm{\dot{x}}_t}{\partial \bm{x}_t} \bigg) \mathrm{d} \bm{x}_t \\
            & = \int_V \bigg( \frac {\partial \rho}{\partial t} + \bm{\nabla} \cdot (\rho \bm{v}) \bigg) \mathrm{d} \bm{x}_t \\
            & = 0 
        \end{aligned}\end{equation}
        如果要求局部没有粒子的产生与湮灭, 该积分可以在任意小的体积元内成立, 则微分形式的连续性方程也成立. 

        在非相对论量子力学中粒子数守恒, 因此上述连续性方程依然成立. 
        \begin{equation}
            \int |\psi(\bm{x},t)|^2 \mathrm{d}x = 1
        \end{equation}

    \section{玻姆动力学}

        考虑含时Schrodinger方程
        \begin{equation}
            \mathrm{i}\hslash \frac {\partial}{\partial t} \psi(\bm{x},t) = \bigg( - \frac {\hslash^2}{2m}\bm{\nabla}^2 + V(\bm{x})\bigg) \psi(\bm{x},t)
        \end{equation}
        如果我们将波函数写成如下形式, 其中$\rho(\bm{x},t)$与$S(\bm{x},t)$是两个实函数, 且$\rho(\bm{x},t)$非负. 在后面我们会发现$\rho(\bm{x},t)$具有概率密度的意义, $S(\bm{x},t)$具有作用量的意义. 
        \begin{equation}
            \psi(\bm{x},t) = \sqrt{\rho(\bm{x},t)} \mathrm{e}^{\frac {\mathrm{i}S(\bm{x},t)}{\hslash}}
        \end{equation}
        将波函数的形式代入含时Schrodinger方程. 对比方程两侧虚部与实部, 可以得到连续性方程以及Hamilton-Jacobian方程
        \begin{equation}\begin{aligned}
            &\frac {\partial \rho}{\partial t} + \bm{\nabla} \cdot (\rho \bm{v}) = 0 \\
            &\frac {1}{2m} \left(\frac {\partial S}{\partial x} \right)^2 + V(\bm{x}) - \frac {\hslash^2}{2m} \frac {\nabla^2 \sqrt{\rho}}{\sqrt{\rho}} = -\frac {\partial S}{\partial t}
        \end{aligned}\end{equation}
        其中连续性方程中的速度场为
        \begin{equation}
            \bm{v} = \frac 1m \frac {\partial S}{\partial \bm{x}}
        \end{equation}
        Hamilton-Jacobian方程前两项可以和经典情况类比, 第三项是由于量子效应产生的, 定义为\textbf{量子势}
        \begin{equation}
            Q(\bm{x},t) = - \frac {\hslash^2}{2m} \frac {\nabla^2 \sqrt{\rho}}{\sqrt{\rho}}
        \end{equation}

        从Hamilton-Jacobi方程出发, 可以得到运动方程
        \begin{equation}
            \frac 12 \bigg(\frac {\partial S}{\partial \bm{x}_t}\bigg)^{\mathrm{T}} \mb{M}^{-1} \frac {\partial S}{\partial \bm{x}_t} + V(\bm{x}) + Q(\bm{x}, t) = -\frac {\partial S}{\partial t}
        \end{equation}
        考虑作用量的全导
        \begin{equation}
            \frac {\mathrm{d}S}{\mathrm{d}t} = \frac {\partial S}{\partial t} + \frac {\partial S}{\partial \bm{x}_t} \dot{\bm{x}_t} = \frac 12 \bigg(\frac {\partial S}{\partial \bm{x}_t}\bigg)^{\mathrm{T}} \mb{M}^{-1} \frac {\partial S}{\partial \bm{x}_t} - V(\bm{x}) - Q(\bm{x}, t)
        \end{equation}
        之后可以对位置求偏导(即对方程两侧求${\partial} / {\partial \bm{x}_t}$), 最终总结得到以下第二个式子. 配以动量的定义, 可以得到$\bm{x}$与$\bm{p}$的演化方程. 
        \begin{equation}\begin{aligned}
            \dot{\bm{x}}_t &= \mb{M}^{-1} \frac {\partial S(\bm{x}_t,t)}{\partial \bm{x}_t} \\
            \dot{\bm{p}}_t &= -\frac {\partial V(\bm{x}_t)}{\partial \bm{x}_t} - \frac {\partial Q(\bm{x}_t, t)}{\partial \bm{x}_t}
        \end{aligned}\end{equation}
        这称为\textbf{量子轨线方程}. 如果对$\rho$求全导
        \begin{equation}
            \frac {\mathrm{d}\rho}{\mathrm{d}t} = \frac {\partial \rho}{\partial t} + \frac {\partial \rho}{\partial \bm{x}_t} \dot{\bm{x}}_t = -\rho \bm{\nabla} \cdot \dot{\bm{x}}_t
        \end{equation}
        由以上方程, 可以通过直接演化量子轨线计算量子体系随时间演化的问题. 但计算速度场的散度在数值上无疑是十分困难的. 

        从量子轨线方程出发,还可以证明量子轨线之间一定不相交。这是量子轨线和经典轨线最重要的不同之一。

        \begin{asg}
            证明量子轨线之间一定不相交。
        \end{asg}

        \bibliographystyle{plain}
        \bibliography{ref_bohmian}
    \chapter{时间关联函数与光谱}
    \section{物理量的期望值与时间关联函数}

    简正坐标下的Hamilton量为:
    \begin{equation}
        H = \frac 12 \bm{P}^\mathrm{T}\bm{P} + \frac 12 \bm{Q}^\mathrm{T} \bm{\Omega Q}
    \end{equation}
    系统热平衡时满足Boltzmann分布, 配分函数为:
    \begin{equation}
        Z = \int \mathrm{e}^{-\beta H} \mathrm{d}\bm{Q}\mathrm{d}\bm{P} = \bigg(\frac {2\pi}{\beta}\bigg)^N \frac 1{\det \bm{\Omega}}
    \end{equation}
    量子化
    \footnote{
        这里的“量子化”并不是量子力学中的“量子化”;我们非常希望通过没有量纲的系综密度函数
        定义一个\textbf{没有量纲}的配分函数,但是对相空间积分时会产生量纲,而且这个量纲会随着系统的
        维数变化。可以看出$\dd x\dd p$具有作用量量纲,这提示我们存在一个具有作用量量纲的
        常数,在量子统计的框架下,可以证明这个常数就是Planck常数$h$。
    }
    以后得到的结果是:
    \begin{equation}
        Z = \frac 1{(\beta \hslash)^N \det \bm{\Omega}}
    \end{equation}
    物理量$B(\bm{Q}, \bm{P})$的期望表示为:
    \begin{equation}
        \langle B \rangle = \frac {\int B(\bm{Q,P})\mathrm{e}^{-\beta H} \mathrm{d}\bm{Q}\mathrm{d}\bm{P}}{\int \mathrm{e}^{-\beta H} \mathrm{d}\bm{Q}\mathrm{d}\bm{P}}
    \end{equation}
    在$t$时刻也可以写出$B(\bm{Q}, \bm{P})$的期望:
    \footnote{我们已经知道Boltzmann分布是稳定分布,不会随时间演化,但是对于
    任意的系综密度,其应该按照Liouville方程演化,因此$t$时刻的系综密度分布
    有可能不同于0时刻}
    :
    \begin{equation}
        \langle B(t) \rangle = \frac {\int B(\bm{Q,P}) \rho_t(\bm{Q,P}) \mathrm{d}\bm{Q}\mathrm{d}\bm{P}}{\int \rho_t(\bm{Q,P}) \mathrm{d}\bm{Q}\mathrm{d}\bm{P}}
    \end{equation}
    根据Liouville定理,可以将$t$时刻系综密度表达为积分形式:
    \begin{equation}
        \rho_t (\bm{Q,P}) = \int \rho_0(\bm{Q}_0, \bm{P}_0)\delta(\bm{Q}_0-\bm{Q}_t(\bm{Q}_0, \bm{P}_0)) \delta(\bm{P}_0-\bm{P}_t(\bm{Q}_0, \bm{P}_0)) \mathrm{d}\bm{Q}_0\mathrm{d}\bm{P}_0
    \end{equation}
    其中$\bm{Q}_t(\bm{Q}_0, \bm{P}_0)$表示从$(\bm{Q}, \bm{P})$出发的点,按照Hamilton方程演化$t$时间到达的点。
    代入物理量期望的表达式:
    \begin{equation}
        \langle B(t) \rangle = \frac {\int B(\bm{Q}_t,\bm{P}_t) \rho_0(\bm{Q}_0,\bm{P}_0) \mathrm{d}\bm{Q}_0\mathrm{d}\bm{P}_0}{\int \rho_0(\bm{Q}_0,\bm{P}_0) \mathrm{d}\bm{Q}_0\mathrm{d}\bm{P}_0}
    \end{equation}
    可以发现只要知道初始时刻的分布$\rho_0(\bm{Q}_0,\bm{P}_0)$和系统的演化轨迹$(\bm{Q}_t,\bm{P}_t)$, 就可以求得$t$时刻的物理量, 而无需演化体系的分布. 

    如果假设系统的分布是平衡分布(比如Boltzmann分布就是平衡分布), 即:
    \begin{equation}
        \frac {\partial \rho}{\partial t} = \{ H, \rho\} = 0
    \end{equation}
    则有$\rho_0(x,p) = \rho_{t'}(x, p) = \rho_{\mathrm{eq}}(x, p)$, 于是有
    \begin{equation}
        \langle B(t) \rangle = \langle B(0) \rangle
    \end{equation}

    \splitline

    定义\textbf{两点时间关联函数}:
    \begin{equation}
        \langle A(0)B(t) \rangle = \int \rho_0(\bm{x}_0,\bm{p}_0) A(\bm{x}_0,\bm{p}_0) B(\bm{x}_t,\bm{p}_t) \mathrm{d}\bm{x}_0 \mathrm{d}\bm{p}_0
    \end{equation}
    注意这里$\bm{x}_t$,$\bm{p}_t$均是$\bm{p}_0$与$t$的函数(就是Hamilton方程的届解).
     $A(\bm{x}_0,\bm{p}_0)$与$B(\bm{x}_t,\bm{p}_t)$可以是相同的物理量. 如果它们相同, 
     那么被称作自关联函数(Autocorrelation Function, ACF). 
     下文中为了表达的简洁将在一维下讨论问题, 如果需要高维下的表达, 只需将对应物理量换为矢量即可.

    由于Liouville定理成立
    \footnote{下面第二个等号用到了Liouville定理$\rho_0(x_0, p_0) = \rho_t(x_t, p_t)$}
    , 时间关联函数中两个物理量位置交换顺序并不会产生任何变化:
    \begin{equation}
        \begin{aligned}
            \langle A(0)B(t) \rangle &= \int \rho_0(x_0,p_0) A(x_0,p_0) B(x_t(x_0,p_0),p_t(x_0,p_0)) \mathrm{d}x_0 \mathrm{d}p_0\\
            &= \int \rho_t(x_t(x_0, p_0),p_t(x_0, p_0)) A(x_0,p_0) B(x_t(x_0,p_0),p_t(x_0,p_0)) \mathrm{d}x_0 \mathrm{d}p_0\\
            &= \langle B(t)A(0) \rangle
        \end{aligned}
    \end{equation}
    现在分析$\langle A(0)B(t) \rangle$和$\langle A(-t)B(0) \rangle$的关系. 
    因为积分变量是哑变量,可以利用Hamilton方程的解对积分换元
    \footnote{利用Hamilton方程的解所定义的不同时刻相空间之间的映射}
    ,同时利用Liouville定理可以得到:
    \begin{equation}
        \begin{aligned}
            \langle A(0)B(t) \rangle &= \int \rho_0(x_0,p_0) A(x_0,p_0) B(x_t(x_0,p_0),p_t(x_0,p_0)) \mathrm{d}x_0 \mathrm{d}p_0\\
            &= \int \rho_0 (x_{-t},p_{-t}) A(x_{-t},p_{-t}) B(x_t(x_{-t},p_{-t}),p_t(x_{-t},p_{-t})) \mathrm{d}x_{-t} \mathrm{d}p_{-t}
        \end{aligned}
    \end{equation}
    其中, 第二步作了变量替换:
    \footnote{这个变量替换的含义是将0时刻的相空间映射为$-t$时刻的相空间,可以形象地
    理解为沿着某条相空间中的轨线,从0时刻地点走到$-t$时刻的点}
    \begin{equation}
        x_0 \to x_{-t}
    \end{equation}
    $x_t(x_0,p_0)$是初始时间为0时演化$t$时间的结果, 
    将$x_{-t}$演化$t$时间后为$x_0$. 如果假设系统的分布是一个\textbf{平衡分布}
    $\rho_0(x,p) = \rho_{-t}(x, p)$, 那么
    \footnote{
        这里的记号可能会引起混淆,$(x_0, p_0)$的意义是某条轨线上0时刻的点,
        是$(x_{-t}, p_{-t})$的函数,但是不能写为$x_0(x_{-t}, p_{-t})$,
        因为在我们对于记号的规定中$x_{t'}(x_t, p_t)$表示$t$时刻的点演化了$t'$
        时间后到达的点,那么$x_0(x_{-t}, p_{-t})$表示的就是$x_{-t}$.
    }
    :
    \begin{equation}
        \begin{aligned}
            \langle A(0)B(t) \rangle &= \int \rho_{\color{red}{0}}(x_{-t},p_{-t}) A(x_{-t},p_{-t}) B(x_t(x_{-t},p_{-t}),p_t(x_{-t},p_{-t})) \mathrm{d}x_{-t} \mathrm{d}p_{-t}\\
            &= \int \rho_{\color{red}{-t}}(x_{-t},p_{-t}) A(x_{-t},p_{-t}) B(x_0,p_0) \mathrm{d}x_{-t} \mathrm{d}p_{-t}\\
            &= \langle A(-t)B(0) \rangle
        \end{aligned}
    \end{equation}

    实际上如果分布为平衡分布, 则时间关联函数具有时间平移不变性. 
    \begin{equation}
        \langle A(0)B(t) \rangle = \langle A(t')B(t+t') \rangle
    \end{equation}
    \section{时间关联函数与光谱}

    现在开始研究一些光谱的性质. 设红外光谱为$I(\omega)$
    , 其中$\omega$为吸收光的频率, 让分子不转动, 则得到的红外光谱为分立的线. 
    对红外光谱做Fourier变换, 得到: 
    \begin{equation}
        f(t) = \int I(\omega)\mathrm{e}^{\mathrm{i}\omega t}\mathrm{d}\omega 
    \end{equation}
    它反映了分子的动力学性质. 现在问, $f(t)$这个函数是什么?
    我们可以从经典和量子的两个角度导出$f(t)$这个函数, 并证明它是某个物理量的自关联函数. 

    \subsection{经典光谱}
    从经典的角度上来说, 光谱$I(\omega)$可以写作:
    \begin{equation}
        \begin{aligned}
            I(\omega) 
            &= \lim_{T\to\infty} \frac 1T \left\langle \left| \int_{-T/2}^{T/2} \mathrm{d}t\, B(t) \mathrm{e}^{-\mathrm{i}\omega t} \right|^2 \right\rangle \\
            &= \lim_{T\to\infty} \frac 1T \int_{-T/2}^{T/2} \mathrm{d}t \int_{-T/2}^{T/2} \mathrm{d}t'\, \langle B(t)B(t') \rangle \mathrm{e}^{-\mathrm{i}\omega (t'-t)} \\
        \end{aligned}
    \end{equation}
    然后进行换元$(t,t')\to(t, \tau=t'-t)$. 随后认为观测光谱时外界对体系的扰动是小的, 
    体系仍可以用平衡分布描述. 因此有$\langle B(t)B(t+\tau) \rangle = \langle B(0)B(\tau) \rangle$. 
    与变量$t$无关. 因此可以得出
    \footnote{可以尝试进行一下这个换元,涉及到了积分上下限的变化}
    : 
    \begin{equation}\begin{aligned}
        I(\omega) 
        &= \lim_{T\to\infty} \frac 1T \int_{-T}^{T} \mathrm{d}\tau\, (T - |\tau|) \langle B(0)B(\tau) \rangle \mathrm{e}^{-\mathrm{i}\omega \tau} \\
        &= \int_{-\infty}^{\infty} \mathrm{d}\tau\, \langle B(0)B(\tau) \rangle \mathrm{e}^{-\mathrm{i}\omega \tau}
    \end{aligned}\end{equation}
    最后一个等号从数学上来说并不一定成立. 但对于“一个足够乐观的物理学家”来说, 可以认为含$\tau$项的积分总是有限的, 因而除以T后该项趋于0. 因此可以得到结论, 光谱是某一物理量自关联函数的Fourier变换.

    如果只考虑一个物理量的自关联函数, 作Fourier变换:
    \begin{equation}\begin{aligned}
    I(\omega) = \int_{-\infty}^{+\infty} \mathrm{e}^{-\mathrm{i}\omega t} \langle B(0)B(t) \rangle \mathrm{d}t
    \end{aligned}\end{equation}
    令$t \to -s$, 则
    \begin{equation}\begin{aligned}
    I(\omega) 
    &= -\int_{+\infty}^{-\infty} \mathrm{e}^{\mathrm{i}\omega s} \langle B(0)B(-s) \rangle \mathrm{d}s
    =  \int_{-\infty}^{+\infty} \mathrm{e}^{\mathrm{i}\omega s} \langle B(0)B(-s) \rangle \mathrm{d}s\\
    &= \int_{-\infty}^{+\infty} \mathrm{e}^{\mathrm{i}\omega s} \langle B(s)B(0) \rangle \mathrm{d}s
    =  \int_{-\infty}^{+\infty} \mathrm{e}^{\mathrm{i}\omega s} \langle B(0)B(s) \rangle \mathrm{d}s\\
    &= I(-\omega)
    \end{aligned}\end{equation}
    所以自关联函数的Fourier变换在频率空间是一个偶函数
    \footnote{本质上是因为经典情况下平衡分布的自关联函数是偶函数}
    . 但这不是真实的情况, 并且在量子力学中不成立.

    \subsection{量子力学下的光谱}

    利用含时微扰理论, 考虑将光中的电磁场作为微扰项引入. 
    光是电磁波, 其中既有交变电场也有交变磁场. 那么电磁波与物质相互作用的时候, 是以哪种场的作用为主呢? 
    这个问题的答案可以从洛伦兹力的形式中看出来
    \begin{equation}
        \bm{F} = q (\bm{E} + \bm{v} \times \bm{B})
    \end{equation}
    注意到自然单位制中$|\bm{E}(\bm r,t)| = c|\bm{B}(\bm r,t)|$. 由于外层电子的运动速度$v \ll c$, 所以电场的作用远大于磁场. 一般考虑原子的吸收与发射都是电偶极跃迁, 微扰项可以写为分子固有偶极与外电场的相互作用
    \begin{equation}
        H' = - \bm{M} \cdot \bm{E} = - (\bm{M} \cdot \bm{e}) E_0 \cos(\omega t)
    \end{equation}
    其中$\bm{e}$为电场偏振方向的单位矢量, $\bm{M} = \sum_i q_i \bm{r}_i$为分子的固有偶极矩. 将跃迁矩阵元记为$P_{i\to f}$, 则由费米黄金规则可以得到(或者你可以在\cite{蒋鸿中物化:含时微扰}找到推导过程)
    \begin{equation}
        P_{i\to f}(\omega) = \frac{\pi E_0^2}{2\hslash^2} \left| \langle f|\bm{M} \cdot \bm{e}|i \rangle \right|^2 \left[ \delta(\omega_{fi} - \omega) + \delta(\omega_{fi} + \omega) \right]
    \end{equation}
    这里的$i$与$f$分别表示跃迁的始末态, 它们都是能量本征态; 符号$\omega_{fi} = (E_f - E_i)/\hslash$. 用$\rho_n$标记体系处于态$n$上的概率\footnote{注意到该公式中能量本征态同时也是态密度算符$\hat\rho$的本征态, 所以这个公式只适用于定态}, 我们可以计算体系吸收的能量
    \begin{equation}\begin{aligned}
        E_\mathrm{abs}(\omega) 
        &= \sum_{i,f}  \hslash\omega_{fi} \rho_i P_{i\to f} \\
        &= \frac{\pi E_0^2}{2\hslash} \sum_{i,f} \omega_{fi} \rho_i \left| \langle f|\bm{M} \cdot \bm{e}|i \rangle \right|^2 \left[ \delta(\omega_{fi} - \omega) + \delta(\omega_{fi} + \omega) \right] \\
        &= \frac{\pi E_0^2}{2\hslash} \sum_{i,f} \omega_{fi} (\rho_i - \rho_f) \left| \langle f|\bm{M} \cdot \bm{e}|i \rangle \right|^2 \delta(\omega_{fi} - \omega)
    \end{aligned}\end{equation}
    这其实是基于一阶含时微扰法导出的定态束缚态系统吸收(或者辐射)的功率谱, 也是实际观察到的光谱. 如果$\rho_n = \frac 1Z \mathrm{e}^{-\beta E_n}$是玻尔兹曼分布, 则有
    \begin{equation}
        E_\mathrm{abs}(\omega) = \frac{\pi \omega E_0^2}{2\hslash} (1 - \mathrm{e}^{-\beta\hslash\omega}) \sum_{i,f} \rho_i \left| \langle f|\bm{M} \cdot \bm{e}|i \rangle \right|^2 \delta(\omega_{fi} - \omega)
    \end{equation}

    我们定义光谱$I(\omega)$\footnote{这里的常数并不重要, 只是让最后的式子好看一点而已}
    \begin{equation}\label{eq:5-1}
        I(\omega) = 3 \sum_{i,f} \rho_i \left| \langle f|\bm{M} \cdot \bm{e}|i \rangle \right|^2 \delta(\omega_{fi} - \omega)
    \end{equation}
    展开$\delta$函数, 可以得到
    \begin{equation}\begin{aligned}
        I(\omega) = 
        &= \frac{3}{2\pi} \sum_{i,f} \rho_i \left| \langle f|\bm{M} \cdot \bm{e}|i \rangle \right|^2 \int_{-\infty}^{\infty} \mathrm{d}t\, \mathrm{e}^{\mathrm{i}\left[(E_f - E_i)/\hslash - \omega\right]t} \\
        &= \frac{3}{2\pi} \int_{-\infty}^{\infty} \mathrm{d}t\, \sum_{i,f} \rho_i \langle i|\bm{M} \cdot \bm{e}|f \rangle \langle f| \mathrm{e}^{\mathrm{i}E_f/\hslash t} \left(\bm{M} \cdot \bm{e}\right) \mathrm{e}^{-\mathrm{i}E_i/\hslash t} |i \rangle \mathrm{e}^{-\mathrm{i} \omega t} \\
        &= \frac{3}{2\pi} \int_{-\infty}^{\infty} \mathrm{d}t\, \sum_{i} \rho_i \langle i|\bm{M} \cdot \bm{e} \left(\sum_{f} |f \rangle \langle f| \right) \left( \mathrm{e}^{\mathrm{i}H_0/\hslash t} \bm{M} \mathrm{e}^{-\mathrm{i}H_0/\hslash t} \right) \cdot \bm{e}  |i \rangle \mathrm{e}^{-\mathrm{i} \omega t} \\
        &= \frac{3}{2\pi} \int_{-\infty}^{\infty} \mathrm{d}t\, \sum_{i} \rho_i \langle i|\bm{M}(0) \cdot \bm{e} \, \bm{M}(t) \cdot \bm{e} |i \rangle \mathrm{e}^{-\mathrm{i} \omega t} \\
    \end{aligned}\end{equation}
    鉴于通常入射光是各向同性的, 需要计算单位立体角内的各向同性光谱, 即对$\bm{e}$在单位球上进行积分并除以$4\pi$. 取向量$\bm{A}$的方向为极轴, 记极坐标中向量$\bm{B} = (B, \psi, 0)$, 利用球面三角中边的余弦公式计算积分
    \begin{equation}\begin{aligned}
        I_{\bm{AB}} 
        &= \int (\bm{A} \cdot \bm{e})(\bm{B} \cdot \bm{e}) \,\mathrm{d} \Omega_{\bm e}\\
        &= \int (A\cos\theta)(B\cos\psi\cos\theta + B\sin\psi\sin\theta\cos\phi ) \sin\theta \,\mathrm{d}\theta \mathrm{d}\phi\\
        &= 2\pi \int_0^{\pi}\bm{A}\cdot\bm{B}\cos^2\theta\sin\theta\,\mathrm{d}\theta
         = \frac{4\pi}{3}\bm{A}\cdot\bm{B}
    \end{aligned}\end{equation}
    由此可得
    \begin{equation}\begin{aligned}
        I(\omega) 
        &= \frac{1}{2\pi} \int_{-\infty}^{\infty} \mathrm{d}t\, \sum_{i} \rho_i \langle i|\bm{M}(0) \cdot \bm{M}(t) |i \rangle \mathrm{e}^{-\mathrm{i} \omega t} \\
        &= \frac{1}{2\pi} \int_{-\infty}^{\infty} \mathrm{d}t\, \langle \bm{M}(0) \cdot \bm{M}(t) \rangle_\rho \mathrm{e}^{-\mathrm{i} \omega t}
    \end{aligned}\end{equation}
    可以看到光谱是分子的偶极矩自关联函数的Fourier变换, 但我们要注意的是这里的自关联函数是量子力学中的自关联函数\footnote{注意这里$\langle (\cdot) \rangle_\rho$表示的是求物理量在分布$\rho$下的系综平均, 而不是量子力学中常见的期望值$\langle (\cdot) \rangle$. 要避免歧义也可以写为$\langle \hat\rho (\cdot) \rangle$}. 特别地, $\bm{M}(0)$与$\bm{M}(t)$不对易. 因此$\langle \bm{M}(0) \cdot \bm{M}(t) \rangle \neq \langle \bm{M}(t) \cdot \bm{M}(0) \rangle$, 因而有$\langle \bm{M}(0) \cdot \bm{M}(t) \rangle \neq \langle \bm{M}(0) \cdot \bm{M}(-t) \rangle$\footnote{对于量子力学中的自关联函数, 在哈密顿不显含时间时$\langle \hat A(0) \hat A(t) \rangle = \langle \hat A(t') \hat A(t'+t) \rangle$仍然成立}. 量力力学中的自关联函数不是偶函数, 因而频谱也不是偶函数.

    我们从光谱的定义式(式\ref{eq:5-1}), 可以得到
    \begin{equation}
        \frac{I(\omega_{fi})}{I(\omega_{if})} = \frac{\rho_i}{\rho_f} = \mathrm{e}^{\beta\hslash\omega_{fi}}
    \end{equation}
    由于$\omega_{fi} = - \omega_{if}$, 所以可以得到
    \begin{equation}
        \mathrm{e}^{-\beta \hslash \omega} I(\omega) = I(-\omega)
    \end{equation}
    这与前面所得到的经典情况下Fourier变换得到的频谱为偶函数的结论并不相同, 称为\textbf{细致平衡原理}. 经典极限下$\hslash \to 0$, 光谱就变成了偶函数. 
    
    \bibliographystyle{plain}
    \bibliography{ref_timecorr}

\end{document}
