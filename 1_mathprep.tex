\chapter{数学准备}

    在开始本书的阅读之前,我们希望读者已经学过了高等数学(或称“微积分”)和线性代数课程,
    能够熟练进行简单的积分计算。

    \section{Gauss积分及其拓展}
        回顾一维Gauss积分
        \footnote{事实上, 这种类型的积分是Euler最先通过将两个高斯函数相乘, 
        再进行极坐标变换算出来的, 按理来说应该称为“Euler积分”。}
        的计算:
        \begin{equation}
            \begin{aligned}
                I &= \int_0^{+\infty} \mathrm{e}^{-ax^2} x^{n} \mathrm{d}x
            \end{aligned}
        \end{equation}
        令$t = ax^2$, 则$\mathrm{d}t = 2ax\mathrm{d}x$
        所以
        \begin{equation}
            \begin{aligned}
                I &= \int_0^{+\infty} \mathrm{e}^{-t} \bigg(\frac ta\bigg)^{\frac n2} \frac {\mathrm{d}t}{\sqrt{at}}\\
                &= \frac 1{2a^{\frac {n+1}2}} \int_0^{+\infty} \mathrm{e}^{-t} t^{\frac {n-1}2} \mathrm{d}t\\
                &= \frac {\Gamma(\frac {n+1}2)}{2a^{\frac {n+1}2}}
            \end{aligned}
        \end{equation}

        利用一维Gauss积分的计算结果可以计算多维的Gauss积分. 
        先举一个最简单的例子,其中应用的思想可以用于计算很多不同种类的Gauss
        积分:
        \begin{equation}
            I = \int \ee^{-\bm{x}^\mathrm{t}\mb{A}\bm{x}}\dd \bm{x}
        \end{equation}
        其中$\bm{A}$是正定实对称矩阵
        \footnote{
            这里必须要求$\mb{A}$是正定的,否则这个Gauss积分不收敛。
        }
        ,根据线性代数中的定理,$\mb{A}$可以在正交
        变换下对角化:
        \begin{equation}
            \mb{D} = \mb{S}^{\mr{t}}\mb{AS}
        \end{equation}
        其中$\mb{S}$为正交矩阵,那么原积分可以表示为:
        \begin{equation}
            \begin{split}
                I &= \int\ee^{-(\mb{S}\bm{x})^{\mr{t}}\mb{S}^{\mr{t}}\mb{AS}\mb{S}^{\mr{t}}\bm{x}}\dd \bm{x}\\
                &= \int\ee^{-(\mb{S}\bm{x})^{\mr{t}}\mb{D}\mb{S}^{\mr{t}}\bm{x}} \dd \bm{x}
            \end{split}
        \end{equation}
        做变量替换$\bm{y} = \mb{S}^{\mr{t}}\bm{x}$, 利用重积分的换元:
        \begin{equation}
            \begin{split}
                I &= \int\ee^{-\bm{y}^{\mr{t}}\mb{D}\bm{y}}\left|\pdv{\bm{x}}{\bm{y}}\right|\dd \bm{y}\\
                &= \int\ee^{-\bm{y}^{\mr{t}}\mb{D}\bm{y}}\left|\mb{S}\right|\dd \bm{y}\\
                &= \prod_{j=1}^{N}\int_{-\infty}^{+\infty}\ee^{-y_j^2D_{jj}}\dd y_j\\
                &= \prod_{j=1}^{N}\sqrt{\frac{\pi}{D_{jj}}}\\
                &= \frac{\pi^{\frac{N}{2}}}{\sqrt{\det{\mb{A}}}}
            \end{split}
        \end{equation}
        稍微复杂点的一个例子, Gauss函数与二次型相乘: 
        \begin{equation}
            I = \int \bm{x}^\mathrm{t}\mb{B}\bm{x} \mathrm{e}^{-\bm{x}^\mathrm{t}\mb{A}\bm{x}}\mathrm{d}\bm{x}
        \end{equation}
        其中$\mb{A}$为正定实对称矩阵, $\mb{B}$为实对称矩阵
        \footnote{
            任意二次型均可以表示为$\bm{x}^{\mr{t}}\mb{B}\bm{x}$, 其中$\mb{B}$为对称矩阵.
        }
        . 仿照上一个例子将$\bm{A}$对角化, 得到:
        \begin{equation}
            \mb{S}^\mathrm{t} \mb{AS} = \mb{D}
        \end{equation}
        所以:
        \begin{equation}
            I = \int \bm{x}^\mathrm{T}\mb{B}\bm{x} \mathrm{e}^{-\bm{x}^\mathrm{t}\mb{SDS}^\mathrm{t}\bm{x}}\mathrm{d}\bm{x}
        \end{equation}
        对重积分进行换元$\bm{y} = \mb{S}^\mathrm{t}\bm{x}$: 
        \begin{equation}
            \mathrm{d}\bm{y} = \det\mb{S}^\mathrm{t}\mathrm{d}\bm{x} = \mathrm{d}\bm{x}
        \end{equation}
        原积分化为
        \begin{equation}
            \begin{aligned}
                I = \int \bm{y}^\mathrm{t}\mb{S}^\mathrm{t}\mb{BS}\bm{y}
                 \mathrm{e}^{-\bm{y}^\mathrm{t}\mb{D}\bm{y}}\mathrm{d}\bm{y}
            \end{aligned}
            \label{change variable}
        \end{equation}
        令$\mb{E} = \mb{S}^\mathrm{t}\mb{BS}$, 于是
        \begin{equation}
            \begin{aligned}
                I &= \int \bm{y}^\mathrm{t}\mb{E}\bm{y} \mathrm{e}^{-\bm{y}^\mathrm{t}\bm{D}\bm{y}}\mathrm{d}\bm{y}\\
                &= \int \sum_{i=1}^{N} \sum_{j=1}^{N} E_{ij} y_i y_j \mathrm{e}^{-\sum_{k=1}^{N} D_{kk} y_k^2} \prod_{k=1}^{N} \mathrm{d}y_k
            \end{aligned}
        \end{equation}
        可以将求和号放到积分外面, 可以分别计算每一个积分, 显然$i \neq j$的项都
        是奇函数对全空间的积分, 得到的结果为0, 只有$i=j$的项会有贡献. 于是积分化简为
        \footnote{请仔细验证一下}:
        \begin{equation}
            \begin{aligned}
                I &= \int \sum_{i=1}^N E_{ii} y_i^2 \mathrm{e}^{-\sum_{k=1}^{N} D_{kk} y_k^2} \prod_{k=1}^{N} \mathrm{d}y_k\\
                &= \prod_{k=1}^{N} \sqrt{\frac {\pi}{D_{kk}}} \sum_{i=1}^{N} \frac {E_{ii}}{2D_{ii}}\\
                &= \frac {\pi^{\frac N2}}{2\sqrt{\det{\mb{D}}}} \mathrm{Tr} (\mb{ED}^{-1})
            \end{aligned}
        \end{equation}
        利用迹的性质进一步化简:
        \begin{equation}
            \begin{aligned}
                \det{\mb{D}} = \det{\mb{S}^{\mr{t}}\mb{A}\mb{S}} = \det{\mb{SS}^{\mr{t}}\mb{A}} = \det{\mb{A}}
            \end{aligned}
        \end{equation}
        类似的: 
        \begin{equation}
            \begin{aligned}
                \mathrm{Tr}(\mb{ED}^{-1}) = \mathrm{Tr}(\mb{S}^\mathrm{t}\mb{BS}\mb{S}^\mathrm{t} \mb{A}^{-1}\mb{T})
                = \mathrm{Tr}(\mb{S}^\mathrm{t}\mb{BA}^{-1}\mb{S})
                = \mathrm{Tr}(\mb{BA}^{-1})
            \end{aligned}
        \end{equation}
        所以:
        \begin{equation}
            \begin{aligned}
                I = \frac {\pi^{\frac N2}}{2\sqrt{\det{\mb{A}}}} \mathrm{Tr} (\mb{BA}^{-1})
            \end{aligned}
        \end{equation}

        接下来尝试计算
        \footnote{注意这里是对于一个矩阵的积分,很容易理解,这相当于对每一个矩阵元积分}
        :
        \begin{equation}
            \begin{aligned}
                \mb{I} = \int \bm{xx}^\mathrm{T} \mathrm{e}^{-\bm{x}^\mathrm{t}\mb{A}\bm{x}}\mathrm{d}\bm{x}
            \end{aligned}
        \end{equation}
        用相同的换元方法得到:
        \begin{equation}
            \begin{aligned}
                \mb{I} = \int \mb{S}\mb{y}\bm{y}^\mathrm{t}\mb{S}^\mathrm{t} \mathrm{e}^{-\bm{y}^\mathrm{t}\mb{D}\bm{y}}\mathrm{d}\bm{y}
            \end{aligned}
        \end{equation}
        计算每个元素: 
        \begin{equation}
            \begin{aligned}
                I_{ij} &= \int (\mb{S}\bm{y})_{i}(\bm{y}^\mathrm{t}\mb{S}^\mathrm{t})_{j} \mathrm{e}^{-\bm{y}^\mathrm{t}\mb{D}\bm{y}}\mathrm{d}\bm{y}\\
                &= \int\sum_{k=1}^{N} S_{ik}y_k \sum_{l=1}^{N} S_{jl} y_l \mathrm{e}^{-\bm{y}^\mathrm{t}\mb{D}\bm{y}}\mathrm{d}\bm{y}\\
                &= \int \sum_{k,l=1}^{N} S_{ik} S_{jl} y_k y_l \mathrm{e}^{-\bm{y}^\mathrm{t}\mb{D}\bm{y}}\mathrm{d}\bm{y}
            \end{aligned}
        \end{equation}
        同样, 只有在$k=l$的项对积分有贡献才有贡献, 那么:
        \begin{equation}
            \begin{aligned}
                I_{ij} &= \int \sum_{k=1}^{N} S_{ik} S_{jk}y_k^2 \mathrm{e}^{-\bm{y}^\mathrm{t}\mb{D}\bm{y}}\mathrm{d}\bm{y}\\
                &= \prod_{l=1}^{N} \sqrt{\frac {\pi}{D_{ll}}} \sum_{k=1}^N S_{ik} S_{jk} \frac 1{2D_{kk}}\\
                &= \frac {\pi^{\frac N2}}{2 \sqrt{\det{\mb{D}}}} (\mb{SD}^{-1}\mb{S}^\mathrm{t})_{ij}\\
                &= \frac {\pi^{\frac N2}}{2 \sqrt{\det{\mb{A}}}} \mb{A}^{-1}_{ij}
            \end{aligned}
        \end{equation}
        因此: 
        \begin{equation}
            \begin{aligned}
                \bm{I} = \frac {\pi^{\frac n2}}{2 \sqrt{\det{\bm{A}}}} \bm{A}^{-1}
            \end{aligned}
        \end{equation}
        利用之前的计算结果,可以得到一个有趣的结论:
        \begin{equation}
            \frac{\int\ee^{-\bm{x}^{\mr{t}}\mb{A}\bm{x}}(\bm{x}\bm{x}^{\mr{t}})\dd \bm{x}}{\int\ee^{-\bm{x}^{\mr{t}}\mb{A}\bm{x}}}\dd \bm{x}
            = \frac{1}{2}\mb{A}^{-1}
        \end{equation}
        最为普遍的一类Gauss积分可以定义为:
        \begin{equation}
            I = \int\ee^{-\bm{x}^{\mr{t}}\mb{A}\bm{x}}g(\bm{x})\dd \bm{x}
        \end{equation}
        其中$g(\bm{x})$可以取
        \footnote{对于最一般的$g(\bm{x})$,此积分也可以计算,感兴趣可以查阅相关资料}
        :
        \begin{equation}
            g(\bm{x}) = 
            \left\{
            \begin{split}
                    &\exp\left[-(\bm{x} - \bm{x_0})^{\mr{t}}\mb{C}(\bm{x} - \bm{x_0})\right]\\
                    &\exp\left[-V_0^{\mr{t}}(\bm{x} - \bm{x_0})\right]\\
                    &(\bm{x}^{\mr{t}}\mb{B}\bm{x})\cdot(\bm{x}^{\mr{t}}\mb{D}\bm{x})
            \end{split}
            \right.
        \end{equation}
        \begin{asg}
            计算正文中提到的Gauss积分。
        \end{asg}

        
        \bibliographystyle{plain}
        \bibliography{ref_mathprep}